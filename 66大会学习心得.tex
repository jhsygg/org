% Created 2013-11-13 三 09:04
\documentclass[11pt]{article}
\usepackage[utf8]{inputenc}
\usepackage[T1]{fontenc}
\usepackage{fixltx2e}
\usepackage{graphicx}
\usepackage{longtable}
\usepackage{float}
\usepackage{wrapfig}
\usepackage{soul}
\usepackage{textcomp}
\usepackage{marvosym}
\usepackage{wasysym}
\usepackage{latexsym}
\usepackage{amssymb}
\usepackage{hyperref}
\tolerance=1000
\providecommand{\alert}[1]{\textbf{#1}}

\title{水钢6.6干部大会学习心得体会}
\author{jhsygg}
\date{\today}
\hypersetup{
  pdfkeywords={},
  pdfsubject={},
  pdfcreator={Emacs Org-mode version 7.8.11}}

\begin{document}

\maketitle

\setcounter{tocdepth}{3}
\tableofcontents
\vspace*{1cm}

2013年6月6日参加了水钢干部大会,会上总经理张新建同志作了题为《亡羊补牢守底线 卧薪尝胆求生存 万众一心 实事求是 向管理要效益的重要讲话》,张槐祥书记作了动员总结发言。
\section{会议的内函及信息}
\label{sec-1}
\subsection{会议主题}
\label{sec-1-1}

参加会议,让我子解到:
所谓“亡羊补牢”是因我们的管理“积淀隐藏的致命顽疾”,用对生产、设备统管变革、“查隐患、反违章、学三规四制、提素质”来弥补我们的管理漏洞。所谓“守住底线”,就是水钢当前的环境下能求得生存权,不被淘汰,要求通过挖潜,万众一心,实事求是,向管理要效益,全力以赴,从改变自我,历练自我,挑战自我,超越自我开始。战胜各种困难,实现最危险的2013年成功突,在取得钢铁行业生存权的底线上,积蓄出打赢生存战的能力和素质,凝练出赢得持久战的精神的文化。
\subsection{当前形势}
\label{sec-1-2}

新建总经理总结到:
宏观上,钢铁企业正面对寒冬,生存空间狭小,正面临淘汰赛。必须思考的是我拿什么去求生存?
企业内部,事故频发,影响深远,暴露出深层次的基础管理问题,同时暴露了干部职工基本素质的停顿,不适应当前设备及生产工艺的要求。
\subsection{会议交给的任务}
\label{sec-1-3}
\subsubsection{坚定不移地实施生产、设备统管变革。}
\label{sec-1-3-1}

一是有职有岗必照规范的责任唯一管理。每一名领导、干部、职工都必须对自己工作中的言行举止承担责任,兑现奖惩。
二是有规有制必循标准的准绳唯一管理。每一个岗位都必须对自己正常履职尽责的应知应会熟练掌握,对相邻的上下道工序基本规程、制度了然于胸,并将规章制度作为唯一准绳,精准作业。
三是有凭有据必遵原则的真实唯一管理。每一级管理者都必须对自己的上级和下级负责,时时事事以大局整体效益为中心,把空间留给公司,讲究凭据,账实相符,坚持原则,真实可控。
\subsubsection{全面开展“查隐患、反违章、学三规四制、提素质”活动。}
\label{sec-1-3-2}

会议要求,正确认识活动的目标任务,牢固树立全员参与意识,全面建立长效学习机制。
\section{结合自身的岗位对今后工作的思考}
\label{sec-2}

作为设备管理岗位,会上通报的几起设备事故让我内心深受触动。结合工作实践,我的认识是,现代设备的特点是大型化、精密化、自动集成化,一但失去管控,造成事故,迫使生产中断,必将给企业造成重大损失。
\subsection{要撑控住设备,首先是要了解设备的自身特点,会正确使用和维护,通过点检避免发生意外,通过劣化分析完成定修及更换。这些,通过学习和实践设备的“三规”可提高员工和撑控能力。}
\label{sec-2-1}
\subsection{设备“三规”需要不断的学习和完善,所以应推动各项设备管理规程的学习和完善工作,这是设备管理岗位长期持续研究的重点内容之一。}
\label{sec-2-2}
\subsection{设备管理是全员参与的管理,操作、点检、维修到管理者必须有明确的任务和责任,并规范实施。}
\label{sec-2-3}

\end{document}