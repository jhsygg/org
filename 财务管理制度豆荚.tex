% Created 2017-07-04 二 07:46
\documentclass{ctexart}
\usepackage[utf8]{inputenc}
\usepackage{abstract}
\usepackage{geometry}
\geometry{left=2.5cm,right=2.5cm,top=2.5cm,bottom=2.5cm}
\usepackage{flafter}
\setlength{\headheight}{15pt}
\usepackage[title]{appendix}
\usepackage{fixltx2e}
\usepackage{graphicx}
\usepackage{longtable}
\usepackage{float}
\usepackage{wrapfig}
\usepackage{soul}
\usepackage{textcomp}
\usepackage{marvosym}
\usepackage{wasysym}
\usepackage{latexsym}
\usepackage{amssymb}
\usepackage{bpchem}
\usepackage{mhchem}
\usepackage{listings}
\usepackage{xcolor}
\usepackage{multirow,array,multicol,indentfirst}
\usepackage{SIunits,fancyhdr}
\usepackage{tikz,pifont,footnote}
\usepackage[colorlinks,linkcolor=blue,anchorcolor=blue,citecolor=green]{hyperref}
\usepackage{enumerate,comment}
\usepackage{lastpage}
\usepackage{layout}
\newtheorem{thm}{{定理}}
\newtheorem{proposition}{{命题}}
\newtheorem{lemma}{{引理}}
\newtheorem{corollary}{{推论}}
\newtheorem{definition}{{定义}}
\newtheorem{rules}{{规则}}
\newtheorem{suggest}{{建议}}
\newtheorem{example}{{例}}
\CTEXsetup[format={\raggedright}]{section}
\CTEXsetup[format={\raggedright}]{subsection}
\CTEXsetup[format={\raggedright}]{subsubsection}
\pagestyle{fancy}
\author{六盘水豆荚信息技术服务有限公司}
\date{}
\title{}
\hypersetup{
  pdfkeywords={},
  pdfsubject={},
  pdfcreator={Emacs 25.2.1 (Org mode 8.2.10)}}
\begin{document}

\tableofcontents


\section{总则}
\label{sec-1}
\subsection{为了加强公司的财务工作,完善公司的财务管理,强化内部控制,规范财务管理流程,根据国家现行的有关财经法规、财务制度、政策结合本公司的实际情况,制定本制度。}
\label{sec-1-1}
\subsection{财务管理的目标是追求企业价值最大化。}
\label{sec-1-2}
\subsection{财务管理的职能是对企业经济活动进行记录、反映、预算、控制、决策。}
\label{sec-1-3}
\subsection{财务管理的内容包括:}
\label{sec-1-4}
\subsubsection{资金管理,统一监控、调剂单位内部货币资金流动,提高资金使用率,掌握资金动态,为管理决策提供参谋。}
\label{sec-1-4-1}
\subsubsection{收支管理,按照财务管理制度规定的流程、标准和范围等,对涉及财务收支的经济业务实施规范管理。加强内部控制,推动预算管理。}
\label{sec-1-4-2}
\subsubsection{财产管理,对企业库存现金、存货、固定资产等的保存、管理、使用、处置进行规范,明确各环节责任,保证企业财产安全完整,并发挥应有效用。}
\label{sec-1-4-3}
\subsubsection{成本与费用管理,针对企业经营特点,以先进管理思想为指导,规范控制成本费用开支,为完善内部管理提供依据。}
\label{sec-1-4-4}
\subsection{财务部门构架与职责}
\label{sec-1-5}

\subsection{财务总监的岗位职责}
\label{sec-1-6}
\subsubsection{在执行董事的领导下,负责公司财务管理、会计报表、预算控制工作,执行公司决议,并监督公司遵守国家财务法律、法规。}
\label{sec-1-6-1}
\subsubsection{负责公司资金运作管理,日常财务管理与分析、资本运作、筹资方略、对外合作谈判等、负责项目成本核算与控制。}
\label{sec-1-6-2}
\subsubsection{负责公司财务管理及内部控制,根据公司业务发展的计划完成年度财务预算,并跟踪其执行情况。}
\label{sec-1-6-3}
\subsubsection{制定、维护、改进公司财务管理程序和政策,以满足控制风险的要求。}
\label{sec-1-6-4}
\subsubsection{监控可能会对公司造成经济损失的重大经济活动,并及时向公司报告。}
\label{sec-1-6-5}
\subsubsection{控制公司重大投资项目,以确保未经批准的项目不实施,批准的项目在预算范围内进行并在控制之中。}
\label{sec-1-6-6}
\subsection{财务部部长}
\label{sec-1-7}
\subsubsection{负责财务部会计报表及日常管理工作的开展,对财务总监负责。}
\label{sec-1-7-1}
\subsubsection{负责及时办理企业货币资金的收付及会计手续,确保生产经营活动的正常运行。}
\label{sec-1-7-2}
\subsubsection{负责企业的逾期债权、债务催收、清理、对帐。}
\label{sec-1-7-3}
\subsubsection{负责成本核算、分析和控制。}
\label{sec-1-7-4}
\subsubsection{负责及时、准确地编制会计报表和公司经济活动分析。}
\label{sec-1-7-5}
\subsubsection{负责根据公司的资金需求筹措资金。}
\label{sec-1-7-6}
\subsubsection{负责公司预算编制与分析、分解、跟踪、反馈。}
\label{sec-1-7-7}
\subsubsection{负责各部门的纯净考核的跟踪与反馈。}
\label{sec-1-7-8}
\subsubsection{合法开展税收筹划,定期组织税收政策执行情况的自查,发现问题及时报告。}
\label{sec-1-7-9}
\subsubsection{对会计核算工作日常检查,确保会计报表数据准确与时效性,内部税务自查,税收筹划建议。}
\label{sec-1-7-10}
\subsection{会计人员工作交接制度}
\label{sec-1-8}
会计人员工作岗位调动和离职,必须在本部门负责人规定的期限内,将本人所经管的会计工作移交清楚,未办完交接手续,不得离开原工作岗位。

已经受理的经济业务,尚未填制会计凭证,应填制完毕;尚未回复的事项,应予以答复处理。整理应该移交的各种有关资料,对未了事项要写出书面说明材料。编制移交清册,移交清册上应有交接双方和监交人的签字、盖章,还应列明监交人的姓名、监交日期,移交清册应列明的问题等。会计人员办理交接手续时,必须有监交人负责监交。一般会计人员交接,由部门负责人监交;部门负责人交接由财务总监监交;财务总监由法定代表人监交。交接完毕后,交接双方和监交人要在移交清册上签名或盖章。清册一式三份,交接双方各执一份,存档一份。
\section{审批权限}
\label{sec-2}
\subsection{公司的各项费用支出、采购种类物品的款项,由经办人部门经理(主管)审批、财务部部长核对、财务总监审批后,再由总经理批准后方可报销。}
\label{sec-2-1}
\subsection{所有付款申请表连同真实完整合法的原始单据,上交财务部后(除供应商贷款外),报销审批支付时限为五个工作日。}
\label{sec-2-2}
\subsection{固定资产、无形资产购置的审批权限:}
\label{sec-2-3}
\subsubsection{固定资产分类}
\label{sec-2-3-1}
\begin{enumerate}
\item 房屋、建筑物:营业用房、非营业用房、简易房、建筑物。
\label{sec-2-3-1-1}
\item 机械设备:生产设备、检测设备为、维修设备、其他机器设备。
\label{sec-2-3-1-2}
\item 交通运输工具。
\label{sec-2-3-1-3}
\item 办公设备:办公家具、办公设备。
\label{sec-2-3-1-4}
\item 电子设备:照相机、电子计算机系统设备、复印机及打印机设备等。
\label{sec-2-3-1-5}
\end{enumerate}
\subsubsection{无形资产:包括电脑软件、专利权、非专利技术、土地使用权、特许权、商誉等。}
\label{sec-2-3-2}
\subsubsection{审批权限:经部门负责人审核后,财务总监审批后,再由总经理批准。}
\label{sec-2-3-3}
\subsection{费用报销审批程序:报销人员填写}
\label{sec-2-4}
\section{收支管理}
\label{sec-3}

\section{财产管理}
\label{sec-4}

\section{销售与收款业务管理}
\label{sec-5}

\section{购货与付款业务管理}
\label{sec-6}

\section{成本与费用管理}
\label{sec-7}

\section{原始凭证与财务档案管理}
\label{sec-8}

\section{财务责任追究制度}
\label{sec-9}

\section{附则}
\label{sec-10}
% Emacs 25.2.1 (Org mode 8.2.10)
\end{document}
%%% Local Variables:
%%% mode: latex
%%% TeX-master: ""
%%% End:
