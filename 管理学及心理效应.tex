% Created 2013-06-08 六 12:26
\documentclass[11pt]{ctexart}
\usepackage[utf8]{inputenc}
\usepackage[T1]{fontenc}
\usepackage{fixltx2e}
\usepackage{graphicx}
\usepackage{longtable}
\usepackage{float}
\usepackage{wrapfig}
\usepackage{soul,geometry}
\usepackage{textcomp}
\usepackage{marvosym}
\usepackage{wasysym}
\usepackage{latexsym}
\usepackage{amssymb}
\usepackage[colorlinks,linkcolor=blue,anchorcolor=blue,citecolor=green]{hyperref}
\geometry{left=2.5cm,right=2.5cm,top=2.5cm,bottom=2.5cm}


\title{101个著名的管理学及心理学效应}

\begin{document}

\maketitle

\setcounter{tocdepth}{3}
\tableofcontents

\vspace*{1cm}
\section{阿基米德与酝酿效应}
\label{sec-1}


在古希腊,国王让人做了一顶纯金的王冠,但他又怀疑工匠在王冠中掺了银子。可问题是这顶王冠与当初交给金匠的一样重,谁也不知道金匠到底有没有捣
鬼。国王把这个难题交给了阿基米德。阿基米德为了解决这个问题冥思苦想,他起初尝试了很多想法,但都失败了。有一天他去洗澡,一边他一边坐进澡盆,
以便看到水往外溢,同时感觉身体被轻轻地托起,他突然恍然大悟,运用浮力原理解决了问题。

不管是科学家还是一般人,在解决问题的过程中,我们都可以发现“把难题放在一边,放上一段时间,才能得到满意的答案”这一现象。心理学家将其称为“酝
酿效应”。阿基米德发现浮力定律就是酝酿效应的经典故事。

日常生活中,我们常常会对一个难题束手无策,不知从何入手,这时思维就进入了“酝酿阶段”。直到有一天,当我们抛开面前的问题去做其他的事情时,百
思不得其解的答案却突然出现在我们面前,令我们忍不住发出类似阿基米德的惊叹,这时,“酝酿效应”就绽开了“思维之花”,结出了“答案之果”。古代诗词
说“山重水复疑无路,柳暗花明又一村”正是这一心理的写照。

心理学家认为,酝酿过程中,存在潜在的意识层面推理,储存在记忆里的相关信息在潜意识里组合,人们之所以在休息的时候突然找到答案,是因为个体消
除了前期的心理紧张,忘记了个体前面不正确的、导致僵局的思路,具有了创造性的思维状态。因此,如果你面临一个难题,不妨先把它放在一边,去和朋
友散步、喝茶,或许答案真的会“踏破铁鞋无觅处,得来全不费功夫”。
\section{阿伦森效应}
\label{sec-2}


阿伦森效应是指人们最喜欢那些对自己的喜欢、奖励、赞扬不断增加的人或物,最不喜欢那些显得不断减少的人或物。

阿伦森是一位著名的心理学家,他认为,人们大都喜欢那些对自己表示赞赏的态度或行为不断增加的人或事,而反感上述态度或行为不断减少的人或事。为什么会这样呢?其实主要是挫折感在作怪。从倍加褒奖到小的赞赏乃至不再赞扬,这种递减会导致一定的挫折心理,但一次小的挫折一般人都能比较平静地加以承受。然而,继之不被褒奖反被贬低,挫折感会陡然增大,这就不大被一般人所接受了。递增的挫折感是很容易引起人的不悦及心理反感的。

阿伦森效应的实验:

阿伦森效应的实验是将实验人分4组对某一人给予不同的评价,借以观察某人对哪一组最具好感。第一组始终对之褒扬有加,第二组始终对之贬损否定,第三组先褒后贬,第四组先贬后褒。

此实验对数十人进行过后,发现绝大部分人对第四组最具好感,而对第三组最为反感。

阿伦森效应的启示:

阿伦森效应提醒人们,在日常工作与生活中,应该尽力避免由于自己的表现不当所造成的他人对自己印象不良方向的逆转。同样,它也提醒我们在形成对别人的印象过程中,要避免受它的影响而形成错误的态度。

阿伦森效应的举例:
\subsection{有效利用}
\label{sec-2-1}


在宿舍楼的后面,停放着一部烂汽车,大院里的孩子们每当晚上7点时,便攀上车厢蹦跳,嘭嘭之声震耳欲聋,大人们越管,众孩童蹦得越欢,见者无奈。这天,一个人对孩子们说:“小朋友们,今我们比赛,蹦得最响的奖玩具手枪一支。”众童呜呼雀跃,争相蹦跳,优者果然得奖。次日,这位朋友又来到车前,说:“今天继续比赛,奖品为两粒奶糖。”众童见奖品直线下跌,纷纷不悦,无人卖力蹦跳,声音疏稀而弱小。第三天,朋友又对孩子们言:“今日奖品为花生米二粒。”众童纷纷跳下汽车,皆说:“不蹦了,不蹦了,真没意思,回家看电视了。”

分析:“正面难攻”的情况下,采用“奖励递减法”可起到奇妙心理效应。
\subsection{反例}
\label{sec-2-2}


小刚大学毕业后分到一个单位工作,刚一进单位,他决心好好地积极表现一番,以给领导和同事们留下非常好的第一印象。于是,他每天提前到单位打水扫地,节假日主动要求加班,领导布置的任务有些他明明有很大的困难,也硬着头皮一概承揽下来。本来,刚刚走上工作岗位的青年人积极表现一下自我是无可厚议的。但问题是小刚的此时表现与其真正的思想觉悟、为人处世的一贯态度和行为模式相差甚远,夹杂着 “过分表演”的成分。因而就难以有长久的坚持性。没过多久,小刚水也不打了,地也不拖了,还经常迟到,对领导布置的任务更是挑肥拣瘦。结果,领导和同事们对他的印象由好转坏,甚至比那些刚开始来的时候表现不佳的青年所持的印象还不好。因为大家对他已有了一个“高期待、高标准”,另外,大家认为他刚开始的积极表现是“装假”,而“诚实”是我们社会评定一个人所运用的“核心品质”。
\subsection{一个故事}
\label{sec-2-3}


国外一位老人,退休后想图个清净,于是就在湖区买了一所房子.住下的前几周倒还太平。可是不久,有几个年轻人开始在附近追逐打闹、踢垃圾桶、且大喊大叫。老人受不了这些噪音,出去对这些年轻人说:”你们玩得真开心。我喜欢热闹,如果你们每天都来这里玩耍,我给你们每人一元钱。“年轻人当然高兴,既玩了还能得钱,何乐而不为呢?于是他们更加卖力地闹将起来。过了两天,老人:愁眉苦脸”地说:“我到现在还没收到养老金,所以,从明天起,每天只能给你们五角钱了。”年轻人虽然显得不太开心,但还是接受了老人的钱。每天下午继续来这里打闹,又过了几天,老人“非常愧疚”地对他们讲“真对不起,通货膨胀使我不得不重新计划我的开支,所以每天只能给你们一毛钱了。”“一毛钱?”一个年轻人脸色发青,“我们才不会为区区一毛钱在这里浪费时间呢,不干了。”从此,老人有了安静悠然的日子。

这个故事中,老人的智慧其实暗合了心理学上的“阿伦森效应”。

实际上,“阿伦森效应”在组织生活中也是常见的。比如以为刚刚毕业的大学生来到政府部门工作,从被保护的环境一下子跳入了一个竞争性的环境,很容易发生“适应不良症”。作为新人,开始时的勤奋工作可能被领导和同事重视并得到赞扬,但日子一长,从局外人逐渐成为局内人,领导的表扬没了,同事的赞赏少了,他会感到不自在,感到自己可有可无,无足轻重,产生挫折心理,所以工作积极性大受影响,没有初来时的那股干劲了,孰不知这种由勤到不勤的转变,对领导和同事而言,同样会产生:“褒奖递减”作用,形成“阿伦森效应”,对其表露出不满。这会进一步加剧该学生的挫折感,使其更加懒散,进而大家更没有好印象。这种恶性循环会使这位大学生越来越陷入一种非常失败的关系之中。
\section{安泰效应}
\label{sec-3}


安泰效应是指:一旦脱离相应条件就失去某种能力的现象。因此,要学会依靠大家、依靠集体。

安泰效应源自于,古希腊神化中有一个大力神叫安泰,他是海神波塞冬与地神盖娅的儿子,他力大无比,百战百胜。但他有一个致命的弱点,那就是他一旦离开大地,离开母亲的滋养,就失去了一切力量,他的对手刺探了这个秘密,设计让他离开大地,把他高高举起,在空中把他杀了。后来,人们把一旦脱离相应条件就失去某种能力的现象称为“安泰效应”。

寓意:没有群众的支持,任何支持都是软弱无力的;水失鱼,犹为水;鱼无水,不成鱼。

安泰效应的启示

“安泰效应”启示我们,人不能失去力量的源泉,不能失去赖以生存和发展的必要环境。在企业建设管理中,企业领导管理者,应善于建设集体,让员工有一个必要的环境,并通过教育员工的集体观念,从而使员工明确:组织是肥沃的大地,而自己是生长在这大地上的一株小草,离开了大地,他将枯萎。如果组织凝聚力不强,刚不能给员工的安全的依靠。因此,要学会依靠大家、依靠集体,“我为人人”才有可能“人人为我”。失去了力量和源泉,你纵有“力拨山兮气盖世”的能耐,也终有失败的时候。

常言说的好:“众人十柴火焰高”,“众人划桨才能开大船”。
\section{暗示效应}
\label{sec-4}


心理学中,在无对抗条件下,用含蓄、抽象诱导的方法对人民的心理和行为产生影响,从而使人们按照一定的方式去行动或接受一定的意见,使其思想、行为与暗示者期望的相符合,这种现象称为“暗示效应”。

一般来说儿童比成人更容易接受暗示,在家教中,家长就可以应用一个“抽象诱导语”的暗示策略使孩子产生暗示效应。

管理中常用的是语言暗示,如班主任在集体场合对好的行为进行表扬,就是对其他同学起到暗示作用。也可以使用手势、眼色、击桌、停顿、提高音量或放低音量等等。有经验的班主任还常常针对学生的某一缺点和错误,选择适当的电影、电视、文学作品等同学生边看边议论,或给学生讲一些有针对性的故事,都能产生较好的效果。

所谓的暗示是指:人或环境以非常自然的方式向个体发出信息,个体无意中接受了这种信息,从而做出相应的反应的一种心理现象。巴甫洛夫认为:暗示是人类最简化、最典型曲条件反射。然而随着研究的深入,人们发现暗示就像一把“双刃剑”,它可以救治一个人,也可以毁掉一个人,关键在于接受心理暗示的个体自身如何运用并把握暗示的意义。

如何利用暗示效应进行心理调节

生活在社会中的每一个人,其实经常使用着暗示,或暗示别人,或接受别人的暗示,或进行自我暗示。积极的心态,如热情、激励、赞许或对他人有力的支持等等,使他人不仅得到积极暗示,而且得到温暖,得到战胜困难的力量。反之,消极的心态,如冷淡、泄气、退缩、萎靡不振等等,则会使人受到消极暗示的影响,使人承受的不仅仅是暗示带来的痛苦与压力,而且还会波及到人的身体健康。
\section{安慰剂效应}
\label{sec-5}


所谓安慰剂,是指既无药效、又无毒副作用的中性物质构成的、形似药的制剂。安慰剂多由葡萄糖、淀粉等无药理作用的惰性物质构成。安慰剂对那些渴求治疗、对医务人员充分信任的病人能产生良好的积极反应,出现希望达到的药效,这种反应就称为安慰剂效应。使用安慰剂时容易出现相应的心理和生理反应的人,称为“安慰剂反应者”。这种人的特点是:好与人交往、有依赖性、易受暗示、自信心不足,经常注意自身的各种生理变化和不适感,有疑病倾向和神经质。
\section{巴纳姆效应}
\label{sec-6}
人们常常认为一种笼统的、一般性的人格描述十分准确地揭示了自己的特点,心理学上将这种倾向称为“巴纳姆效应”(Barnum effect)。

巴纳姆效应又叫福勒效应,因为它最早是由心理学家伯特伦.福勒于1948年通过试验证明的。
\subsection{实验:}
\label{sec-6-1}


弗拉于1948年对学生进行一项个性测验,并根据测验结果分析,测试后学生对测验结果与本身特质的契合度评分,(0分最低,5分最高),平均4.26。事实上,所有学生得到的“个人分析”都是相同的。
\subsection{过程}
\label{sec-6-2}

1、学生们被要求做一个个性测试,做完后会得到一份相应的个性分析。
2、学生们被要求判断,这个分析是否如实精确的反映了自己的性格特点。
3、每个学生最后得到的个性分析报告是完全一样的,没有一个字的差别。
4、打分标准是0——5分,学生们的平均判断分是4.26分,是一个相当高的分数,基本上每个学生们都认为自己的个性很符合同一份个性报告。
\subsection{个性报告}
\label{sec-6-3}

1、你企求受到他人喜爱却对自己吹毛求疵。
2、虽然人格有些缺陷,大体而言你都有办法弥补。
3、你拥有可观的未开发潜能尚未就你的长处发挥。
4、看似强硬、严格自律的外在掩盖着不安与忧虑的内心。
5、许多时候,你严重的质疑自己是否做了对的事情或正确的决定。
6、你喜欢一定程度的变动并在受限时感到不满。
7、你为自己是独立思想家而自豪,并且不会接受没有充分证据的言论。
8、你认为对他人过度坦率是不明智的。
9、有些时候你外向、亲和、充满社会性,有些时候你却内向、谨慎而沉默。
10、你的一些抱负是不切实际的。
\subsection{后期研究}
\label{sec-6-4}

上述个性报告其实是人类普遍的大致性格特征,而且描述模棱两可,其正是巴纳姆效应的语言描述模式。在后期的研究发现,假如以下的条件实现,实验对
象将会对于分析给予更高的准确评价:
1、实验对象相信该分析只应用于他们身上
2、实验对象相信分析者的权威
3、分析主要集中在正面描述方面
\subsection{运用:}
\label{sec-6-5}


著名魔术师巴纳姆说过,他之所以很受欢迎是因为节目中包含了每个人都期待出现的成分,所以他使得“每一分钟都有人上当受骗”。
人们在生活中往往自觉或不自觉地运用着巴纳姆效应,每个人都经常受到他人的影响和暗示,同时也在影响和暗示着别人。利用巴纳姆效应空泛的表述,会让人感到很习惯的情景。利用巴纳姆效应隐秘的各种暗示,会让他人赞同你的观点,采取你需要的行动。
巴纳姆效应在生活中十分普遍,曾经发生过这样一件事,没一个电影院里突然失火,前排的人从窗户上逃离,后排的人也争先恐后地要从这扇窗户里逃生,结果没烧死人,却挤死了不少人,而这一事件足以体现出巴纳姆效应的力量。
很多人请教过算命先生后都认为算命先生说的“很准”。其实,那些求助算命的人本身就有易受暗示的特点。当人的情绪处于低落、失意的时候,对生活失去控制感,于是,安全感也受到影响。一个缺乏安全感的人,心理的依赖性也大大增强,受暗示性就比平时更强了。加上算命先生善于揣摩人的内心感受,稍微能够理解求助者的感受,求助者立刻会感到一种精神安慰。算命先生接下来再说一段一般的、无关痛痒的话便会使求助者深信不疑。

如何避免巴纳姆效应:

第一,要学会面对自己。有这样一个测验人的情商的题目是:当一个落水昏迷的女人被救起后,她醒来发现自己一丝不挂时,第一个反应会是捂住什么呢?

答案是尖叫一声,然后用双手捂着自己的眼睛。

从心理学上来说,这是一个典型的不愿面对自己的例子,因为自己有“缺陷”或者自己认为是缺陷,就通过自己方法把它掩盖起来,但这种掩盖实际上也像上面的落水女人一样,是把自己眼睛蒙上。所以,要认识自己,首先必须要面对自己。

第二,培养一种收集信息的能力和敏锐的判断力。很少有人天生就拥有明智和审慎的判断力,实际上,判断力是一种在收集信息的基础上进行决策的能力,信息对于判断的支持作用不容忽视,没有相当的信息收集,很难做出明智的决断。

有一个故事说,一个替人割草的孩子打电话给一位陈太太说:“您需不需要割草?”陈太太回答说:“不需要了,我已有了割草工。”这个孩子又说:“我会帮您拔掉花丛中的杂草。”陈太太回答:“我的割草工也做了。”这孩子又说:“我会帮您把草与走道的四周割齐。”陈太太说:“我请的那人也已做了,谢谢你,我不需要新的割草工人。”孩子便挂了电话。孩子的哥哥在一旁问他:“你不是就在陈太太那儿割草打工吗?为什么还要打这电话?”孩子带着得意的笑容说:“

我只是想知道我做得有多好!”

这个孩子可以说是十分关于收集针对自己的信息,因此可以预见他的未来成长以及可能取得的成就,绝非是一般小孩子能比。

第三,以人为镜,通过与自己身边的人在各方面的比较来认识自己。在比较的时候,对象的选择至关重要。找不如自己的人作比较,或者拿自己的缺陷与别人的优点比,都会失之偏颇。因此,要根据自己的实际情况,选择条件相当的人作比较,找出自己在群体中的合适位置,这样认识自己,才比较客观。

第四,通过对重大事件,特别是重大的成功和失败认识自己。重大事件中获得的经验和教训可以提供了解自己的个性、能力的信息,从中发现自己的长处和不足。越是在成功的巅峰和失败的低谷,就越能反映一个人的真实性格。

有人说“成功时认识自己,失败时认识朋友”固然有一定的道理,但归根结底,我们认识的都是自己。无论是成功还是失败时,都应坚持辨证的观点,不忽视长处和优点,也要认清短处与不足。
\section{巴霖效应}
\label{sec-7}


源自於马戏团经理巴霖先生的一句名言:每分钟都有一名笨蛋诞生。”巴霖效应” 多少解释了为什麽有些星座或生肖书刊能够”准确的”指出某人的性格。原因在此,那些用来描述性格的词句,其实根本属”人之常情”或基本上适用於大部分人身上的。换言之,那些词句的适用范围是如此的空泛,以至往往”说了等於没说。例如:水瓶座理性而爱好自由,巨蟹座感性而富爱心;然而巨蟹座的人就永远没理性,水瓶座的人就缺乏爱心吗?我们不去否定星座存在的价值,毕竟它存有统计的基础在。但如果你想成为一个聪明人,不去迷信星座,我又得告诉你,你又错了!知道什麽叫做”天醉人亦醉”吗?既然身旁有超过半数的人相信星座,你又何苦试着去推翻那根植於心的观念(实际上也不太可能)?如果一对情侣在星座学中是不甚相配的,即使两人都不迷信,但他们的心理必然会承受一股不小的压力,在往後交往的时间中,若有了冲突磨擦,心中既存的那种”原来真的不合适”的预设就会被强迫成立,最终难逃分手命运!
\section{半途效应}
\label{sec-8}


半途效应是指在激励过程中达到半途时,由于心理因素及环境因素的交互作用而导致的对于目标行为的一种负面影响。

大量的事实表明,人的目标行为的中止期多发生在''半途''附近,在人的目标行为过程的中点附近是一个极其敏感和极其脆弱的活跃区域。

心理学家研究,当人们追求一个目标做到一半时,常常会对自己能否达到这目标产生怀疑,甚至对这个目标的意义产生怀疑,这时候的心理会变得极为敏感和脆弱,这样就容易导致半途而废,心理学上称之为半途效应。

导致半途效应的原因主要有两个,一是目标选择的合理性,目标选择的越不合理越容易出现半途效应;二是个人的意志力,意志力越弱的人越容易出现半途效应。

解决方案:

多注意进行意志力的磨练。

行为学家提出了''大目标、小步子''的方法,对于防止半途效应的发生具有积极的意义。
\section{贝尔纳效应}
\label{sec-9}


英国学者贝尔纳是一位非常有趣的科学天才,他具有天章云锦般的想象和深刻过人的洞察能力。据说,他在饭桌上的一席话所进溅出的思想火花,就是足够别人干一辈子的研究课题。他本人除在结晶学、分子生物学等方面做过重大贡献外,还在“科学学”等其它领域里放射出了创造的光芒。

贝尔纳的同事和学生们都相信,按创造天赋讲,贝尔纳是可以不止一次地获得诺贝尔奖金的。然而,他一生中最高的荣誉不过是获得英国皇家学会勋章和国外院士之职。贝尔纳为什么没有获得诺贝尔奖金有一种公认的回答是“他总是喜欢提出一个题目,抛出一个思想。首先自己涉足一番,然后,就留给他人去创造出最后的成果。全世界有许许多多的其原始思想应归功于贝尔纳的论文,都在别人的名下出版问世了,他一直由于缺乏‘面壁十年’的恒心而蒙受了损失”。

这句话提出了一个关键问题,即兴趣过于广泛、思维过于发散,对科学创造是非常不利的。后人就将这种现象成为贝尔纳效应。
\subsection{效应启示}
\label{sec-9-1}


尔纳效应要求组织的领导者具有伯乐精神、人梯精神和绿叶精神,以组织的大局为先,以组织的发展为重,以工作的需要为急,慧眼识才,潜心育才,放手用才,大胆提拔任用能力比自己强的人,积极为有才干的干部创造脱颖而出的机会和环境。

接下来用贝尔纳效解释达.芬奇、罗蒙诺索夫和罗素现象。

达.芬奇是意大利文艺复兴时的“三杰”(另二位是米开朗基罗与拉斐尔)之首,他不仅是画家,而且是建筑工程师和数学家。

罗蒙诺索夫是俄罗斯著名的化学家、物质不灭定律的发现者、俄罗斯语言奠基人、数学家与诗人。

罗素是英国著名的数学家、哲学家与诺贝尔文学奖得主。

显然,他们都具备极好的发散型思维能力,这三个人都是跨文、理两大科的重量级学者。但不可忽视的是,达•芬奇时代与罗蒙诺索夫时代,自然科学分工远不及现在如此细密,其研究深度也远不及今天如此精到,有时一个课题,一个实验,就要十年、几十年。罗素最早只是研究理论数学,其后,他将主要精力用到对哲学、史学、文学的涉猎与探讨上。如果他一辈子只搞数学或某一个方面的专项研究,也难以有那么多的精力涉足这么多的人文类学科。

现今时代,很难见到先天有艺术灵感者,还可以在游刃有余地玩艺术的同时,又在某一个数理领域的职业有所建树从而达到文理兼融的人;更不用说在某专业上有惊人的成就,同时还精通文、史、哲的奇人。也就是说,在现今科技高度专业化的时代,人们无不受到贝尔纳效应的制约。
\section{贝勃规律}
\label{sec-10}


有人做过一个实验:一个人右手举着300克的砝码,这时在其左手上放305克的砝码,他并不会觉得有多少差别,直到左手砝码的重量加至306克时才会觉得有些重.如果右手举着600克,这时左手上的重量要达到612克才能感觉到重了.也就是说,原来的砝码越重,后来就必须加更大的量才能感觉到差别.

这种现象被称为:贝勃定律

贝勃定律在生活中到处可见.比如5毛钱一分的晚报突然涨了5块钱,那么你会觉得不可思议,无法接受.但是,如果原本500万的房产也涨了5块,甚至500块,你都会觉得价钱根本没有变化.

在情人节接受两个月,一位意大利的心理学家曾在两对具有大体同的成长背景、年龄阶段和交往过程的恋人当中,做了这样一个送玫瑰花的实验。

心理学家让其中一对恋人中的男孩,每个周末都给自己心爱的姑娘送一束红玫瑰;而让另一对恋人中的男孩,只在情人节那一天向自己心爱的姑娘送去一束红玫瑰。

由于两个男孩的送花频率和时机不同,导致了结果的截然不同:

那个在每个周末收到红玫瑰的姑娘,表现得相当平静。尽管没有大的不满意,但她还是忍住不说了一句:“我看到别人送给自己女友大把的‘蓝色妖姬’,比这普通的红玫瑰漂亮多了,心里真是很羡慕!”

而那个从来没的接过红玫瑰的姑娘,当手捧着男朋友送来的红玫瑰花时,表现出了被呵护、被关爱的极度甜蜜,随后竟然旁若无人、欣喜若狂地与男友紧紧拥吻在一起。

有头脑的人会利用贝勃定律为自己减轻做事的阻力.小到商家的产品价格调整他们会小幅度上涨,在人们都接受以后再加价更多;大至谈判的技巧,一般有经验的谈判专家都是在谈判临近结束时才提出一些棘手的条件,而对方被一开始的优厚条件所诱惑,也就不怎么在意后来才知道的那些缺点了.

有些人总抱怨恋人对自己不如刚认识时那么好了,其实这也是贝勃定律在作怪.在还不熟悉的情况下,对方给你的一点点关怀你都会觉得情深似海,而当你们相恋许多之后,与原来相同的那些关爱你也会觉得平淡如水了

我们对亲人朋友的关爱习以为常;而陌生人的一点帮助,却我们就感激不已。这便是“贝勃定律”在操作我们的感觉。对于亲人朋友,我们对他们的关爱习以为常,而且期望值很高。有时他们少了一丝关爱,我们甚至会恶言相向。对于陌生人,我们没有抱着多大的期望,因此,他们的一点点帮助,我们都感动不已。

一个新人刚开始工作,在单位拼命表现,兢兢业业,然后慢慢熟悉环境后就松懈下来,周围人会觉得这个人矫情,前面的表现都是假的,对这个人的人品也提出质疑;另外一个新人,开始就显得一无是处,懒散不守纪律,慢慢熟悉之后,懂得了单位的规矩。仅仅能做到按时上班,但大家立刻都会夸奖他进步,表现越来越好,觉得这个人要求上进,比前者好很多。其实,前者已经做的工作总量不知道比后者多了多少。

俗话说,好人难做。你辛辛苦苦地耕耘,却因为做错一件事而把前面的功劳全部葬送;而坏人却可以因为做件普通的好事而受到称赞。从而,大家对事物的感觉也都产生错觉,似乎后者的“真小人”比前者“伪君子”更值得信任。其实这些都是贝勃定律在操控你的感觉而已。

所以,变了的不是事实,而只是你的感受变了.

我们的感觉很敏感,但也有惰性;它会蒙骗我们的眼睛,也会加重我们的感受而迷失理性.所以,不能太自以为是,我们应带着谦卑的心对待万物众生,才可以少犯错误,积累智慧.

贝勃定律告诉我们,给予方要多做雪中送炭的事,少做锦上添花的事,尽量不做画蛇添足的事;而受予方要懂得珍惜自己的点滴所得,善待身边的人。
\section{比马龙效应,期望效应,罗森塔尔效应}
\label{sec-11}


评价主体低估被评价者能力,认定被评价者是不求上进的、行为差劲的,以致被评价者将这种观念内化,促使被评价者表现不良行为。

远古时候,塞浦路斯王子皮格马利翁喜爱雕塑。一天,他成功塑造了一个美女的形象,爱不释手,每天以深情的眼光观赏不止。看着看着,美女竟活了。

1968年,两位美国心理学家来到一所小学,他们从一至六年级中各选3个班,在学生中进行了一次煞有介事的“发展测验”。然后,他们以赞美的口吻将有优异发展可能的学生名单通知有关老师。8个月后,他们又来到这所学校进行复试,结果名单上的学生成绩有了显著进步,而且情感、性格更为开朗,求知欲望强,敢于发表意见,与教师关系也特别融洽。

实际上,这是心理学家进行的一次期望心理实验。他们提供的名单纯粹是随便抽取的。他们通过“权威性的谎言”暗示教师,坚定教师对名单上学生的信心,虽然教师始终把这些名单藏在内心深处,但掩饰不住的热情仍然通过眼神、笑貌、音调滋润着这些学生的心田,实际上他们扮演了皮格马利翁的角色。学生潜移默化地受到影响,因此变得更加自信,奋发向上的激流在他们的血管中荡漾,于是他们在行动上就不知不觉地更加努力学习,结果就有了飞速的进步。

这个令人赞叹不已的实验,后来被誉为“皮格马利翁效应”或“期待效应”或“罗森塔尔效应”。

于是,皮格马利翁效应也被总结为:“说你行,你就行,不行也行;说你不行,你就不行,行也不行。”
\subsection{案例:}
\label{sec-11-1}


海伦在这家外贸公司工作已经3年了,国际贸易专业毕业的她在公司的业绩表现一直平平。原因是她以前的上司胡悦是个非常傲慢和刻薄的女人,她对海伦的所有工作都不加以赞赏,反而时常泼些冷水。一次,海伦主动搜集了一些国外对公司出口的纺织品类别实行新的环保标准的信息,但是上司知道了,不但不赞赏她的主动工作,反而批评她不专心本职工作,后来海伦再也不敢关注自己的业务范围之外的工作了。海伦觉得,胡悦之所以不欣赏她,是因为她不像其他同事一样奉承她,但是她自问自己不是能溜须拍马的人,所以不可能得到胡悦的青睐,她也就自然地在公司沉默寡言了。

直到后来,公司新调来主管进出口工作的Sam,新上司新作风,从美国回来的Sam性格开朗,对同事经常赞赏有加,特别提倡大家畅所欲言,不拘泥于部门和职责限制。在他的带动下,海伦也积极地发表自己的看法了。由于Sam的积极鼓励,海伦工作的热情空前高涨,她也不断学会新东西,起草合同、参与谈判、跟外商周旋……海伦非常惊讶,原来自己还有这么多的潜能可以发掘,想不到以前那个沉默害羞的女孩,今天能够跟外国客商为报价争论得面红耳赤。

点评:其实,海伦的变化,就是我们说的皮格马利翁效应起了作用。在不被重视和激励、甚至充满负面评价的环境中,人往往会受到负面信息的左右,对自己做比较低的评价。而在充满信任和赞赏的环境中,人则容易受到启发和鼓励,往更好的方向努力,随着心态的改变,行动也越来越积极,最终做出更好的成绩。
\section{彼得原理}
\label{sec-12}


彼得原理(The Peter Principle)正是彼得根据千百个有关组织中不能胜任的失败实例的分析而归纳出来的。其具体内容是:“在一个等级制度中,每个职工趋向于上升到他所不能胜任的地位”。彼得指出,每一个职工由于在原有职位上工作成绩表现好(胜任),就将被提升到更高一级职位;其后,如果继续胜任则将进一步被提升,直至到达他所不能胜任的职位。由此导出的彼得推论是,“每一个职位最终都将被一个不能胜任其工作的职工所占据。层级组织的工作任务多半是由尚未达到不胜任阶层的员工完成的。”每一个职工最终都将达到彼得高地,在该处他的提升商数(PQ)为零。至于如何加速提升到这个高地,有两种方法。其一,是上面的“拉动”,即依靠裙带关系和熟人等从上面拉;其二,是自我的“推动”,即自我训练和进步等,而前者是被普遍采用的。

在对层级组织的研究中,彼得还分析归纳出彼德反转原理:一个员工的胜任与否,是由层级组织中的上司判定,而不是外界人士。如果上司已到达不胜任的阶层,他或许会以制度的价值来评判部属。例如,他会注重员工是否遵守规范、仪式、表格之类的事;他将特别赞赏工作迅速、整洁有礼的员工。总之,类似上司是以输入评断部属。于是对于那些把手段和目的关系弄反了、方法重于目标、文书作业重于预定的目的、缺乏独立判断的自主权、只是服从而不作决定的职业性机械行为者而言,他们会被组织认为是能胜任的工作者,因此有资格获得晋升,一直升到必须作决策的职务时,组织才会发现他们已到达不胜任的阶层。而以顾客、客户或受害者的观点来看,他们本来就是不胜任的。
\section{帕金森定律}
\label{sec-13}


美国著名历史学家诺斯古德.帕金森通过长期调查研究,写了一本名叫《帕金森定律》的书,他在书中阐述了机构人员膨胀的原因及后果:一个不称职的官员,可能有三条出路。第一是申请退职,把位子让给能干的人;第二是让一位能干的人来协助自己工作;第三是任用两个水平比自己更低的人当助手。

这第一条路是万万走不得的,因为那样会丧失许多权力;第二条路也不能走,因为那个能干的人会成为自己的对手;看来只有第三条路最适宜。于是,两个平庸的助手分担了他的工作,他自己则高高在上发号施令。两个助手既无能,也就上行下效,再为自己找两个无能的助手。如此类推,就形成了一个机构臃肿、人浮于事、相互扯皮、效率低下的领导体系。由此得出结论:在行政管理中,行政机构会像金字塔一样不断增多,行政人员会不断膨胀,每个人都很忙,但组织效率越来越低下。这条定律又被称为“金字塔上升”现象。
\section{边际效应}
\label{sec-14}


有时也称为边际贡献,是指消费者在逐次增加1个单位消费品的时候,带来的单位效用是逐渐递减的(虽然带来的总效用仍然是增加的)。

举一个通俗的例子,当你肚子很饿的时候,有人给你拿来一笼包子,那你一定感觉吃第一个包子的感觉是最好的,吃的越多,单个包子给你带来的满足感就越小,直到你吃撑了,那其它的包子已经起不到任何效用了。

边际效应的应用非常广泛,例如经济学上的需求法则就是以此为依据,即:用户购买或使用商品数量越多,则其愿为单位商品支付的成本越低(因为后购买的商品对其带来的效用降低了)。当然也有少数例外情况,例如嗜酒如命的人,是越喝越高兴,或者集邮爱好者收藏一套文革邮票,那么这一套邮票中最后收集到的那张邮票的边际效应是最大的。

了解边际效应的概念,你就可以尝试去在实际生活中运用它,例如:你是公司管理层,要给员工涨工资,给 3K 月薪的人增加 1K 带来的效应一般来说是比6K 月薪增加 1K 大的,可能和 6K 月薪的人增加 2K 的相当,所以似乎给低收入的人增加月薪更对公司有利;另外,经常靠增加薪水来维持员工的工作热情看来也是不行的,第一次涨薪 1K 后,员工非常激动,大大增加了工作热情;第二次涨薪 1K,很激动,增加了一些工作热情;第三次涨薪 2K ,有点激动,可能增加工作热情;第四次 \ldots{} \ldots{} ,直至涨薪已经带来不了任何效果。如果想避免这种情况,每次涨薪都想达到和第一次涨薪 1K 相同的效果,则第二次涨薪可能需要 2K ,第三次需要 3K \ldots{} \ldots{} ,或者使用其它激励措施,例如第二次可以安排其参加职业发展培训,第三次可以对其在职位上进行提升,虽然花费可能想当,但由于手段不同,达到了更好的效果。研究经济学其实也很有意思,只是对很多人来说,与 IT 这个行业不可完全兼得。
\section{波纹效应}
\label{sec-15}


是指在学习的集体中,教师对有影响力的学生施加压力,实行惩罚,采取讽刺、挖苦等损害人格的作法时,会引起师生对立,出现抗拒现象,有些学生甚至会故意捣乱,出现一波未平,一波又起的情形。这时教师的影响力往往下降或消失不见,因为这些学生在集体中有更大的吸引力。这种效应对学生的学习、品德发展、心理品质和身心健康会产生深远而恶劣的影响。
\section{布里丹毛驴效应}
\label{sec-16}


法国哲学家布里丹养了一头小毛驴,每天向附近的农民买一堆草料来喂。

这天,送草的农民出于对哲学家的景仰,额外多送了一堆草料,放在旁边。这下子,毛驴站在两堆数量、质量和与它的距离完全相等的干草之间,可是为难坏了。它虽然享有充分的选择自由,但由于两堆干草价值相等,客观上无法分辨优劣于是它左看看,右瞅瞅,始终也无法分清究竟选择哪一堆好。

于是,这头可怜的的毛驴就这样站在原地,一会儿考虑数量,一会儿考虑质量,一会儿分析颜色,一会儿分析新新鲜度,犹犹豫豫,来来回回,在无所适从中活活地饿死了。

在我们每一个人的生活中也经常面临着种种抉择,如何选择对人生的成败得失关系极,因而人们都希望得到最佳的抉择,常常在抉择之前反复权衡利弊,再三仔细斟酌,甚至犹豫不决,举棋不定。但是,在很多情况下,机会稍纵即逝,并没有留下足够的时间让我们去反复思考,反而要求我们当机立断,迅速决策。如果我们犹豫不决,就会两手空空,一无所获。

有人把决策过程中这种犹豫不定、迟疑不决的现象称之为“布里丹毛驴效应”。我们没有理由说驴比狼更愚蠢,如果说愚蠢,有时人比驴和狼都蠢。古人讲:“用兵之害,犹豫最大;三军之灾,生于狐疑。”

有个农民的妻子和孩子同时被洪水冲走,农民从洪水中救起了妻子,不幸孩子被淹死了。对此,人们议论纷纷,莫衷一是。有的说农民先救妻子做得对,因为妻子不能死而复生,孩子却可以再生一个;有的却说农民做得不对,应该先救孩子,因为孩子死了无法复活,妻子却可以再娶一个。

一位记者听了这个故事,也感到疑惑不解,便去问那个农民,希望能找到一个满意的答案。想不到农民告诉他:“我当时什么也没有想到,洪水袭来时妻子就在身边,便先抓起妻子往边上游,等返回再救孩子时,想不到孩子已被洪水冲走了。”

“布里丹毛驴效应”是决策之大忌。当我们面对两堆同样大小的干草时,或者“非理性地”选择其中的一堆干草,或者“理性地”等待下去,直至饿死。前者要求我们在已有知识、经验基础上,运用直觉、想象力、创新思维,找出尽可能多的方案进行抉择,以“有限理性”求得“满意”结果。
\subsection{采用稳健的决策方式。}
\label{sec-16-1}


有一个流传很广的笑话说:齐国有个女孩,两个人同时来求婚。东家的儿子很丑但是家财万贯,西家的儿子相貌英俊但是很穷。那女孩的父母不能决定选谁,就去问他们的女儿想嫁给哪个。女孩不好意思说话,母亲就说,你想嫁哪个就露出哪边的胳臂。结果女孩露出两个胳臂。母亲奇怪地问她原因,女孩说:“我想在东家吃饭,西家住。”

在东家吃饭在西家住,看上去是一个笑话,但却不失为了一种稳健的决策取向。在很多情况下,当一种趋势出现时,有些人一个劲地陷入哪个好哪个坏的争论之中,事实上没有这个必要,只要没有明确的二者择一的必要,就不必太早决策。
\subsection{要养成独立思考的习惯。}
\label{sec-16-2}


不能独立思考,总是人云亦云,缺乏主见的人,是不可能做出正确决策的。如果不能有效运用自己的独立思考能力,随时随地因为别人的观点而否定自己的计划,将会使自己的决策很容易出现失误。

从前,有兄弟两个看见天空中一只大雁在飞,哥哥准备把它射下来。说:“等我们射下来就煮着吃,一定会很香的!”这时,他的弟弟抓住他的胳膊争执起来:“鹅煮着才会好吃,大雁要烤着才好吃,你真不懂吃。”哥哥已经把弓举起来,听到这里又把弓放下,为怎么吃这只大雁而犹豫起来。就在这时,有一位老农从旁边经过,于是他们就向老农请教。老农听了以后笑了笑说:“你们把雁分开,煮一半烤一半,自己一尝不就知道哪一种方法更好吃了?”

哥哥大喜,拿起弓箭再回头要射大雁时,大雁早已无影无踪了,连一根雁毛都没有留下。
\subsection{严格执行一种决策纪律。}
\label{sec-16-3}


一个越国人为了捕鼠,特地弄回一只擅于捕老鼠的猫,这只猫擅于捕鼠,也喜欢吃鸡,结果越国人家中的老鼠被补光了,但鸡也所剩无几,他的儿子想把吃鸡的猫弄走,作父亲的却说:“祸害我们家中的是老鼠不是鸡,老鼠偷我们的食物咬坏我们的衣物,挖穿我们的墙壁损害我们的家具,不除掉它们我们必将挨饿受冻,所以必须除掉它们!没有鸡大不了不要吃罢了,离挨饿受冻还远着哩!”

利与弊往往是事情的一体两面,很难分割。有的人明明事先已经编制了能有效抵御风险的决策纪律,但是一旦现实中的风险牵涉到自己的切身利益时,往往就不容易下决心执行了。很多股民在处于有利状态时会因为赚多赚少的问题而犹豫不决,在处于不利状态时,虽然有事先制定好的止损计划和止损标准,可常常因为最终使自己被套牢。
\subsection{不要总是试图获取最多利益。}
\label{sec-16-4}


过高的目标不仅没有起到指示方向的作用,反而由于目标定得过高,带来一定心理压力,束缚决策水平的正常发挥。事实上多数环境中,如果没有良好的决策水平做支撑,一味地追求最高利益,势必将处处碰壁。

而且,很多人不了解尽快停损的重要性,当情况开始恶化时,依然紧抱着飘渺的勾想,无法客观分析状况,以赌徒的心态,盲目坚守以致持续深陷,直至无法挽回的地步。这时平衡的心态往往更重要。

有个人布置了一个捉火鸡的陷阱,他在一个大箱子的里面和外面撒了玉米,大箱子有一道门,门上系了一根绳子,他抓着绳子的另一端躲在一处,只要等到火鸡进入箱子,他就拉扯绳子,把门关上。有一次,12只火鸡进入箱子里,不巧1只溜了出来,他想等箱子里有12只火鸡后,就关上门,然而就在他等第12只火鸡的时候,又有2只火鸡跑出来了,他想等箱子里再有11只火鸡,就拉绳子,可是在他等待的时候,又有3只火鸡溜出来了,最后,箱子里1只火鸡也没剩。
\subsection{在不利环境中不能逆势而动。}
\label{sec-16-5}


当不利环境造成损失时,很多人急于弥补损失。但是,环境的变化是不以人的意志为转移的。当环境变坏,机会稀少的时候,如果强行采取冒险和激进的决策,或频繁的增加操作次数,只会白白增加投资失误的概率。

美国通用电气公司总裁杰克.韦尔奇把决策能力看成是“面对困难处境勇于作出果断决定的能力”,看成是“始终如一执行的能力。”因此,决策具有复合性,是一种合力,我们必须从自己的洞察力、分析能力、直觉能力、创新能力、行动能力和意志力等方面进行不断地训练,在不断地失败与成功之间,我们才能够不断地摆脱犹豫不决,进行相对理性的选择,才不会成为布里丹的驴子!

把眼前的机会抓住了,把手头的事情办好了,就意味着胜利,意味着成功。与其在那里好高骛远设计,绞尽脑汁地编织出一个又一个方案,不如面对现实,抓住机会,竭尽全力,把眼前最重要的事情办好。
\section{不值得定律}
\label{sec-17}


最直观的表达为:不值得做的事情,就不值得做好。

这个定律反映出人们的一种心理,一个人如果从事的是一份自认为不值得的事情,往往会持冷嘲热讽、敷衍了事的态度。不仅成功率小,即使成功,也不会觉得有多大的成就感。
\subsection{价值观。}
\label{sec-17-1}


关于价值观我们已经谈了很多,只有符合我们价值观的事,我们才会满怀热情去做。
\subsection{个性和气质。}
\label{sec-17-2}


一个人如果做一份与他的个性气质完全背离的工作,他是很难做好的,如一个好交往的人成了档案员,或一个害羞者不得不每天和不同的人打交道。
\subsection{现实的处境。}
\label{sec-17-3}


同样一份工作,在不同的处境下去做,给我们的感受也是不同的。例如,在一家大公司,如果你最初做的是打杂跑腿的工作,你很可能认为是不值得的,可是,一旦你被提升为领班或部门经理,你就不会这样认为了。

总结一下,值得做的工作是:符合我们的价值观,适合我们的个性与气质,并能让我们看到期望。如果你的工作不具备这三个因素,你就要考虑换一个更合适的工作,并努力做好它。

因此,对个人来说,应在多种可供选择的奋斗目标及价值观中挑选一种,然后为之奋斗。选择你所爱的,爱你所选择的,才可能激发我们的斗志,也可以心安理得。而对一个企业或组织来说,则要很好地分析员工的性格特性,合理分配工作,如让成就欲较强的职工单独或牵头完成具有一定风险和难度的工作,并在其完成时,给予及时的肯定和赞扬;让依附欲较强的职工,更多地参加到某个团体中共同工作;让权力欲较强的职工,担任一个与之能力相适应的主管。同时要加强员工对企业目标的认同感,让员工感觉到自己所做的工作是值得的,这样才能激发职工的热情。”
\section{超限效应}
\label{sec-18}


美国著名幽默作家马克.吐温有一次在教堂听牧师演讲。最初,他觉得牧师讲得很好,使人感动,准备捐款。过了10分钟,牧师还没有讲完,他有些不耐烦了,决定只捐一些零钱。又过了10分钟,牧师还没有讲完,于是他决定,1分钱也不捐。到牧师终于结束了冗长的演讲,开始募捐时,马克.吐温由于气愤,不仅未捐钱,还从盘子里偷了2元钱。

这种刺激过多、过强和作用时间过久而引起心理极不耐烦或反抗的心理现象,称之为“超限效应”。

超限效应在家庭教育中时常发生。如:当孩子不用心而没考好时,父母会一次、两次、三次,甚至四次、五次重复对一件事作同样的批评,使孩子从内疚不安到不耐烦最后反感讨厌。被“逼急”了,就会出现“我偏要这样”的反抗心理和行为。因为孩子一旦受到批评,总需要一段时间才能恢复心理平衡,受到重复批评时,他心里会嘀咕:“怎么老这样对我?”孩子挨批评的心情就无法复归平静,反抗心理就高亢起来。可见,家长对孩子的批评不能超过限度,应对孩子“犯一次错,只批评一次”。如果非要再次批评,那也不应简单地重复,要换个角度,换种说法。这样,孩子才不会觉得同样的错误被“揪住不放”,厌烦心理、逆反心理也会随之减低。

超限效应的启示

1、刺激过多、过强或作用时间过久,往往会引起对方心理极不耐烦或逆反,这样会事与愿违,就象马克.吐温一样不仅不捐钱,反而还从盘子里偷走了2元钱。

2、超限效应反应了几个问题:以自我为中心; 没有注意方式、方法;没能注意“度”的把握;没有换位思考。
\section{拆屋效应}
\label{sec-19}


鲁迅先生曾于1927年在《无声的中国》一文中写下了这样一段文字:''中国人的性情总是喜欢调和、折中的,譬如你说,这屋子太暗,说在这里开一个天窗,大家一定是不允许的,但如果你主张拆掉屋顶,他们就会来调和,愿意开天窗了。”这种先提出很大的要求,接着提出较小较少的要求,在心理学上被称为''拆屋效应''。虽然这一效应在现实生活中多见,但也有不少学生学会了这些。如有的学生犯了错误后离家出走,班主任很着急,过了几天学生安全回来后,班主任反倒不再过多地去追究学生的错误了。实际上在这里,离家出走相当于''拆屋'',犯了错误相当于''开天窗'',用的就是拆屋效应。因此,班主任在教育学生的过程中,教育方法一定要恰当,能被学生所接受,同时,对学生的不合理要求或不良的行为绝不能迁就,特别要注意不能让学生在这些方面养成与班主任讨价还价的习惯。
\subsection{拆屋效应的原因}
\label{sec-19-1}


我们如何来解释这种现象呢?我们拿两种情况做一下对比,第一种是先提出一个不合理要求,再提出一个相对较小的要求,第二种是直接提出这个较小的要求,比较哪种情况下的要求更易被接受。实验结果表明,在前一种情况下提出的要求更容易被人们所接受,而直接提出要求反而不容易被接受。通常人们不太愿意两次连续地拒绝同一个人,当你对第一个无理要求拒绝后,你会对被拒绝的人有一种歉疚,所以当他马上提出一个相对较易接受的要求时,你会尽量地满足他,而不太愿意连续两次摆出拒绝的姿态,毕竟我们并不想因为自己的行为而让人觉得我们想拒绝这个人。
\subsection{拆屋效应在谈判上的使用}
\label{sec-19-2}


拆屋效应也是在谈判中常用的和有效的技巧,有时候我们需要在谈判一开始就抛出一个看似无理而令对方难以接受的条件,但这却并不意味着我们不想继续谈判下去,而只代表着一种谈判的策略罢了。这是个非常有效的策略,它能让你在谈判一开始就占据着比较主动的地位,但记住这只是“拆屋”,如果想让谈判真正有所进展,不要忘记“开天窗”。所以,如果你的一个要求别人很难接受时,在此前你不妨试试提出个他更不可能接受的要求,或许你会有意外的收获。
\section{成败效应}
\label{sec-20}


是指努力后的成功效应和失败效应,是格维尔茨在研究中发现的。他的研究是,学习材料为几套难度不等的问题,由学生们自由地选择地解决。他发现能力较强的学生,解决了一类中一个问题之后,便不愿意再解决另一个相似的问题,而挑较为复杂的艰难的问题,藉以探索新的解决方法,而感到兴趣更浓。这就是显示学生的兴趣,不仅是来自容易的工作获得成功,而是要通过自己的努力,克服困难,以达到成功的境地,才会感到内心的愉快与愿望的满足。这就是努力后的 \textbf{成功效应} 。在另一方面,能力较差的学生,如果经过极大的努力而仍然不能成功,失败经验累积的次数过多之后,往往感到失望灰心,甚至厌弃学习。这就是努力后的 \textbf{失败效应} 。因此,教师应帮助能力强的学生将目标逐渐提高,帮助能力较弱的学生将目标适当放低,以便适合其能力和经验。
\section{重叠效应:}
\label{sec-21}


在一前一后的记忆活动中,识记的东西是相类似的,对于保存来说是不利的。这是因为重复出现内容相同的东西时,相同性质的东西由于互相抑制,互相干涉而发生了遗忘的结果。柯勒把这种现象命名为“重叠效应”。可见,我们在学习汉字、外文单词以及其他材料时,一定要注意不要把相类似的东西集中在一起,这样容易产生重叠效应。如果要放在一起学习时,最起码有一些材料是很熟的,这样可能会产生同化作用,把生疏的材料同化于已熟记的材料之中。
\section{刺猬法则}
\label{sec-22}


“刺猬”法则可以用这样一个有趣的现象来形象地说明:两只困倦的刺猬由于寒冷而拥在一起,可怎么也睡不舒服,因为各自身上都长着刺,紧挨在一块,反而无法睡得安宁。几经折腾,两只刺猬拉开距离,尽管外面寒风呼呼,可它们却睡得甜乎乎的。

刺猬法则强调的就是人际交往中的“心理距离效应”。运用到管理实践中,就是领导者如要搞好工作,应该与下属保持亲密关系,但这是“亲密有间”的关系,是一种不远不近的恰当合作关系。与下属保持心理距离,可以避免下属的防备和紧张,可以减少下属对自己的恭维、奉承、送礼、行贿等行为,可以防止与下属称兄道弟、吃喝不分。这样做既可以获得下属的尊重,又能保证在工作中不丧失原则。一个优秀的领导者和管理者,要做到“疏者密之,密者疏之”,这才是成功之道。

法国总统戴高乐

法国总统戴高乐就是一个很会运用刺猬法则的人。他有一个座右铭:“保持一定的距离”!这也深刻地影响了他和顾问、智囊和参谋们的关系。在他十多年的总统岁月里,他的秘书处、办公厅和私人参谋部等顾问和智囊机构,没有什么人的工作年限能超过两年以上。他对新上任的办公厅主任总是这样说:“我使用你两年,正如人们不能以参谋部的工作作为自己的职业,你也不能以办公厅主任作为自己的职业。”这就是戴高乐的规定。这一规定出于两方面原因:一是在他看来,调动是正常的,而固定是不正常的。这是受部队做法的影响,因为军队是流动的,没有始终固定在一个地方的军队。二是他不想让“这些人”变成他“离不开的人”。这表明戴高乐是个主要靠自己的思维和决断而生存的领袖,他不容许身边有永远离不开的人。只有调动,才能保持一定距离,而惟有保持一定的距离,才能保证顾问和参谋的思维和决断具有新鲜感和充满朝气,也就可以杜绝年长日久的顾问和参谋们利用总统和政府的名义营私舞弊。

戴高乐的做法是令人深思和敬佩的。没有距离感,领导决策过分依赖秘书或某几个人,容易使智囊人员干政,进而使这些人假借领导名义,谋一己之私利,最后拉领导干部下水,后果是很危险的。两相比较,还是保持一定距离好。

通用电气公司

通用电气公司的前总裁斯通在工作中就很注意身体力行刺猬理论,尤其在对待中高层管理者上更是如此。在工作场合和待遇问题上,斯通从不吝啬对管理者们的关爱,但在工余时间,他从不要求管理人员到家做客,也从不接受他们的邀请。正是这种保持适度距离的管理,使得通用的各项业务能够芝麻开花节节高。与员工保持一定的距离,既不会使你高高在上,也不会使你与员工互相混淆身份。这是管理的一种最佳状态。距离的保持靠一定的原则来维持,这种原则对所有人都一视同仁:既可以约束领导者自己,也可以约束员工。掌握了这个原则,也就掌握了成功管理的秘诀。
\section{从众效应,羊群效应}
\label{sec-23}


从众效应(conformity):从众效应是指人们自觉不自觉地以多数人的意见为准则,作出判断、形成印象的心理变化过程。

这是指作为受众群体中的个体在信息接受中所采取的与大多数人相一致的心理和行为的对策倾向。从众是合乎人们心意和受欢迎的。不从众不仅不受欢迎,还会引起灾祸。例如,车流滚滚的道路上,一位反道行驶的汽车司机;弹雨纷飞的战场上,一名偏离集体、误入敌区的战士;万众屏气静观的剧场里,一个观众突然歇斯底里的大声喊叫……公众几乎都讨厌越轨者,甚至会对他群起而攻之。

从众效应作为一个心理学概念,是指个体在真实的或臆想的群体压力下,在认知上或行动上以多数人或权威人物的行为为准则,进而在行为上努力与之趋向一致的现象。从众效应既包括思想上的从众,又包括行为上的从众。从众是一种普遍的社会心理现象,从众效应本身并无好坏之分,其作用取决于在什么问题及场合上产生从众行为,具体表现在两个方面:

一是具有积极作用的从众正效应;

二是具有消极作用的从众负效应。

积极的从众效应可以互相激励情绪,做出勇敢之举,有利于建立良好的社会氛围并使个体达到心理平衡,反之亦然。

有这么一个实验:某高校举办一次特殊的活动,请德国化学家展示他最近发明的某种挥发性液体。当主持人将满脸大胡子的“德国化学家”介绍给阶梯教师里的学生后,化学家用沙哑的嗓音向同学们说:“我最近研究出了一种强烈挥发性的液体,现在我要进行实验,看要用多长时间能从讲台挥发到全教室,凡闻到一点味道的,马上举手,我要计算时间。”说着,他打开了密封的瓶塞,让透明的液体挥发……不一会,后排的同学,前排的同学,中间的同学都先后举起了手。不到2分钟,全体同学举起了手。

此时,“化学家”一把把大胡子扯下,拿掉墨镜,原来他是本校的德语老师。他笑着说:“我这里装的是蒸馏水!”

这个实验,生动的说明了同学之间的从众效应——看到别人举手,也跟着举手,但他们并不是撒谎,而是受“化学家”的言语暗示和其他同学举手的行为暗示,似乎真的闻到了一种味道,于是举起了手。

谈判中的陷阱——从众效应

首先,对任何“热点”都持冷静态度,做好热门交易都极有可能迅速变“冷”的心理准备,迅速设立停损位,一旦热点变冷,接近停损位,立即出手。在我们进行一笔大交易之前要有耐心,花点时间进行大量的市场调查、实地考察和分析工作。抵制迅速达成交易的诱惑。然后,对于热点我们要关注长期利益,警惕那些基于“早进场,早得利”理念的交易。

在一群羊前面横放一根木棍,第一只羊跳了过去,第二只、第三只也会跟着跳过去;这时,把那根棍子撤走,后面的羊,走到这里,仍然像前面的羊一样,向上跳一下,尽管拦路的棍子已经不在了,这就是所谓的“羊群效应”也称“从众心理”。
\section{淬火效应}
\label{sec-24}


金属工件加热到一定温度后,浸入冷却剂(油、水等)中,经过冷却处理,工件的性能更好、更稳定。长期受表扬头脑有些发热的学生,不妨设置一点小小的障碍,施以“挫折教育”,几经锻炼,其心理会更趋成熟,心理承受能力会更强;对于麻烦事或者已经激化的矛盾,不妨采用“冷处理”,放一段时间,思考得会更周全,办法会更稳妥。
\section{达维多定律}
\label{sec-25}


达维多定律是以英特尔公司副总裁达维多的名字命名的。达维多认为,一家企业要在市场中总是占据主导地位,那么它就要永远做到第一个开发出新一代产品,第一个淘汰自己的产品。

这一定律的基点是着眼于市场开发和利益分割的成效。人们在市场竞争中无时无刻不在抢占先机,因为只有先入市场,才能更容易获得较大的份额和高额的利润。英特尔公司在产品开发和推广上奉行达维多定律,始终是微处理器的开发者和倡导者。他们的产品不一定是性能最好的和速度最快的,但他们一定做到是最新的。为此,他们不惜淘汰自己哪怕是市场上正卖得好的产品。

达维多定律揭示了以下取得成功的真谛:不断创造新产品,及时淘汰老产品,使新产品尽快进入市场,并以自己成功的产品形成新的市场和产品标准,进而形成大规模生产,取得高额利润。
\section{搭便车效应}
\label{sec-26}


群体内的责任扩散鼓励了个体的民懒散。当群体结果无法归因于任何单独个体时,个人的投入与整体的产出之间的关系将不明朗。

是指在利益群体内,某个成员为了本利益集团的利益所作的努力,集团内所有的人都有可能得益,但其成本则由这个人个人承担,这就是搭便车效应。

在合作学习中虽然全体小组成员客观上存在着共同的利益,但是从社会心理学的角度看,却容易形成“搭便车”的心理预期,个别学生活动时缺乏主动性或干脆袖手旁观,坐享其成;也有的学生表面上看参与了活动,实际上却不动脑筋,不集中精力,活动中没有发挥应有的作用等“搭便车”现象。产生“搭便车效应”的原因很多,首先是异质分组客观上使学生的动机、态度和个性有差异,其次许多学生没有完成合作技巧的培训,对于合作学习的评价的“平均主义”,即只看集体成绩不考虑个人成绩的做法等。 “搭便车效应”的危害非常大的,在合作学习过程中,如果更多地强调“合作规则”而忽视小组成员的个人需求,可能会使每个人都希望由别人承担风险,自己坐享其成,这会抑制小组成员为小组的利益而努力的动力。而且“搭便车”心理可能会削弱整个合作小组的创新能力、
凝聚力、积极性等。心理学研究表明,如果合作小组的规模较小,由于每个小组成员的努力对整个小组都有较大影响,其个人的努力与奖励的不对称性相对较小,会使“搭便车效应”明显减弱;而且缩小规模的另外一个作用就是社会惰化现象会削弱,能够取得较高的合作效率和成果。所以在合作学习中建议4-6人为一小组,不要把有些大班简单地分成几个小组。当然还有许多事情可以做,比如要营造一种愉快的合作学习环境;要明确任务与责任合理分工;随时观察学情,监控活动过程,指导合作的技巧,调控学习任务,督促学生完成任务;奖励机制分配上破除“平均主义”。
\section{答布效应}
\label{sec-27}


角色行为的“导演”

我们知道,我们每一个人都不是纯生物性的个体,而是一个个活生生的社会的人。在社会的大舞台上,每个人都扮演着一定的角色。那么,我们每一个人所显现的角色行为,又是由什么所“导演”的呢?

让我们把话题回溯到原始社会吧。根据研究认为,当时就有一种传统的习惯和禁律,史称之为“答布”。 “答布''是人类社会最初期的一种生活规范。当时虽然还没有宗教、道德、法律等观念存在,但是人们在生活中,已经混合这三者观念统一使用。史学家通称“答布”为“法律诞生前的公共的规范”。 “答布”为什么能有这样一种效应呢?社会心理学家分析,这是由于原始社会的科学文化水平很低,所以人们对于所谓的神怪或是污秽事物有一种禁忌心理,认为如果触犯禁忌,便要蒙受灾害,故而必须远远地躲避它们、畏敬它们,而由这种信念所形成的习俗,就是“答布”。同时,当时的文化发展水平也使人们初步认识到作为参加社会活动的个体,其行为必须要服从于一定的法则、一定的行为规范。这便是“答布效应”的由来。现代社会的科学文化发展水平,当然不是原始社会可
以比拟的。现代社会所赖以维持的力量,现代人角色行为的“导演”,已经不是什么“答布”了。但是,从社会心理学的意义上来说, “答布效应”所揭示的角色行为由角色规范“导演”这一内涵,却是不会过时的。

社会心理学对人类行为(外显的和内潜的)研究的主要贡献之一,就在于阐明了一个社会如何使其成员的行为遵从社会现行的,适合一定阶级要求和需要的行为规范与道德准则,或是倡导其成员如何遵从本民族的文化规范。社会是规范的体系。任何一个社会都有一套约定俗成的行为规范,其所有成员都必须遵守。从这个角度而言,只要我们不是把“答布效应”中的“答布”仅仅理解为原始社会里的“答布”,而是把它理解为角色行为的“导演”一角色规范,那么,我们就可以说, “答布效应”在任何社会里都是客观存在的。

现代社会里的“答布效应”有狭义和广义之分。狭义的是指那些经过一定程序使之成为可见的条文,如宪法、各种法律政策规定、党纪,各种道德法规、各类公约守则等等。广义的则是指那些不成文的东西,它们存在于人们的头脑中,通过舆论的形式表现出来,这就是风俗习惯、道德观念。这些东西虽然没有写进有关条文,但却渗透在每一个角色扮演者的心理和行为之中。上述明文规定和没有明文规定的行为准则之总和,就是社会对每一个成员提出的要求,就是对所有角鱼扮演者表现角色行为的“总导演”。可见,这些约定或俗成的行为规范,概括起来说,最集中地体现在法制观念和道德观念这两个方面。但它们又涉及到生活的各个领域,内容和形式是相当广泛、多样的。如果需要作具体分类的话,那么大致可分为:

(1)正式规范,即由正式文件明文规定的规范,如规章制度和守则等等。

(2)非正式规范,即由群众自发形成的规范,如朋友见面时的招呼方式,衣服式样,等等,如果违反之,就会在一定范围的群体中受到众人的冷眼,产生心理压力。

(3)所属规范,指个体成员所参加群体的规范,如你成为某一协会的会员,就必须遵守该协会的章程等规范。

(4)参考规范,即个人往往以心目中的模范人物作为自己参照的行为准则。

(5)地区性规范,指某个地区的群体所特有的规范,如少数民族的风俗习惯以及语言规范等等,我们常常说“入乡随俗”,便是这种规范的表现。

凡此种种,都表明规范就是一种标准化的观念,角色规范就是角色扮演者必须遵守的已经确立的思想,评价和行为的标准。有了这个标准,角色粉演者就明白应该做什么,不应该做什么,在什么情况下应该表现出这样的行为,在什么情况下不应该表现出这样的行为。

“答布效应”的原理告诫我们要用角色规范来“导演”角色行为,这不仅表观于社会对每一个成员的总体要求,必须在法制观念和道德观念的规范框架内活动,而且还反映在对个体扮演某一具体角色时也要符合特殊的角色规范。这好比在舞台上演出,每一个演员首先都必须贯彻导演的总体要求,诸如台风要正,思想集中、听从安排等等;此外,你扮演的是旦角或者是武生或者是别的什么具体角色,还应该根据这一角色的特殊要求去唱、去做。这两方面的紧密结合,才是角色行为的统一体。我们在社会的大舞台上扮演社会角色,也是同样的道理。一方面,我们要遵守社会规范对所有社会角色扮演者的共同要求,另一方面,还要内化对某一种角色扮演的特殊规范。比如,你在家里已经扮演起年轻的爸爸角色,那你就应当懂得社会对家长角色的一些特殊要求,表现出为社会教育好子女等方面良好的角色行为。假如你是一位小学教师的话,那么,你在各方面都应当符合为人师表的角色规范。如此等等,你都可以而且应当根据你的角色位置,去思考、去行动,以使自己的角色行为既符合角色规范的普遍要求又落实了特殊要求。普遍性和特殊性的结合,共性与个性的体现,社会舞台上的角色扮演者不正是这样既阵营整齐、又多彩多姿的吗?!

你要表现出良好的角色行为,你要提高角色的扮演水平,请别忘了, “导演”就站在你的身边。关键是要你认识他的面貌、理解他的意图、落实他的要求。
——“他”,就是角色规范的代名词: “答布效应”。
\section{德西效应}
\label{sec-28}


心理学家德西在1971年做了一个专门的实验。他让大学生做被试者,在实验室里解有趣的智力难题。实验分三个阶段,第一阶段,所有的被试者都无奖励;第二阶段,将被试者分为两组,实验组的被试者完成一个难题可得到1美元的报酬,而控制组的被试者跟第一阶段相同,无报酬;第三阶段,为休息时间,被试者可以在原地自由活动,并把他们是否继续去解题作为喜爱这项活动的程度指标。实验组(奖励组)被试者在第二阶段确实十分努力,而在第三阶段继续解题的人数很少,表明兴趣与努力的程度在减弱,而控制组(无奖励组)被试者有更多人花更多的休息时间在继续解题,表明兴趣与努力的程度在增强。

德西在实验中发现:在某些情况下,人们在外在报酬和内在报酬兼得的时候,不但不会增强工作动机,反而会减低工作动机。此时,动机强度会变成两者之差。人们把这种规律称为德西效应。这个结果表明,进行一项愉快的活动(即内感报酬),如果提供外部的物质奖励(外加报酬),反而会减少这项活动对参与者的吸引力。

为什么会产生这种有趣的德西效应呢?可能的解释是:

1、原有的外加报酬距有关需要满足的水平太远,对外加报酬的要求太强烈;

2、直接激励的原有强度不足;

3、价值观(思想信念)的某种偏差,未能将需要层给结构调整得合乎工作要求。
\section{得寸进尺效应}
\label{sec-29}


美国社会心理学家弗里得曼做了一个有趣的实验:他让助手去访问一些家庭主妇,请求被访问者答应将一个小招牌挂在窗户上,她们答应了。过了半个月,实验者再次登门,要求将一个大招牌放在庭院内,这个牌子不仅大,而且很不美观。同时,实验者也向以前没有放过小招牌的家庭主妇提出同样的要求。结果前者有55\%的人同意,而后者只有不到17\%的人同意,前者比后者高3倍。后来人们把这种心理现象叫作“得寸进尺效应”。

心理学认为,人的每个意志行动都有行动的最初目标,在许多场合下,由于人的动机是复杂的,人常常面临各种不同目标的比较、权衡和选择,在相同情况下,那些简单容易的目标容易让人接受。另外,人们总愿意把自己调整成前后一贯、首尾一致的形象,即使别人的要求有些过分,但为了维护印象的一贯性,人们也会继续下去。

上述心理效应告诉我们,要让他人接受一个很大的、甚至是很难的要求时,最好先让他接受一个小要求,一旦他接受了这个小要求,他就比较容易接受更高的要求。差生作为一个特殊群体,其身心素质和学习基础等方面都低于一般水平。转化差生,也要像弗里得曼一样善于引导,善于“搭梯子”,使之逐渐转化;应贯彻“小步子、低台阶、勤帮助、多照应”的原则,注意“梯子”依靠的地方要正确、间距不宜太大、太陡,做到扶一扶“梯子”,托一托人。
\section{等待效应}
\label{sec-30}


由于人们对某事的等待而产生态度、行为等方面的变化,这种现象称等待效应。

在管理中,优秀管理者常常利用这种效应的作用,使员工产生一种对新任务的等待心理,以促进员工的工作兴趣、态度和行为发生积极的变化。

在教学中,优秀教师常常利用这种效应的作用,使学生产生一种对新课文或新学单元的等待心理,以促进学生自己去自学。这就有助于上下课文或前后单元的连续,更为重要的是它能使学生的学习兴趣、态度和行为发生积极的变化。

他提出了被广泛认可和采用的顾客等待心理八条原则:

\begin{enumerate}
\item 无所适事的等待比有事可干的等待感觉要长(Unoccupied waiting feels longer than occupied waiting);
\item 过程前、过程后的等待的时间比过程中等待的时间感觉要长(Pre-process and post-process waits feel longer than in-process waits);
\item 焦虑使等待看起来比实际时间更长(Anxiety makes waits seem longer);
\item 不确定的等待比已知的、有限的等待时间更长(Uncertain waits are longer than known, finite waits);
\item 没有说明理由的等待比说明了理由的等待时间更长(Unexplained waits are longer than explained waits);
\item 不公平的等待比平等的等待时间要长(Unfair waits are longer than equitable waits);
\item 服务的价值越高,人们愿意等待的时间就越长(The more valuable the service, the longer people will wait);
\item 单个人等待比许多人一起等待感觉时间要长(Solo waits feel longer than group waits);
\end{enumerate}
\section{第一印象效应}
\label{sec-31}


一位心理学家曾做过这样一个实验:他让两个学生都做对30道题中的一半,但是让学生A做对的题目尽量出现在前15题,而让学生B做对的题目尽量出现在后15道题,然后让一些被试对两个学生进行评价:两相比较,谁更聪明一些?结果发现,多数被试都认为学生A更聪明。这就是第一印象效应。第一印象效应是指最初接触到的信息所形成的印象对我们以后的行为活动和评价的影响,实际上指的就是“第一印象”的影响。第一印象效应是一个妇孺皆知的道理,为官者总是很注意烧好上任之初的“三把火”,平民百姓也深知“下马威”的妙用,每个人都力图给别人留下良好的“第一印象”
\section{定势效应}
\label{sec-32}


有一个农夫丢失了一把斧头,怀疑是邻居的儿子偷盗,于是观察他走路的样子,脸上的表情,感到言行举止就像偷斧头的贼。后来农夫找到了丢失的斧头,他再看邻居的儿子,竟觉得言行举止中没有一点偷斧头的模样了。这则故事描述了农夫在心理定势作用下的心理活动过程。所谓心理定势是指人们在认知活动中用“老眼光”——已有的知识经验来看待当前的问题的一种心理反应倾向,也叫思维定势或心向。

在人际交往中,定势效应表现在人们用一种固定化了的人物形象去认知他人。例如:我们与老年人交往中,我们会认为他们思想僵化,墨守成规,跟不上时代;而他们则会认为我们年纪轻轻,缺乏经验,“嘴巴无毛,办事不牢”。与同学相处时,我们会认为诚实的人始终不会说谎;而一旦我们认为某个人老奸巨猾,既使他对你表示好感,你也会认为这是“黄鼠狼给鸡拜年没安好心”。心理定势效应常常会导致偏见和成见,阻碍我们正确地认知他人。所以我们要“士别三日,当刮目相看”他人呀!不要一味地用老眼光来看人处事。

定势是心理学中的一个概念。大意是指以前的心理活动会对以后的心理活动形成一种准备状态或心理倾向,从而影响以后心理的活动。在对陌生人形成最初印象时,这种作用特别明显。俄国社会心理学家包达列夫曾作过这样一个实验:他向两组大学生出示了同一个人的照片。在出示之前,向第一组说,将出示的照片上的人是个十恶不赦的罪犯;向另一组说他是位大科学家。然后让两组被试用文字描绘照片上的人的相貌。第一组的评价是:深陷的双眼证明内心的仇恨,突出的下巴证明沿犯罪的道路走到底的决心等等,第二组的评价是:深陷的双眼表明思想的深度,突出的下巴表明在知识道路上克服困难的意志力等等。这个实验有力地说明了定势的作用。
\section{多看效应}
\label{sec-33}


转在许多人眼中,喜新厌旧是人的天性。然而,事实果真是如此吗?

20世纪60年代,心理学家查荣茨做过试验:先向被试出示一些照片,有的出现了20多次,有的出现了10多次,有的只出现一两次,然后请别试评价对照片的喜爱程度,结果发现,被试更喜欢那些只看过几次的新鲜照片,既看的次数增加了喜欢的程度.

这种对越熟悉的东西就越喜欢的现象,心理学上称为多看效应.在人际交往中,如果你细心观察就会发现,那些人缘很好的人,往往将多看效应发挥的淋漓尽致:他们善于制造双方接触的机会,已提高彼此间的熟悉度,然后互相产生更强的吸引力.

人际吸引难道真的是如此的简单?有社会心理学的实验做佐证:在一所大学的女生宿舍楼里,心理学家随机找了几个寝室,发给她们不同口味的饮料,然后要求这几个寝室的女生,可以以品尝饮料为理由,在这些寝室间互相走动,但见面时不得交谈.一段时间后,心理学家评估她们之间的熟悉和喜欢的程度,结果发现:见面的次数越多,互相喜欢的程度越大:见面的次数越少或根本没有,相互喜欢的程度也较低.

可见,若想增强人际吸引,就要留心提高自己在别人面前的熟悉度,这样可以增加别人喜欢你的程度.因此,一个自我封闭的人,或是一个面对他人就逃避和退缩的人,由于不易让人亲近而另人费解,也就是太讨人喜欢.

当然,多看效应发挥作用的前提,是首因效应要好,若给人的第一印象不很差,则见面越多就越讨人厌,多看效应反而起了副用.
\section{多米诺骨牌效应}
\label{sec-34}


在一个存在内部联系的体系中,一个很小的初始能量就可能导致一连串的连锁反应。

楚国有个边境城邑叫卑梁,那里的姑娘和吴国边境城邑的姑娘同在边境上采桑叶,她们在做游戏时,吴国的姑娘不小心踩伤了卑梁的姑娘。卑梁的人带着受伤的姑娘去责备吴国人。吴国人出言不恭,卑梁人十分恼火,杀死吴人走了。吴国人去卑梁报复,把那个卑梁人全家都杀了。

卑梁的守邑大夫大怒,说:“吴国人怎么敢攻打我的城邑?”

于是发兵反击吴人,把当地的吴人老幼全都杀死了。

吴王夷昧听到这件事后很生气,派人领兵入侵楚国的边境城邑,攻占夷以后才离去。吴国和楚国因此发生了大规模的冲突。吴国公子光又率领军队在鸡父和楚国人交战,大败楚军,俘获了楚军的主帅潘子臣、小帷子以及陈国的大夫夏啮,又接着攻打郢都,俘虏了楚平王的夫人回国。

从做游戏踩伤脚,一直到两国爆发大规模的战争,直到吴军攻入郢都,中间一系列的演变过程,似乎有一种无形的力量把事件一步步无可挽回地推入不可收拾的境地。这种现象,我们称之为多米诺骨牌效应。

提出多米诺骨牌效应,还要从我国的宋朝开始说起。

宋宣宗二年(公元1120年),民间出现了一种名叫“骨牌”的游戏。这种骨牌游戏在宋高宗时传入宫中,随后迅速在全国盛行。当时的骨牌多由牙骨制成,所以骨牌又有“牙牌”之称,民间则称之为“牌九”。

1849年8月16日,一位名叫多米诺的意大利传教士把这种骨牌带回了米兰。作为最珍贵的礼物,他把骨牌送给了小女儿。多米诺为了让更多的人玩上骨牌,制作了大量的木制骨牌,并发明了各种的玩法。不久,木制骨牌就迅速地在意大利及整个欧洲传播,骨牌游戏成了欧洲人的一项高雅运动。

后来,人们为了感谢多米诺给他们带来这么好的一项运动,就把这种骨牌游戏命名为“多米诺”。到19世纪,多米诺已经成为世界性的运动。在非奥运项目中,它是知名度最高、参加人数最多、扩展地域最广的体育运动。

从那以后,“多米诺”成为一种流行用语。在一个相互联系的系统中,一个很小的初始能量就可能产生一连串的连锁反应,人们就把它们称为“多米诺骨牌效应”或“多米诺效应”

头上掉一根头发,很正常;再掉一根,也不用担心;还掉一根,仍旧不必忧虑……长此以往,一根根头发掉下去,最后秃头出现了。哲学上叫这种现象为“秃头论证”。

往一匹健壮的骏马身上放一根稻草,马毫无反应;再添加一根稻草,马还是丝毫没有感觉;又添加一根……一直往马儿身上添稻草,当最后一根轻飘飘的稻草放到了马身上后,骏马竟不堪重负瘫倒在地。这在社会研究学里,取名为“稻草原理”。

第一根头发的脱落,第一根稻草的出现,都只是无足轻重的变化。当是当这种趋势一旦出现,还只是停留在量变的程度,难以引起人们的重视。只有当它达到某个程度的时候,才会引起外界的注意,但一旦“量变”呈几何级数出现时,灾难性镜头就不可避免地出现了!

多米诺骨牌效应告诉我们:一个最小的力量能够引起的或许只是察觉不到的渐变,但是它所引发的却可能是翻天覆地的变化。这有点类似于蝴蝶效应,但是比蝴蝶效应更注重过程的发展与变化。

第一棵树的砍伐,最后导致了森林的消失;一日的荒废,可能是一生荒废的开始;第一场强权战争的出现,可能是使整个世界文明化为灰烬的力量。这些预言或许有些危言耸听,但是在未来我们可能不得不承认它们的准确性,或许我们惟一难以预见的是从第一块骨牌到最后一块骨牌的传递过程会有多久。

有些可预见的事件最终出现要经历一个世纪或者两个世纪的漫长时间,但它的变化已经从我们没有注意到的地方开始了。
\section{凡勃伦效应}
\label{sec-35}


美国经济学家凡勃伦提出凡勃伦效应:商品价格定得越高越能畅销。它是指消费者对一种商品需求的程度因其标价较高而不是较低而增加。它反映了人们进行挥霍性消费的心理愿望。

款式、皮质差不多的一双皮鞋,在普通的鞋店卖80元,进入大商场的柜台,就要卖到几百元,却总有人愿意买。1.66万元的眼镜架、6.88万元的纪念表、168万元的顶级钢琴,这些近乎“天价”的商品,往往也能在市场上走俏。其实,消费者购买这类商品的目的并不仅仅是为了获得直接的物质满足和享受,更大程度上是为了获得心理上的满足。这就出现了一种奇特的经济现象,即一些商品价格定得越高,就越能受到消费者的青睐。由于这一现象最早由美国经济学家凡勃伦注意到,因此被命名为“凡勃伦效应”。

随着社会经济的发展,人们的消费会随着收入的增加,而逐步由追求数量和质量过渡到追求品位格调。了解了“凡勃伦效应”,我们也可以利用它来探索新的经营策略。比如凭借媒体的宣传,将自己的形象转化为商品或服务上的声誉,使商品附带上一种高层次的形象,给人以“名贵”和“超凡脱俗”的印象,从而加强消费者对商品的好感。

有一天,一位禅师为了启发他的门徒,给他一块石头,叫他去蔬菜市场,并且试着卖掉它,这块石头很大,很美丽。但是师父说:“不要卖掉它,只是试着卖掉它。注意观察,多问一些人,然后只要告诉我在蔬菜市场它能卖多少。”这个人去了。在菜市场,许多人看着石头想:它可作很好的小摆件,我们的孩子可以玩,或者我们可以把它当作称菜用的秤砣。于是他们出了价,但只不过几个小硬币。那个人回来。他说:“它最多只能卖几个硬币。”师父说:“现在你去黄金市场,问问那儿的人。但是不要卖掉它,光问问价。”从黄金市场回来,这个门徒很高兴,说:“这些人太棒了。他们乐意出到1000块钱。”

师父说:“现在你去珠宝市场那儿,低于50万不要卖掉。”他去了珠宝商那儿。他简直不敢相信,他们竟然乐意出5万块钱,他不愿意卖,他们继续抬高价格——他们出到10万。但是这个门徒说:“这个价钱我不打算卖掉它。”他们说:“我们出20万、30万!”这个门徒说:“这样的价钱我还是不能卖,我只是问问价。”
虽然他觉得不可思议:“这些人疯了!”他自己觉得蔬菜市场的价已经足够了,但是没有表现出来。最后,他以50万的价格把这块石头卖掉了。

他回来,师父说:“不过现在你明白了,这个要看你是不是有试金石、理解力。如果你不要更高的价钱,你就永远不会得到更高的价钱。”

在这个故事里,师父要告诉徒弟是关于实现人生价值的道理,但是从门徒出售石头的过程中,却反映出一个经济规律:凡勃伦效应。
\section{非零和效应}
\label{sec-36}


零和效应之意是:实力相当的双方在谈判时做出大体相等的让步,方可取得结果,亦即每一方所得与所失的代数和大致为零,谈判便可成功。然而,人类社会发展的历程越来越走向“非零和”也就是我们现在所说的双赢。 “非零和效应”对学校管理的启示是:要向教师不断灌输“合作行为”的重要意义,尤其在当前课程改革的过程中,要大力提倡“师生合作”和 “师师合作”,力求取得“双赢”成效。
\section{飞去来器效应}
\label{sec-37}


在社会心理学中,人们把行为反应的结果与预期目标完全相反的现象,称为“飞去来器效应”即“飞镖效应”。

这好比用力把飞去来器往一个方向掷,结果它却飞向了相反的方向。飞去来器为澳洲土著使用的一种抛出去又会重新回来的武器。此处借喻情绪逆反的心理现象。原苏联心理学家纳季控什维制首先提出的。

日常工作与生活中常会发生这种飞去来器效应。例如在宣传一种不能使人接受的观点时,假如宣传者对这种观点做出肯定的评价并竭力说服听众接受,其结果反而使听众越来越反感,使听众的信念朝着宣传的相反方向发展,距离宣传的观点更远,从而导致宣传工作的彻底失败。又如为了把学习成绩提升上去,有些学生拼命加班加点和开夜车、搞题海战术和疲劳战术,弄得整天头昏脑胀的,毫无学习效率可言,结果考试成绩适得其反,一败涂地。又如有位教师拖堂引起的学生“情绪逆反”现象,情况是这样的:已是上午第四节课了。同学们都期望着教师能早点下课,最起码是按时下课,因为他们实在有点疲劳了。但是,化学教师还没有察觉到学生的心理反应,还一个劲地往下讲。下课铃声响了。他仍津津有味地讲着课。看得出来,这位教师是位认真负责的教师,干劲十足,毫不马虎。但学生听课的劲越来越差,开始还认真听讲,继而心不在焉,东张西望,最后交头接耳,传递纸片,甚至故意咳嗽,搬动桌椅,打哈欠,整个教室骚动起来。弄得这位教师丈二和尚摸不着头脑。

为什么会如此多的发生飞去来器效应呢?其原因兹分述如下:

一是目标与手段不协调一致。目标是我们行动反应后所要取得的东西,手段是我们实现目标的方式。目标与手段必须匹配,而且必须是最佳的匹配。上述几例“飞去来器效应”事实上就出在当事人把目标与手段相分离,只是把注意力盯在要达到的目标上,而忽视了手段的择优选取和最佳匹配的问题,以致手段与目标不匹配,因而引发了一系列中间反应,对实现目标起了干扰作用。

二是心厌引起的情绪逆反作用。常言道:“话不投机半句多。”也就是说,话不投机就会产生心厌现象,再加上强行灌输就更增加了厌烦情绪,以致情绪越来越向相反的方面发展。夫妻之间吵架,正在气头上,如果一方想对另一方解释,此时越解释越糟糕,因为他会认为你是在别有用心,你是在用计谋,而不是在陈述事实,因此,你越解释他越不爱听,越会来气,以致大闹一场。可见,话不投机须沉默,此时沉默就是金。
\section{改宗效应}
\label{sec-38}


美国社会心理学家哈罗德.西格尔有一个出色的研究,题目是“改宗的心理学效应”。研究表明,在一个问题对某人来说是十分重要的时候,如果他在这个问题上能使一个“反对者”改变意见而和自己的观点一致,他宁愿要那个“反对者”,而不要一个同意者。改宗效应”使我们明白:某些没有是非观念的“好好先生”之所以被人瞧不起,乃是因为他们给人一种没有能力的感觉;而不少敢于直言是非,勇于开展批评的人,最终所以能受到人们的喜爱,乃是因为他们给人一种富有才能的感染力。
\section{共生效应}
\label{sec-39}


自然界有这样一种现象:当一株植物单独生长时,显得矮小、单调,而与众多同类植物一起生长时,则根深叶茂,生机盎然。人们把植物界中这种相互影响、相互促进的现象,称之为“共生效应”。事实上,我们人类群体中也存在“共生效应”。英国“卡迪文实验室”从1901年至1982年先后出现了25位诺贝尔获奖者,便是“共生效应”一个杰出的典型。
\section{古烈治效应}
\label{sec-40}


这是一个美国笑话,说的是有一位美国前总统和夫人可尼基去一家农场参观养鸡舍,夫人看见公鸡在母鸡身上踩蛋,忽发奇想问陪同的农场主说:你能否告诉我公鸡一天在母鸡身上尽多少次“丈夫”的责任?答:时时尽责一日十余次。夫人说:请把结论告诉总统。农场主过去给总统刚一说完,总统问道:每次都在同一只母鸡身上尽责任吗?答:次次更换伴侣。总统说:请把结论转告夫人。

这个故事充分说明了男女思维的差异,男女都没有错,各人都有自己思考问题的角度。后来它就成了男人见异思迁喜新厌旧(或淡旧)的著名心理学效应了。
\section{光环效应,晕轮效应}
\label{sec-41}


晕轮效应最早是由美国著名心理学家爱德华.桑戴克于20世纪20年代提出的。他认为,人们对人的认知和判断往往只从局部出发,扩散而得出整体印象,也即常常以偏概全。一个人如果被标明是好的,他就会被一种积极肯定的光环笼罩,并被赋予一切都好的品质;如果一个人被标明是坏的,他就被一种消极否定的光环所笼罩,并被认为具有各种坏品质。这就好象刮风天气前夜月亮周围出现的圆环(月晕),其实呢,圆环不过是月亮光的扩大化而已。据此,桑戴克为这一心理现象起了一个恰如其分的名称“晕轮效应”,也称作“光环作用”。

心理学家戴恩做过一个这样的实验。他让被试看一些照片,照片上的人有的很有魅力,有的无魅力,有的中等。然后让被试在与魅力无关的特点方面评定这些人。结果表明,被试对有魅力的人比对无魅力的赋予更多理想的人格特征,如和蔼、沉着,好交际等。

晕轮效应不但常表现在以貌取人上,而且还常表现在以服装定地位、性格,以初次言谈定人的才能与品德等方面。在对不太熟悉的人进行评价时,这种效应体现得尤其明显。

从认知角度讲,晕轮效应仅仅抓住并根据事物的个别特征,而对事物的本质或全部特征下结论,是很片面的。因而,在人际交往中,我们应该注意告诫自己不要被别人的晕轮效应所影响,而陷入晕轮效应的误区。
\section{过度理由效应}
\label{sec-42}


每个人都力图使自己和别人的行为看起来合理,因而总是为行为寻找原因。一旦找到足够的原因,人们就很少再继续找下去,而且,在寻找原因时,总是先找那些显而易见的外在原因。因此,如果外部原因足以对行为做出解释时,人们一般就不再去寻找内部的原因了。这就是社会心理学上所说的“过度理由效应”。

在日常生活中我们常有这样的体验:亲朋好友帮助我们,我们不会觉得奇怪,因为“他是我的亲戚”、“他是我的朋友”,理所当然他们会帮助我们;但是如果一个陌生人向我们伸出援手,我们却会认为“这个人乐于助人”。因为我们无法用“亲戚”、“朋友”这样的外部理由来解释别人的行为,只能追究到他人格内部的这个原因。

过度理由效应的存在给我们两个启示:

第一,不要止步于任何外部理由,而要深入发掘外部理由背后的原因,哪怕这种理由看上去是一种无稽之谈。

一天,一个客户写信给美国通用汽车公司的庞帝雅克部门,抱怨道:他家习惯每天在饭后吃冰淇淋。最近买了一部新的庞帝雅克后,每次只要他买的冰淇淋是香草口味,从店里出来车子就发不动。但如果买的是其它口味,车子发动就很顺利。

庞帝雅克派一位工程师去查看究竟,发现确是这样。这位工程师当然不相信这辆车子对香草过敏。他经过深入了解后得出结论,这位车主买香草冰淇淋所花的时间比其它口味的要少。原来,香草冰淇淋最畅销,为便利顾客选购,店家就将香草口味的特别分开陈列在单独的冰柜,并将冰柜放置在店的前端;而将其它口味的冰淇淋放置在离收银台较远的地方。

深入查究,发现问题出在“蒸气锁”上。当这位车主买其它口味时,由于时间较长,引擎有足够的时间散热,重新发动时就没有太大的问题。买香草冰淇淋由于花的时间短,引擎还无法让“蒸气锁”有足够的散热时间。

第二,如果我们希望某种行为得以保持,就不要给它过于充分的外部理由。

处于管理岗位的人都会发现,奖励的刺激会在某种程度上促使别人保持高涨的热情,对于处于低潮中的人尤其如此。但是如果在很长一段时间里保持不变,就会使奖励成为工作的过度理由,一旦失去外在奖励或者奖励无法满足其需要时,结果就会反而不如从前。

激励是一种策略,更是一种艺术,它应包括精神上的沐泽,而不是单纯的物质刺激。使一个人持续不断的努力,应该激发其内在的动力,而不能只靠外在奖励。

心理学实验证明,表扬、鼓励和信任,往往能激发一个人的自尊心和上进心。但奖励的原则应是精神奖励重于物质奖励,否则易造成“为钱而工作”、的心态。同时奖励要抓住时机,掌握分寸,不断升华。管理者如果希望自己的员工努力工作,在给予恰当物质奖励的同时,还必须让职员认为他自己勤奋、上进,喜欢这份工作,喜欢这家公司,而不能简单地把工作与待遇挂钩。
\section{海格立斯效应}
\label{sec-43}


人生在世,人际间或群体间的摩擦、误解乃至纠葛恩怨总是在所难免,如果肩上扛着“仇恨袋”,心中装着“仇恨袋”,生活只会是如负重登山、举步维艰了,最后,只会堵死自己的路。这就是海格力斯效应。

海格力斯效应会使人陷入无休止的烦恼之中,错过人生中许多美丽的风景,再没有真正的快乐,在没有新的进步了。

以眼还眼,以牙还牙”,“以其人之道还治其人之身”,“你跟我过不去,我也让你不痛快”。被称为“海格力斯效应”。这是一种人际间或群体间存在的怨怨相报、致使仇恨越来越深的社会心理效应。

希腊神话故事中有位英雄大力士,叫海格力斯,一天,他走在坎坷不平的路上,看见脚边有个像鼓起的袋子样的东西,很难看,诲格力斯便踩了那东西一脚。谁知那东西不但没被海格力斯一脚踩破,反而膨胀起来,并成倍成倍地加大,这激怒了英雄海格力斯。他顺手操起—根碗口粗的木棒砸那个怪东西,好家伙,那东西竟膨胀到把路也堵死了。海格力斯奈何不了他,正在纳闷,一位圣者走到海格力斯跟前对他说:“朋友.快别动它了,忘了它,离它远去吧。它叫仇恨袋,你不惹它,它便会小如当初;你若侵犯它,它就会膨胀起来与你敌对到底。”

仇恨正如海格力斯所遇到的这个袋子,开始很小,如果你忽略它,矛盾化解,它会自然消失;如果你与它过不去,加恨于它,它会加倍地报复。

生活中这种现象比比皆是:两人出于误解或嫉妒,闹了矛盾,你若想报复对方,便会加深对方对你的仇恨.于是他会更挖空心思地加害于你;你若再不罢休,他会更恶毒地报复你,直到两败俱伤。
\section{黑暗效应}
\label{sec-44}


在光线比较暗的场所,约会双方彼此看不清对方表情,就很容易减少戒备感而产生安全感。在这种情况下,彼此产生亲近的可能性就会远远高于光线比较亮的场所。心理学家将这种现象称之为“黑暗效应”。

譬如在一些例如酒吧、练哥厅等场所,我们往往容易对异性产生异样的好感,这就是黑暗效应的最形象的表现。

有个这样的案例:有一位男子钟情于一位女子,但每次约会,他总觉得双方谈话不投机。有一天晚上,他约那位女子到一家光线比较暗的酒吧,结果这次谈话融洽投机。从此以后,这位男子将约会的地点都选择在光线比较暗的酒吧。几次约会之后,他俩终于决定结下百年之好。

社会心理学家研究后的结论是,在正常情况下,一般的人都能根据对方和外界条件来决定自己应该掏出多少心里话,特别是对还不十分了解但又愿意继续交往的人,既有一种戒备感,又会自然而然地把自己好的方面尽量展示出来,把自己弱点和缺点尽量隐藏起来。因此,这时双方就相对难以沟通。
\section{华盛顿合作规律}
\label{sec-45}


华盛顿合作规律说的是:一个人敷衍了事,两个人互相推诿,三个人则永无事成之日。多少有点类似于我们“三个和尚”的故事。

人与人的合作,不是人力简单相加,而是要复杂和微妙得多。在这种合作中,假定每个人的能力都为1,那么,10个人合作结果有时比10大得多,有时甚至比1还小。因为人不是静止物,而更像方向各异的能量,相互推动时,自然事半功倍,相互抵触时,则一事无成。

我们传统的管理理论中,对合作研究的并不多,最直观的反映就是,目前的大多数的管理制度和行为都是致力于减少人力的无畏消耗,而非利用组织提高人的效能。换言之,不放说管理的主要目的不是让每个人做到最好,而是避免过多内耗。

钓过螃蟹的人或许都知道,篓子中放一群螃蟹,不必盖上盖子,螃蟹是爬不出来的。因为只要有一只想往上爬,其他螃蟹便会纷纷攀附在它的身上,把它也拉下来,最后没有一只能够出去。

寓言解读

人与人的合作不是力气的简单相加,而要微妙和复杂得多。在人与人的合作中,假定每个人的能量都为1,那么10个人的能量可能比10大得多,也可能甚至比1还小。因为人的合作不是静止的,它更像方向各异的能量,互相推动时自然事倍功半,相互抵触时则一事无成。合作是一个问题,如何合作也是一个问题。企业里常会有一些人,嫉妒别人的成就与杰出表现,天天想尽办法进行破坏与打压。如果企业不把这种人除去,久而久之,组织里就只下一群互相牵制、毫无生产力的“螃蟹”。

与此类似的是邦尼人力定律:“一个人一分钟可以挖一个洞,60个人一秒钟挖不了一个洞。”
\section{蝴蝶效应}
\label{sec-46}


先从美国麻省理工学院气象学家洛伦兹(Lorenz)的发现谈起。为了预报天气,他用计算机求解仿真地球大气的13个方程式。为了更细致地考察结果,他把一个中间解取出,提高精度再送回。而当他喝了杯咖啡以后回来再看时竟大吃一惊:本来很小的差异,结果却偏离了十万八千里!计算机没有毛病,于是,洛伦兹(Lorenz)认定,他发现了新的现象:“对初始值的极端不稳定性”,即:“混沌 ”,又称“蝴蝶效应”,亚洲蝴蝶拍拍翅膀,将使美洲几个月后出现比狂风还厉害的龙卷风!这个发现非同小可,以致科学家都不理解,几家科学杂志也都拒登他的文章,认为“违背常理”:相近的初值代入确定的方程,结果也应相近才对,怎幺能大大远离呢!线性,指量与量之间按比例、成直线的关系,在空间和时间上代表规则和光滑的运动;而非线性则指不按比例、不成直线的关系,代表不规则的运动和突变。如问:两个眼睛的视敏度是一个眼睛的几倍?很容易想到的是两倍,可实际是 6-10倍!这就是非线性:1+1不等于2。激光的生成就是非线性的!当外加电压较小时,激光器犹如普通电灯,光向四面八方散射;而当外加电压达到某一定值时,会突然出现一种全新现象:受激原子好象听到“向右看齐”的命令,发射出相位和方向都一致的单色光,就是激光。非线性的特点是:横断各个专业,渗透各个领域,几乎可以说是:“无处不在时时有。” 如:天体运动存在混沌;电、光与声波的振荡,会突陷混沌;地磁场在400万年间,方向突变16次,也是由于混沌。甚至人类自己,原来都是非线性的:与传统的想法相反,健康人的脑电图和心脏跳动并不是规则的,而是混沌的,混沌正是生命力的表现,混沌系统对外界的刺激反应,比非混沌系统快。由此可见,非线性就在我们身边,躲也躲不掉了。 1979年12月,洛伦兹(Lorenz)在华盛顿的美国科学促进会的一次讲演中提出:一只蝴蝶在巴西扇动翅膀,有可能会在美国的德克萨斯引起一场龙卷风。他的演讲和结论给人们留下了极其深刻的印象。从此以后,所谓“蝴蝶效应”之说就不胫而走,名声远扬了。

“蝴蝶效应”之所以令人着迷、令人激动、发人深省,不但在于其大胆的想象力和迷人的美学色彩,更在于其深刻的科学内涵和内在的哲学魅力。混沌理论认为在混沌系统中,初始条件的十分微小的变化经过不断放大,对其未来状态会造成极其巨大的差别。我们可以用在西方流传的一首民谣对此作形象的说明。这首民谣说:丢失一个钉子,坏了一只蹄铁;坏了一只蹄铁,折了一匹战马;折了一匹战马,伤了一位骑士;伤了一位骑士,输了一场战斗;输了一场战斗,亡了一个帝国。马蹄铁上一个钉子是否会丢失,本是初始条件的十分微小的变化,但其“长期”效应却是一个帝国存与亡的根本差别。这就是军事和政治领域中的所谓“蝴蝶效应”。有点不可思议,但是确实能够造成这样的恶果。一个明智的领导人一定要防微杜渐,看似一些极微小的事情却有可能造成集体内部的分崩离析,那时岂不是悔之晚矣?
\section{环境效应}
\label{sec-47}


当回忆时的情境和学习时所情境完全一样时,记忆效果最佳。通常把这一现象称为环境效应。如果你要在某一课堂进行考试,那么在这个课堂里学习材料比在图书馆或宿舍里学习更为有利。这个概念是和刺激的泛化紧密相联的。当然,环境是指一个人学习和回忆时的周围情境,如房子的大小、墙壁的颜色、噪音的量等等。近来,这个概念已经扩大到包括学习者学习和回忆时的生理状态。从某种意义上说,一个人的身体也是他所处环境的一部分。因此,为了得到最佳的记忆,产生积极的环境效应,一个人的身体状况在学习和回忆时也应尽可能地相似。
\section{霍布森选择效应}
\label{sec-48}


1631年,英国剑桥商人霍布森贩马时,把马匹放出来供顾客挑选,但附加一个条件即只许挑选最靠近门边的那匹马。显然,加上这个条件实际上就等于不让挑选。对这种没有选择余地的所谓“选择”,后人讥讽为“霍布森选择效应”。

从社会心理学关于自我选择的角度来说,“霍布森选择效应”显然是社会角色扮演者的一大忌讳。谁如果陷入“霍布森选择效应”的困境,谁就不可能进行创造性的学习、生活和工作。

道理很简单:好与坏、优与劣,都是在对比中发现的,只有拟定出一定数量和质量的可能方案供对比选择,判断、决策才能做到合理。一个人在进行判断、决策的时候,他必须在多种可供选择的方案中决定取舍。如果一种判断只需要说“是”或“不”的话,这能算是判断吗?

只有在许多可保选择的方案中进行研究,并能够在对其了解的某础上判断,才算得上判断。在我们还没有考虑各种可供选择的方法之前,”我们的思想是闭塞的。倘若只有一个方案;就无法对比,也就难以辨认其优劣。因此,没有选择余地的选择,就等于无法判断,等于扼杀创造。鉴此,有成效的社会角色扮演者,其头脑中总是有多种“备择方案”,总是高度重视“多方案选择”,因为他们始终认为要有许多种可供采用的衡量方法,这样才可以从中选择一种适当的方法。在这里,生活的辩证法正如一句格言所说的:“如果你感到似乎只有条路可走,那很可能这条路就是走不通的。”
\section{霍桑效应}
\label{sec-49}


霍桑实验是一项以科学管理的逻辑为基础的实验。从1924年开始到1932年结束,在将近8年的时间内,前后共进行过两个回合:第一个回合是从1924年11月至1927年5月,在美国国家科学委员会赞助下进行的;第二个回合是从1927年至1932年,由梅奥主持进行。整个实验前后经过了四个阶段。

阶段一,车间照明实验——“照明实验”

照明实验的目的是为了弄明白照明的强度对生产效率所产生的影响。这项实验前后共进行了两年半的时间。然而照明实验进行得并不成功,其结果令人感到迷惑不解,因此有许多人都退出了实验。

阶段二,继电器装配实验——“福利实验”

1927年梅奥接受了邀请,并组织了一批哈佛大学的教授成立了一个新的研究小组,开始了霍桑的第二阶段的“福利实验”。

“福利实验”的目的是为了能够找到更有效地控制影响职工积极性的因素。梅奥他们对实验结果进行归纳,排除了四种假设:(1)在实验中改进物质条件和工作方法,可导致产量增加;(2)安排工间休息和缩短工作日,可以解除或减轻疲劳;(3)工间休息可减少工作的单调性;(4)个人计件工资能促进产量的增加。最后得出“改变监督与控制的方法能改善人际关系,能改进工人的工作态度,促进产量的提高”的结论。

阶段三,大规模的访谈计划——“访谈实验”

既然实验表明管理方式与职工的土气和劳动生产率有密切的关系,那么就应该了解职工对现有的管理方式有什么意见,为改进管理方式提供依据。于是梅奥等人制定了一个征询职工意见的访谈计划,在1928年9月到1930年5月不到两年的时间内,研究人员与工厂中的两万名左右的职工进行了访谈。

在访谈计划的执行过程中,研究人员对工人在交谈中的怨言进行分析,发现引起他们不满的事实与他们所埋怨的事实并不是一回事,工人在表述自己的不满与隐藏在心理深层的不满情绪并不一致。比如,有位工人表现出对计件工资率过低不满意,但深入地了解以后发现,这位工人是在为支付妻子的医药费而担心。

根据这些分析,研究人员认识到,工人由于关心自己个人问题而会影响到工作的效率。所以管理人员应该了解工人的这些问题,为此,需要对管理人员,特别是要对基层的管理人员进行训练,使他们成为能够倾听并理解工人的访谈者,能够重视人的因素,在与工人相处时更为热情、更为关心他们,这样能够促进人际关系的改善和职工士气的提高。

阶段四,继电器绕线组的工作室实验——“群体实验”

这是一项关于工人群体的实验,其目的是要证实在以上的实验中研究人员似乎感觉到在工人当中存在着一种非正式的组织,而且这种非正式的组织对工人的态度有着极其重要的影响。

实验者为了系统地观察在实验群体中工人之间的相互影响,在车间中挑选了14名男职工,其中有9名是绕线工,3名是焊接工,2名是检验工,让他们在一个单独的房间内工作。

实验开始时,研究人员向工人说明,他们可以尽力地工作,因为在这里实行的是计件工资制。研究人员原以为,实行了这一套办法会使得职工更为努力地工作,然而结果却是出乎意料的。事实上,工人实际完成的产量只是保持在中等水平上,而且每个工人的日产量都是差不多的。根据动作和时间分析,每个工人应该完成标准的定额为7312个焊接点,但是工人每天只完成了6000~6600个焊接点就不干了,即使离下班还有较为宽裕的时间,他们也自行停工不干了。这是什么原因呢?研究者通过观察,了解到工人们自动限制产量的理由是:如果他们过分努力地工作,就可能造成其他同伴的失业,或者公司会制定出更高的生产定额来。

研究者为了了解他们之间能力的差别,还对实验组的每个人进行了灵敏度和智力测验,发现3名生产最慢的绕线工在灵敏度的测验中得分是最高的。其中1名最慢的工人在智力测验上是排行第一,灵敏度测验排行第三。测验的结果和实际产量之间的这种关系使研究者联想到群体对这些工人的重要性。1名工人可以因为提高他的产量而得到小组工资总额中较大的份额,而且减少失业的可能性,然而这些物质上的报酬却会带来群体非难的惩罚,因此每天只要完成群体认可的工作量就可以相安无事了。即使在一些小的事情上也能发现工人之间有着不同的派别。绕线工就一个窗户的开关问题常常发生争论,久而久之,就可以看出他们之间不同的派别了。

实验结论

1、改变工作条件和劳动效率没有直接关系;

2、提高生产效率的决定因素是员工情绪,而不是工作条件;职工是“社会人”。霍桑等人是在“经济人”的人性模式下进行试验的,试图找出工作条件与生产效率的关系。但随着实验的深入,“经济人”的假设受到动摇。霍桑最终提出,人性模式是“社会人”,即职工不单纯追求经济收入,还有社会方面和心理方面的需求,如同后来的马斯洛指出的一样,需求是多层次的多方面的。因此,必须首先从社会心理方面来鼓励工人提高劳动生产率。

3、关心员工的情感和员工的不满情绪,有助于提高劳动生产率。新的领导方式在于提高职工的满足度。霍桑认为,管理者的目的在于使人们为实现组织的共同目标而合作。为了实现合作,必须发展一种新的领导方式。在这种新的领导方式下,管理者必须一方面为满足成员物质的、经济的需要而进行生产和分配物质资料,即发挥技术性技能;另一方面,为实现满足成员物质需要的目标而确保成员间的自发性合作,使每个人获得人类的满足,即发挥社会性技能。

4、企业来除了正式组织,还存在非正式组织。霍桑认为企业中不仅存在“正式组织”,还存在人们在共同劳动中形成的非正式团体,他们有自己的规范、情感和倾向,并且左右着团体内每个成员的行为。譬如湘军中的哥老会,学校里的同乡会。“非正式组织”对组织既有利,也有弊。管理人员要想实施有效的管理,要注意在非正式组织的感情逻辑和正式组织的效率逻辑之间保持平衡
\section{棘轮效应}
\label{sec-50}


棘轮效应,又称制轮作用,是指人的消费习惯形成之后有不可逆性,即易于向上调整,而难于向下调整。尤其是在短期内消费是不可逆的,其习惯效应较大。这种习惯效应,使消费取决于相对收入,即相对于自己过去的高峰收入。

消费者易于随收入的提高增加消费,但不易于收入降低而减少消费,以致产生有正截距的短期消费函数。这种特点被称为棘轮效应。

在科学共同体中,也存在这样的一种效应,即科学精英一旦因为自己的工作而获得某种承认与地位,就再也不会退回到原来的地位,就像有棘爪防止倒转的棘轮一样。“棘轮效应”表明科学界分层结构中的流动是单向的,科学家只会升迁不会降格,这种效应在科学金字塔结构的越高层表现得越突出。朱克曼通过对美国诺贝尔奖获得者的研究指出:“一旦成为一个诺贝尔奖获得者,不论是好是歹,都将稳固地居于科学界的精英行列。”

实际上棘轮效应可以用宋代政治家和文学家司马光一句著名的话来概括:由俭入奢易,由奢入俭难。
\section{缄默效应}
\label{sec-51}


在人际交往中,做到基本上不使用强迫手段并不难。人们虽然会在皮鞭面前屈服,可那不过是表面上的服从,内心却充满了反叛、仇恨的复杂感情。不仅在感情上,在日常生活中也存在着正确信息的传播受到限制的现象。

对统治者,人们大都愿意挑对方喜欢的、迎合对方的话来说,尽量避免说让对方不快或有可能降低自身价值的话。这就叫''缄默(MUM)效应''。职员在工作上犯了错误后因为害怕上司的威严而保持''缄默'',这样上司便得不到正确的信息,结果就会因错误得不到及时纠正而造成日后的重大损失。

从长远考虑,无论是在感情上还是在工作上都应尽量不使用强制手段。但对于上司或父母、教师等身份的人来说,强制手段不失为一种对下属或晚辈、学生发挥作用的简单快捷的好办法。同时,越是对自己的才干和人格魅力没有信心的人越会行使强制手段,因为他们自认为没有其他行之有效的办法去说服别人。就像风和太阳的寓言所讲的那样,光靠猛烈的暴风雨是掀不掉人身上的衣服的,而平时以礼相待,在认为有必要发作时点到为止,这才是最有效的。
\section{结伴效应}
\label{sec-52}


是指两个人或几个人结伴从事相同的一项活动时(并不进行竞赛)相互之间会产生刺激作用,提高活动效率。例如:学生在一起作作业比独立完成作业的效率高。可以相应组成学习小组。教师应注意课上的效率,注意时间分配。
\section{进门坎效应}
\label{sec-53}


在心理学中,“进门坎效应”指的是如果一个人接受了他人的微不足道的一个要求,为了避免认知上的不协或是想给他人留下前后一致的印象,就极有可能接受其更大的要求。关于这个效应的理论是美国社会心理学家弗里德曼与弗雷瑟在实验中提出的。实验过程是这样的:实验者让助手到两个居民区劝说人们在房前竖一块写有“小心驾驶”的大标语牌。他们在第一个居民区直接向人们提出这个要求,结果遭到很多居民的拒绝,接受的仅为被要求者的17%。而在第二个居民区,实验者先请求众居民在一份赞成安全行驶的请愿书上签字,这是很容易做到的小小要求,几乎所有的被要求者都照办了。他们在几周后再向这些居民提出竖牌的有关要求,这次的接受者竟占被要求者的55%。为什么同样都是竖牌的要求,却会产生如此截然不同的结果呢?

研究者认为,人们拒绝难以做到的或违反个人意愿的请求是很自然的,但一个人若是对于某种小请求找不到拒绝的理由,就会增加同意这种要求的倾向;而当他卷入了这项活动的一小部分以后,便会产生自己以行动来符合所被要求的各种知觉或态度。这时如果他拒绝后来的更大要求,自己就会出现认知上的不协调,而恢复协调的内部压力会支使他继续干下去或做出更多的帮助,并使态度的改变成为持续的过程。运用这个方法来使别人接受自己的要求的现象,心理学上叫做“进门坎技术”。

如果在日常生活中学会运用这样的技巧来与人们进行沟通,就可能更易于得到对方的配合与支持。比如:交警在执勤时,发现有人违章驾驶,截停违章司机后用严厉的言语训斥他,或粗暴地责令其交出驾驶执照以登记罚款,这样的态度很容易造成司机心理上的抵触,从而人为地增加了工作的难度。这时,不妨考虑根据“进门坎效应”的原则,换一种沟通方式与司机进行交流。这里提供一种思路供参考:如截停当事人后,首先微笑并敬礼示意,再对他进行简短的交通安全常识宣传,然后指出其属于哪一种违章,可能会导致什么样的后果,尽量从当事人自身安全的角度来劝说,使他真正意识到自己的过错。

最好还能配合使用小的宣传彩页或安全常识小卡片,让市民清楚知道自己违反了哪一条驾驶规则,设计一套“友情提示卡通图案”让他从所犯错误的“肇事者”卡通系列贴纸里选取相应的那种贴在方向盘上,以便在以后的驾驶中随时提醒自己不要违反交通规则。这样的方法既能达到教育管理的目的,又能在和谐的氛围中形成比较良好的警民关系。即使必须采取罚款等措施的,经过这样的铺垫,也有利于使对方心悦诚服,采取主动配合的态度。

其实“进门坎效应”也能在生活的各个方面中得到运用,这需要我们慢慢去摸索和体验。可以在与周围人们的交往中使用,让他人从心底里愿意接受你提出的观点。在实际生活里灵活地用好这个心理学小原理,经由沟通交往的过程,一步步地迈进他人的“心田”,给对方留下亲切友好的印象。
\section{禁果效应}
\label{sec-54}


禁果效应也叫做“罗密欧与朱丽叶效应”,越是禁止的东西,人们越要得到手。这与人们的好奇心与逆反心理有关。

在生活中常常会遇到这样的情况:你越想把一些事情或信息隐瞒住不让别人知道,越会引来他人更大的兴趣和关注,人们对你隐瞒的东西充满好奇和窥探的欲望,甚至千方百计通过别的渠道试图获得这些信息。而一旦这些信息突破你的掌握,进入了传播领域,会因为它所具有的“神秘”色彩被许多人争相获取,并产生一传十、十传百的效果,从而与你隐瞒该信息的愿望背道而驰。这一现象被称作传播中的“禁果效应”。所谓禁果效应,指一些事物因为被禁止,反而更加吸引人们的注意力,使更多地人参与或关注。有一句谚语:“禁果格外甜”,就是这个道理。

“禁果效应”存在的心理学依据在于,无法知晓的“神秘”的事物,比能接触到的事物对人们有更大的诱惑力,也更能促进和强化人们渴望接近和了解的诉求。我们常说的“吊胃口”、“卖关子”,就是因为受传者对信息的完整传达有着一种期待心理,一旦关键信息的阙如在受传者心里形成了接受空白,这种空白就会对被遮蔽的信息产生了强烈的召唤。这种“期待-召唤”结构就是“禁果效应”存在的心理基础。特别在涉及公众切身利益的问题上,人们恐惧的往往不是确定的事实,而是不确定的、难以知晓的事情,在无法知晓和渴望知晓的搏杀过程中,公众会因为恐惧心理而像饕餮一样渴望获得信息。
\section{近因效应}
\label{sec-55}


一由于最近了解的东西掩盖了对某人一贯了解的心理现象叫做近因效应。心理学家研究表明,对陌生人的知觉,第一印象有更大的作用;而对于熟悉的人,对他们的新异表现容易产生近因效应。近因效应在学生交往中也是常见的,例如两个学生本来相处得很好,甲对乙堪称关怀备至,可是却因最近一次''得罪''了乙,就遭到乙的痛恨,这就属于近因效应的作用。同样,在学生的成长过程中,大部分人都不可能始终给人留下很好的第一印象,这就要求班主任一是要不断提高自己的能力,增强自身的吸引力;二是要不断鼓励学生进步,让学生能以新的姿态展现在外人面前,不断激励学生进步。

二所谓近因效应,指的是在交往过程中最近一次接触给人留下的印象对社会知觉的影响作用。

首因效应一般在对陌生人的知觉中起重要作用,而近因效应则在熟悉的人之间起重要作用。在经常接触、长期共事的人之间,彼此之间往往都将对方的最后一次印象作为认识与评价的依据。并常常使彼此的人际交往和人际关系发生质和量的变化。现实生活中的友谊破裂、夫妻反目、朋友绝交等,都与近因效应有关。

近因效应使我们仅仅根据人的一时一事去评价一个人或人际关系,割裂了历史与现实、现象与本质的关系,妨碍我们客观地、历史地看待人和客观事实,常常造成人与人之间的心理冲突,影响了我们对人和事作出客观、正确的评价和判断,对我们的实际工作和生活有着消极的影响。
\section{竞争优势效应}
\label{sec-56}


在双方有共同的利益的时候,人们也往往会优先选择竞争,而不是选择对双方都有利的“合作”。这种现象,被心理学家称为“竞争优势效应”。

心理学上有这样一个经典的实验:让参与实验的学生两两结合,但是不能商量,各自在纸上写下来自己想得到得钱数。如果两个人的钱数之和刚好等于100或者小于100,那么,两个人就可以得到自己写在纸上的钱数;如果两个人的钱数之和大于100,比如说是120,那么,他们两就要分别付给心理学家60元。

结果如何呢?几乎没有哪一组的学生写下的钱数之和小于100,当然他们就都得付钱。

社会心理学家认为,人们与生俱来有一种竞争的天性,每个人都希望自己比别人强,每个人都不能容忍自己的对手比自己强,因此,人们在面对利益冲突的时候,往往会选择竞争,拼个两败俱伤也在所不惜;就是在双方有共同的利益的时候,人们也往往会优先选择竞争,而不是选择对双方都有利的“合作”。

除此之外,心理学家还认为,沟通的缺乏也是人们选择竞争的一个重要原因。如果双方曾经就利益分配问题进行商量,达成共识,合作的可能性就会大大增加。如果在上面的实验中允许参加实验的两个人互相商量,或者两个人对对方的选择有充分的把握,结果必然会是另外一个样子。
\section{酒与污水定律}
\label{sec-57}


管理学上一个有趣的定律叫“酒与污水定律”,意思是一匙酒倒进一桶污水,得到的是一桶污水;把一匙污水倒进一桶酒里,得到的还是一桶污水。显而易见,污水和酒的比例并不能决定这桶东西的性质,真正起决定作用的就是那一勺污水,只要有它,再多的酒都成了污水。

几乎在任何组织里,都存在几个难弄的人物,他们存在的目的似乎就是为了把事情搞糟。他们到处搬弄是非,传播流言、破坏组织内部的和谐。最糟糕的是,他们像果箱里的烂苹果,如果你不及时处理,它会迅速传染,把果箱里其它苹果也弄烂,''烂苹果''的可怕之处在于它那惊人的破坏力。一个正直能干的人进入一个混乱的部门可能会被吞没,而一个人无德无才者能很快将一个高效的部门变成一盘散沙。组织系统往往是脆弱的,是建立在相互理解、妥协和容忍的基础上的,它很容易被侵害、被毒化。破坏者能力非凡的另一个重要原因在于,破坏总比建设容易。一个能工巧匠花费时日精心制作的陶瓷器,一头驴子一秒钟就能毁坏掉。如果拥有再多的能工巧匠,也不会有多少像样的工作成果。如果你的组织里有这样的一头驴子,你应该马上把它清除掉;如果你无力这样做,你就应该把它拴起来。

在企业中,总难免会有污水,而污水又总会给企业带来各种各样的矛盾和冲突,这就要求企业管理者要掌握酒与污水的冲突与协调的技巧。酒和污水在一个组织中也存在着相互博弈的过程。发现人才、善用人才,在人才大战中占得先机,是精明的企业管理者引领企业走向成功的重要砝码,而有效运用酒和污水定律,则是组织一个高效团队的最佳途径。现代企业管理的一项带有根本性的任务,就是对团体中的人才加以指引和筛选,剔除具有破坏力“污水”,使合格者的力量指向同一目标,这就是人才的运作。
\section{角色效应}
\label{sec-58}


现实生活中,人们以不同的社会角色参加活动,这种因角色不同而引起的心理或行为变化被称为角色效应。

人的角色的形成首先是建立在社会和他人对角色的期待上的。

角色效应的产生要经历三个过程:

一是社会和他人对角色的期待。就目前教育情况而言,普遍存在着对孩子社会角色期望的偏差,比如“好学生”在不少家长教师心目中就是“学习好”,“学习好”就是分数高。由此引发出教育指导思想的失误,造成教育重智育,轻德、体、美和劳动教育的倾向,降低了教育质量;

二是对自己扮演的社会角色的认知。

三是在角色期望和角色认知的基础上,通过具体的角色规范,实现角色期待和角色行为。
\section{刻板效应,定型效应}
\label{sec-59}


刻板效应,又称定型效应,是指人们用刻印在自己头脑中的关于某人、某一类人的固定印象,以此固定印象作为判断和评价人依据的心理现象。

有些人总是习惯于把人进行机械的归类,把某个具体的人看作是某类人的典型代表,把对某类人的评价视为对某个人的评价,因而影响正确的判断。刻板印象常常是一种偏见,人们不仅对接触过的人会产生刻板印象,还会根据一些不是十分真实的间接资料对未接触过的人产生刻板印象,例如:老年人是保守的,年轻人是爱冲动的;北方人是豪爽的,南方人是善于经商的;英国人是保守的,美国人是热情的等等。
\section{空白效应}
\label{sec-60}


心理实验表明,在演讲的过程中,适当地留一些空白,会取得良好的演讲效果,这就是空白效应。
\section{冷热水效应}
\label{sec-61}


一杯温水,保持温度不变,另有一杯冷水,一杯热水。当先将手放在冷水中,再放到温水中,会感到温水热;当先将手放在热水中,再放到温水中,会感到温水凉。同一杯温水,出现了两种不同的感觉,这就是冷热水效应。这种现象的出现,是因为人人心里都有一杆秤,只不过是秤砣并不一致,也不固定。随着心理的变化,秤砣也在变化。当秤砣变小时,它所称出的物体重量就大,当秤砣变大时,它所称出的物体重量就小。人们对事物的感知,就是受这秤砣的影响。人际交往中,要善于运用这种冷热水效应。

一、运用冷热水效应去获得对方的好评

人处世上,难免有事业上滑坡的时候,难免有不小心伤害他人的时候,难免有需要对他人进行批评指责的时候,在这些时候,假若处理不当,就会降低自己在他人心目中的形象。如果巧妙运用冷热水效应,就不但不会降低自己的形象,反而会获得他人一个好的评价。当事业上滑坡的时候,不妨预先把最糟糕的事态委婉地告诉别人,以后即使失败也可立于不败之地;当不小心伤害他人的时候,道歉不妨超过应有的限度,这样不但可以显示出你的诚意,而且会收到化干戈为玉帛的效果;当要说令人不快的话语时,不妨事先声明,这样就不会引起他人的反感,使他人体会到你的用心良苦。这些运用冷热水效应的举动,实质上就是先通过一二处“伏笔”,使对方心中的“秤砣”变小,如此一来,它“称出的物体重量”也就大了。

某汽车销售公司的老李,每月都能卖出30辆以上汽车,深得公司经理的赏识。由于种种原因,老李预计到这个月只能卖出10辆车。深懂人性奥妙的老李对经理说:“由于银根紧缩,市场萧条,我估计这个月顶多卖出5辆车。”经理点了点头,对他的看法表示赞成。没想到一个月过后,老李竟然卖了12辆汽车,公司经理对他大大夸奖一番。假若老李说本月可以卖15辆或者事先对此不说,结果只卖了12辆,公司经理会怎么认为呢?他会强烈地感受到老李失败了,不但不会夸奖,反而可能指责。在这个事例中,老李把最糟糕情况――顶多卖5辆车,报告给经理,使得经理心中的“秤砣”变小,因此当月绩出来以后,对老李的评价不但不会降低,反而提高了。

蔡女士很少演讲,一次迫不得已,她对一群学者、评论家进行演说。她的开场白是:“我是一个普普通通的家庭妇女,自然不会说出精彩绝伦的话语,因此恳请各位专家对我的发言不要笑话……”经她这么一说,听众心中的“秤砣”变小了,许多开始对她怀疑的人,也在专心听讲了。她的简单朴实演说完成后,台下的学者、评论家们感到好极了,他们认为她的演说达到了极高的水平。对于蔡女士的成功演讲,他们抱以热烈的掌声。

当一个人不能直接端给他人一盆“热水”时,不妨先端给他人一盆“冷水”,再端给他人一盆“温水”,这样的话,这人的这盆“温水”同样会获得他人的一个良好评价。

二、运用冷热水效应去促使对方同意

鲁迅先生说:“如果有人提议在房子墙壁上开个窗口,势必会遭到众人的反对,窗口肯定开不成。可是如果提议把房顶扒掉,众人则会相应退让,同意开个窗口。”鲁迅先生的精辟论述,谈的就是运用冷热水效应去促使对方同意。当提议“把房顶扒掉”时,对方心中的“秤砣”就变小了,对于“墙壁上开个窗口”这个劝说目标,就会顺利答应了。冷热水效应可以用来劝说他人,如果你想让对方接受“一盆温水”,为了不使他拒绝,不妨先让他试试“冷水”的滋味,再将“温水”端上,如此他就会欣然接受了。

某化妆品销售公司的严经理,因工作上的需要,打算让家居市区的推销员小王去近郊区的分公司工作。在找小王谈话时,严经理说:“公司研究,决定你去担任新的重要工作。有两个地方,你任选一个。一个是在远郊区的分公司,一个是在近郊区的分公司。”小王虽然不愿离开已经十分熟悉的市区,但也只好在远郊区和近郊区当中选择一个稍好点的——近郊区。而小王的选择,恰恰与公司的安排不谋而合。而且,严经理并没有多费多少唇舌,小王也认为选择了一项比较理想的工作岗位,双方满意,问题解决。在这个事例中,“远郊区”的出现,缩小了小王心中的“秤砣”,从而使小王顺利地接受去近郊区工作。严经理的这种做法,虽然给人一种玩弄权术的感觉,但如果是从大局考虑,并且对小王本人负责,这种做法也是应该提倡的。

老陈、老时是一家大型化工工厂的谈判高手,这对黄金搭档一出马,几乎没有谈不成的业务,他们深得公司员工的尊重和信赖。原来,他两人十分擅长运用冷热水效应去说服对方。一般的,老陈总是提出苛刻的要求,令对方惊惶失措,灰心丧气,一筹莫展,也就是在心理上把对方压倒了。当对方感到“山穷水尽疑无路”时,老时就出场了,他提出了一个折衷的方案,当然这个方案也就是他们谈判的目标方案。面对这个“柳暗花明又一村”,对方愉快地签订了合同。在这种阵势面前,就是该方案中有一些不利于对方的条件,对方也会认为折衷方案非常好,从而接受。这的确是一种奇妙的谈判技巧,预设的苛刻条件大大缩小了对方心中的“秤砣”,使得对方毫不犹豫地同意那个折衷的方案。这种谈判技巧,在经商洽谈中可以发挥巨大作用。

三、运用冷热水效应去激起对方高兴

一位哲人看见一位生活贫困的朋友整天愁肠百转,一脸苦相,他就想出了一个办法让他快乐起来。他对这位朋友说:“你愿意不愿意离开你的妻子?愿意不愿意丢弃你的孩子?愿意不愿意拆掉你的破房?”朋友一一答“不”。哲人说:“对啊!你应该庆幸你有一位默契的伴侣,庆幸有一个可爱的后代,庆幸有一间温暖的旧屋,你应该为此高兴啊!”于是,这位朋友的愁苦脱离了眉梢,忧郁离开了额头。在这个寓言式故事里,哲人运用冷热水效应,缩小了朋友心中的“秤砣”,从而使他对自己的拮据生活感到快乐。一个人快乐不快乐,通常不是由客观的优劣决定的,而是由自己的心态情绪等决定的。运用冷热水效应,可以使一个人从困难、挫折、不幸中挖掘出新的快乐来。

一次,一架民航客机即将着陆时,机上乘客忽然被通知,由于机场拥挤,无法降落,预计到达时间要推迟1个小时。顿时,机舱里一片抱怨之声,乘客们在等待着这难熬的时间渡过。几分钟后,乘务员宣布,再过30分钟,飞机就会安全降落,乘客们如释重负地松了口气。又过了5分钟,广播里说,现在飞机就要降落了。虽然晚了十几分钟,乘客们却喜出望外,纷纷拍手相庆。在这个事例中,机组人员无意之中运用了冷热水效应,首先使乘客心中的“秤砣”变小,当飞机降落后,对晚点这个事实,乘客们不但不厌烦,反而异常兴奋了。

夏厂长经过慎重考虑,决定给刚刚聘请的技术员小宫1.2 万元的年薪,这个薪金数虽然不高,夏厂长认为小宫会接受下来的,惟一担心的是怕这个问题处理不好,影响他的积极性、创造性。老成持重的夏厂长想出了一个妙法,他对小宫说:“基于咱们厂的实际,只能付给你8000元的年薪。”稍一停顿,夏厂长接着说:“不过1万2千元也可以考虑,你认为如何?”小宫一听 “8000元”,就有点儿不乐意,“秤砣”随之缩小了,当听到“1万2千元”时,心里就有点儿高兴了。他爽快地说:“我听厂长您的。”夏厂长说:“1万2 千元相对于厂里的其他人员来说,已经很高了。实话和你说,我这个做厂长的对此也犹豫不决,不过,只要我们齐心协力,顽强拼搏,就是砸锅卖铁,我也要把1万 2千元钱发到你的手上。”小宫心里感动热乎乎的。在这个事例中,夏厂长运用了冷热水效应,使对方对并不算高的薪金数,不但不灰心丧气,反而心情愉快。

假若首先让对方尝尝“冷水”的滋味,就会使他心中的“秤砣”得以缩小,因此他会对获得的“温水”感到高兴。人际交往中,如果让对方在关键时刻甚或平常日子里高高兴兴,还有什么事办不成,还有什么样的硬仗打不赢呢?

综上所述,冷热水效应在人际交往中,通过使他人心中的“秤砣”变小,发挥着三大作用,但如果使对方心中的“秤砣”变大,就会出现三大负作用了。人与人交往,应力避这些负作用的出现。最后说一句,一个人只有保持心中的“秤砣”合情合理,前后一致,才能正确地评价自身和外在的事物。
\section{零和游戏原理}
\label{sec-62}


当你看到两位对弈者时,你就可以说他们正在玩“零和游戏”。因为在大多数情况下,总会有一个赢,一个输,如果我们把获胜计算为得1分,而输棋为-1分,那么,这两人得分之和就是:1+(-1)=0。

这正是“零和游戏”的基本内容:游戏者有输有赢,一方所赢正是另一方所输,游戏的总成绩永远是零。

零和游戏原理之所以广受关注,主要是因为人们发现在社会的方方面面都能发现与“零和游戏”类似的局面,胜利者的光荣后面往往隐藏着失败者的辛酸和苦涩。从个人到国家,从到经济,似乎无不验证了世界正是一个巨大的“零和游戏”场。这种理论认为,世界是一个封闭的系统,财富、资源、机遇都是有限的,个别人、个别地区和个别国家财富的增加必然意味着对其他人、其他地区和国家的掠夺,这是一个“邪恶进化论”式的弱肉强食的世界。

但20世纪人类在经历了两次世界大战,经济的高速增长、科技进步、全球化以及日益严重的环境污染之后,“零和游戏”观念正逐渐被“双赢”观念所取代。人们开始认识到“利己”不一定要建立在“损人”的基础上。通过有效合作,皆大欢喜的结局是可能出现的。但从“零和游戏”走向“双赢”,要求各方要有真诚合作的精神和勇气,在合作中不要耍小聪明,不要总想占别人的小便宜,要遵守游戏规则,否则“双赢”的局面就不可能出现,最终吃亏的还是自己。
\section{留面子效应}
\label{sec-63}


这正好是与“登门槛技术”和“低球技术”相对应的现象。是指人们拒绝了一个较大要求后,对较小要求的接受程度增加的现象。相应地,为了达到推销的最低回报,先提出一个明知别人会拒绝的较大要求,可以提高顾客接受较小要求的可能性。在日常生活中,售货人的标价和侃价就是对这种技术的应用。
\section{马太效应}
\label{sec-64}


马太效应的名字来自于圣经《新约.马太福音》中的一则寓言。

《圣经》中“马太福音”第二十五章有这么几句话:“凡有的,还要加给他叫他多余;没有的,连他所有的也要夺过来。”

《新约全书》中马太福音第25章的寓言(和合本译文):一个国王远行前,就叫了仆人来,把他的家业交给他们。按着各人的才干,给他们银子。一个给了五千,一个给了二千,一个给了一千。就往外国去了。那领五千的,随即拿去做买卖,另外赚了五千。那领二千的,也照样另赚了二千。但那领一千的,去掘开地,把主人的银子埋藏了。

过了许久,那些仆人的主人来了,和他们算账。那领五千银子的,又带着那另外的五千来,说:“主阿,你交给我五千银子,请看,我又赚了五千。”主人说:好,你这又善良又忠心的仆人。你在不多的事上有忠心,我把许多事派你管理。可以进来享受你主人的快乐。”那领二千的也来说:“主阿,你交给我二千银子,请看,我又赚了二千。”主人说: “好,你这又良善又忠心的仆人。你在不多的事上有忠心,我把许多事派你管理。可以进来享受你主人的快乐。”

那领一千的,也来说:“主阿,我知道你是忍心的人,没有种的地方要收割,没有散的地方要聚敛。我就害怕,去把你的一千银子埋藏在地里。请看,你的原银在这里。”主人回答说:“你这又恶又懒的仆人,你既知道我没有种的地方要收割,没有散的地方要聚敛。就当把我的银子放给兑换银钱的人,到我来的时候,可以连本带利收回。于是夺过他这一千来,给了那有一万的仆人。”

马太效应揭示了一个不断增长个人和企业资源的需求原理,关系到个人的成功和生活幸福,因此它是影响企业发展和个人成功的一个重要法则。

1968年,美国科学史研究者罗伯特.莫顿(Robert K. Merton)首次用“马太效应”来描述这种社会心理现象。“对已有相当声誉的科学家做出的贡献给予的荣誉越来越多,而对于那些还没有出名的科学家则不肯承认他们的成绩。”

社会心理学家认为,“马太效应” 是个既有消极作用又有积极作用的社会心理现象。其消极作用是:名人与未出名者干出同样的成绩,前者往往上级表扬,记者采访,求教者和访问者接踵而至,各种桂冠也一顶接一顶地飘来,结果往往使其中一些人因没有清醒的自我认识和没有理智态度而居功自傲,在人生的道路上跌跟头;而后者则无人问津,甚至还会遭受非难和妒忌。其积极作用是:其一,可以防止社会过早地承认那些还不成熟的成果或过早地接受貌似正确的成果;其二,“马太效应”所产生的“荣誉追加”和“荣誉终身”等现象,对无名者有巨大的吸引力,促使无名者去奋斗,而这种奋斗又必须有明显超越名人过去的成果才能获得向往的荣誉。

“马太效应”在社会中广泛存在。尤其是经济领域内广泛存在的一个现象:强者恒强,弱者恒弱,或者说,赢家通吃。
\section{毛毛虫效应}
\label{sec-65}


毛毛虫习惯于固守原有的本能、习惯、先例和经验,而无法破除尾随习惯而转向去觅食。

法国心理学家约翰.法伯曾经做过一个著名的实验,称之为“毛毛虫实验”:把许多毛毛虫放在一个花盆的边缘上,使其首尾相接,围成一圈,在花盆周围不远的地方,撒了一些毛毛虫喜欢吃的松叶。

毛毛虫开始一个跟着一个,绕着花盆的边缘一圈一圈地走,一小时过去了,一天过去了,又一天过去了,这些毛毛虫还是夜以继日地绕着花盆的边缘在转圈,一连走了七天七夜,它们最终因为饥饿和精疲力竭而相继死去。

约翰.法伯在做这个实验前曾经设想:毛毛虫会很快厌倦这种毫无意义的绕圈而转向它们比较爱吃的食物,遗憾的是毛毛虫并没有这样做。导致这种悲剧的原因就在于毛毛虫习惯于固守原有的本能、习惯、先例和经验。毛毛虫付出了生命,但没有任何成果。其实,如果有一个毛毛虫能够破除尾随的习惯而转向去觅食,就完全可以避免悲剧的发生。

后来,科学家把这种喜欢跟着前面的路线走的习惯称之为“跟随者”的习惯,把因跟随而导致失败的现象称为“毛毛虫效应”。

我们甚至可以说,我们人类也难逃这种效应的影响。比如说,在进行工作、学习和日常生活的过程中,对于那些“轻车熟路”的问题,会下意识地重复一些现成的思考过程和行为方式,因此很容易产生思想上的惯性,也就是不由自主地依靠既有的经验,按固定思路去考虑问题,不愿意转个方向、换个角度想问题。

固有的思路和方法具有相对的成熟性和稳定性,有积极的一面。是因为袭用前人的思路和方法,有助于人们进行类比思维,可以缩短和简化解决的过程,更加顺利和便捷地解决某些问题;

但与此同时,它的消极影响也不容忽视,那就是容易使人们盲目运用特定经验和习惯的方法,对待一些貌似而神异的问题,结果浪费时间与精力,妨碍问题的解决。而且经年累月地按照一种既定的模式思考问题,不仅容易使人厌倦,更容易麻痹人的创造能力,影响潜能的发挥。

毛毛虫那种毫无意义的绕圈所导致的悲剧还说明:在实际工作中“一分耕耘,一分收获”的神话并不存在,我们不能只注意自己做了多少工作,而且还要关注这些工作带来多少成果,也就是人们常说的绩效。如果沿着一个错误的方向,老是跟在别人后面走,可能会付出很多无谓的努力,只有找到一个新的方向和思路,才能有更多的收获
\section{免疫效应}
\label{sec-66}


当学习的材料发生了显著的遗忘后再进行复习时,学习者因发现了遗忘的内容,故能激起复习的动机,他不再把复习看成是多余的事,就在复习中加强了努力和注意;在这们的复习中,学习者还能发现造成遗忘的原因,如新获得的知识模糊不清,未充分分化,不稳固等,于是就在复习时想方设法加强薄弱的部分。因此,把它称为遗忘的免疫效应,这种效应可以解释为什么早晚复习的效果无明显差异的现象。
\section{名片效应}
\label{sec-67}


两个人在交往时,如果首先表明自己与对方的态度和价值观相同,就会使对方感觉到你与他有更多的相似性,从而很快地缩小与你的心理距离,更愿同你接近,结成良好的人际关系。在这里,有意识、有目的地向对方所表明的态度和观点如同名片一样把你介绍给对方。

名片效应指的是要让对方接受你的观点、态度、你就要把对方与自己视为一体,首先向交际对方传播一些他们所能接受的和熟悉并喜欢的观点或思想,然后再悄悄地将自己的观点和思想渗透和组织进去,使对方产生一种印象,似乎我们的思想观点与他们已认可的思想观点是相近的。表明自己与对方的态度和价值观相同,就会使对方感觉到你与他有更多的相似性,从而很快地缩小与你的心理距离,更愿同你接近,结成良好的人际关系。具体的操作方式是,在交际中先向对方传播一些他们所能接受的和熟悉并喜欢的观点或思想,然后再悄悄地将自己的观点和思想渗透和组织进去,使对方产生一种印象,似乎我们的思想观点与他们已认可的思想观点是相近的。其要点在于:

首先,要善于捕捉对方的信息,把握真实的态度,寻找其积极的、你可以接受的观点,形成一张有效的名片。

其次,寻找时机,恰到好处地向对方出示自己根据“名片”打造出的形象,这样,你就可以达到目标。

有一位求职青年,应聘几家单位都被拒之门外,感到十分沮丧。最后,他又抱着一线希望到一家公司应聘,在此之前,他先打听该公司老总的历史,通过了解,他发现这个公司老总以前也有与自己相似的经历,于是他如获珍宝,在应聘时,他就与老总畅谈自己的求职经历,以及自己怀才不遇的愤慨,果然,这一席话博得了老总的赏识和同情,最终他被录用为业务经理。这就是所谓的名片效应。也即两个人在交往时,如果首先表明自己与对方的态度和价值观相同,就会使对方感觉到你与他有更多的相似性,从而很快地缩小与你的心理距离,更愿同你接近,结成良好的人际关系。在这里,有意识、有目的地向对方所表明的态度和观点如同名片一样把你介绍给对方。
\section{名人效应}
\label{sec-68}


名人的出现所达成的引人注意、强化事物、扩大影响的效应,或人们模仿名人的心理现象统称为名人效应。名人效应已经在生活中的方方面面产生深远影响,比如名人代言广告能够刺激消费,名人出席慈善活动能够带动社会关怀弱者等等。

简单的说名人效应相当于一种品牌效应,它可以带动人群,它的效应可以如同疯狂的追星族那么强大。

美国心理学家曾做过一个有趣的实验,在给大学心理系学生讲课时,向学生介绍说聘请到举世闻名的化学家。然后这位化学家说,他发现了一种新的化学物质,这种物质具有强烈的气味,但对人体无害。在这里只是想测一下大家的嗅觉。接着打开瓶盖,过了一会儿,他要求闻到气味的同学举手,不少同学举了手,其实这只瓶子里只不过是蒸馏水,''化学家''是从外校请来的德语教师。这种由于接受名人的暗示所产生的信服和盲从现象被称为名人效应。名人效应的产生依赖于名人的权威和知名度,名人之所以成为名人,在他们那一领域必然有其过人之处。名人知名度高,为世人所熟悉、喜爱,所以名人更能引起人们的好感、关注、议论和记忆。
\section{墨菲定律}
\label{sec-69}


墨菲定律主要内容是:事情如果有变坏的可能,不管这种可能性有多小,它总会发生。

墨菲定律并不是一种强调人为错误的概率性定律,而是阐述了一种偶然中的必然性,我们再举个例子:

你兜里装着一枚金币,生怕别人知道也生怕丢失,所以你每隔一段时间就会去用手摸兜,去查看金币是不是还在,于是你的规律性动作引起了小偷的注意,最终被小偷偷走了。即便没有被小偷偷走,那个总被你摸来摸去的兜最后终于被磨破了,金币掉了出去丢失了。

这就说明了,越害怕发生的事情就越会发生的原因,为什么?就因为害怕发生,所以会非常在意,注意力越集中,就越容易犯错误。

墨菲定律的原话是这样说的:If there are two or more ways to do something, and one of those ways can result in a catastrophe, then someone will do it.(如果有两种选择,其中一种将导致灾难,则必定有人会作出这种选择。)

根据“墨菲定律”:
\subsection{任何事都没有表面看起来那么简单;}
\label{sec-69-1}
\subsection{所有的事都会比你预计的时间长;}
\label{sec-69-2}
\subsection{会出错的事总会出错;}
\label{sec-69-3}
\subsection{如果你担心某种情况发生,那么它就更有可能发生。}
\label{sec-69-4}


墨菲定律告诉我们,容易犯错误是人类与生俱来的弱点,不论科技多发达,事故都会发生。而且我们解决问题的手段越高明,面临的麻烦就越严重。所以,我们在事前应该是尽可能想得周到、全面一些,如果真的发生不幸或者损失,就笑着应对吧,关键在于总结所犯的错误,而不是企图掩盖它。
\section{摩西奶奶效应}
\label{sec-70}


美国艺术家摩西奶奶,至暮年才发现自己有惊人的艺术天才,75岁开始学画,80岁举行首次个人画展。摩西奶奶效应告诉我们,一个人如果不去挖掘自己的潜在能力,它就会自行泯灭。

正像格拉宁所说:“如果每个人都能知道自己干什么,那么生活会变得多么好!因为每个人的能力都比他自己感觉到的大得多。”
\section{木桶法则}
\label{sec-71}


“木桶”法则的意思是:一只沿口不齐的木桶,它盛水的多少,不在于木桶上那块最长的木板,而在于木桶上最短的那块木板。要想多盛水——提高木桶的整体效应,不是去增加最长的那块木板的长度,而是要下功夫依次补齐木桶上最短的那块木板。

“木桶”法则告诉管理者:在管理过程中要下功夫狠抓薄弱环节,否则,单位的整体工作就会受到影响。人们常说“取长补短”,即取长的目的是为了补短,只取长不补短,就很难提高工作的整体效应。
\section{南风效应}
\label{sec-72}


北风和南风比威力,看谁能把行人身上的大衣脱掉。北风首先来一个冷风凛冽寒冷刺骨,结果行人为了抵御北风的侵袭,便把大衣裹得紧紧的。南风则徐徐吹动,顿时风和日丽,行人因为觉得春暖上身,始而解开纽扣,继而脱掉大衣,南风获得了胜利。

温暖胜于严寒。运用到管理实践中,南风法则要求管理者要尊重和关心下属,时刻以下属为本,多点“人情味”,多注意解决下属日常生活中的实际困难,使下属真正感受到管理者给予的温暖。这样,下属出于感激就会更加努力积极地为企业工作,维护企业利益。

在诸多的日本公司中,松下公司的做法极富典型性。

与其他日本公司一样,松下尊重职工,处处考虑职工利益,还给予职工工作的欢乐和精神上的安定感,与职工同甘共苦。1930年初,世界经济不景气,日本经济大混乱,绝大多数厂家都裁员,降低工资,减产自保,百姓失业严重,生活毫无保障。松下公司也受到了极大伤害,销售额锐减,商品积压如山,资金周转不灵。这时,有的管理人员提出要裁员,缩小业务规模。这时,因病在家休养的松下幸之助并没有这样做,而是毅然决定采取与其他厂家完全不同的做法:工人一个不减,生产实行半日制,工资按全天支付。与此同时,他要求全体员工利用闲暇时间去推销库存商品。松下公司的这一做法获得了全体员工的一致拥护,大家千方百计地推销商品,只用了不到3个月的时间就把积压商品推销一空,使松下公司顺利渡过了难关。在松下的经营史上,曾有几次危机,但松下幸之助在困难中依然坚守信念,不忘民众的经营思想,使公司的凝聚力和抵御困难的能力大大增强,每次危机都在全体员工的奋力拼搏、共同努力下安全度过,松下幸之助也赢得了员工们的一致称颂。

松下以员工为企业之本的做法在获得了员工们大力欢迎的同时,也为松下公司培养起了一个无坚不摧的团队。二战结束以后的很长一段时间内,松下公司都十分困难。而在这种情况下,占领军出台了要惩罚为战争出过力的财阀的政令,松下幸之助也被列入了受打击的财阀名单。眼看松下就要被消灭了,这时,意想不到的局面出现了:松下电器公司的工会以及代理店联合组织起来,掀起了解除松下财阀指定的请愿活动,参加人数多达几万。在当时的日本,许多被指定为财阀的企业基本上都是被工会接管和占领了。工会起来维护企业的事还是头一遭。面对游行队伍,占领军当局不得不重新考虑对松下的处理。到第二年五月,占领当局解除了对松下财阀的指定,从而使松下摆脱了一场厄运。正是因为松下幸之助始终贯彻以人为本,尊重职工,爱护职工的企业经营理念,才保证了自己的绝处逢生。

古语云:得人心者得天下!只有真正俘获了员工的心灵,员工才会为企业的发展死心塌地地工作。在企业管理中多点人情味,少些铜臭味,有助于培养员工对企业的认同感和忠诚度。有了这些,企业在竞争中就能无往而不胜。
\section{鲶鱼效应}
\label{sec-73}


挪威人喜欢吃沙丁鱼,尤其是活鱼。市场上活沙丁鱼的价格要比死鱼高许多。所以渔民总是千方百计地想办法让沙丁鱼活着回到渔港。可是虽然经过种种努力,绝大部分沙丁鱼还是在中途因窒息而死亡。但却有一条渔船总能让大部分沙丁鱼活着回到渔港。船长严格保守着秘密。直到船长去世,谜底才揭开。原来是船长在装满沙丁鱼的鱼槽里放进了一条以鱼为主要食物的鲶鱼。鲶鱼进入鱼槽后,由于环境陌生,便四处游动。沙丁鱼见了鲶鱼十分紧张,左冲右突,四处躲避,加速游动。这样一来,一条条沙丁鱼欢蹦乱跳地回到了渔港。这就是著名的“鲶鱼效应”。

鲶鱼效应对于“渔夫”来说,在于激励手段的应用。渔夫采用鲶鱼来作为激励手段,促使沙丁鱼不断游动,以保证沙丁鱼活着,以此来获得最大利益。在企业管理中,管理者要实现管理的目标,同样需要引入鲶鱼型人才,以此来改变企业相对一潭死水的状况。

鲶鱼效应对于“鲶鱼”来说,在于自我实现。鲶鱼型人才是企业管理必需的。鲶鱼型人才是出于获得生存空间的需要出现的,而并非是一开始就有如此的良好动机。对于鲶鱼型人才来说,自我实现始终是最根本的。

鲶鱼效应对于“沙丁鱼”来说,在于缺乏忧患意识。沙丁鱼型员工的忧患意识太少,一味地想追求稳定,但现实的生存状况是不允许沙丁鱼有片刻的安宁。“沙丁鱼”如果不想窒息而亡,就应该也必须活跃起来,积极寻找新的出路。以上四个方面都是探讨鲶鱼效应时必须考虑的问题。

鲶鱼效应的根本就是一个管理方法的问题,而应用鲶鱼效应的关键就在于如何应用好鲶鱼型人才。如何对鲶鱼型人才或组织进行有效的利用和管理是管理者必须探讨的问题。由于鲶鱼型人才的特殊性,管理者不可能用相同的方式来管理鲶鱼型人才,已有的管理方式可能有相当部分已经过时。因此,鲶鱼效应对管理者提出了新的要求,不仅要求管理者掌握管理的常识,而且还要求管理者在自身素质和修养方面有一番作为,这样才能够让鲶鱼型人才心服口服,才能够保证组织目标得以实现。因此,企业管理在强调科学化的同时,应更加人性化,以保证管理目标的实现。

鲶鱼型人才在组织中如何安身立命也是一个必须着重说明的问题。历史上有很多“好动”的人才最后都没有落得好下场,原因就在于他们的“好动 ”,而且往往得罪了很多人后,这些人又联合起来将他打压了下去。虽然组织因为这些“好动”的人而得到了长足的发展,但是这些“好动”的人的下场也让很多人想动不敢动。其实鲶鱼型人才在组织中的生存是有规律可寻的。鲶鱼型人才固然要做得最好,但也要学会低调和韬光养晦;鲶鱼型人才固然要忠诚于组织,但也要学会功成身退,毕竟任何忠诚都是有限度的;鲶鱼型人才固然要努力工作,但也要讲究做人做事的方法,或者也可以称作手段。对于鲶鱼型人才来说,最重要的固然是自我价值的实现,但最根本的却是如何求得自身的安全。
\section{鸟笼效应}
\label{sec-74}


鸟笼效应是一个著名的心理现象,其发现者是近代杰出的心理学家詹姆斯。1907年,詹姆斯从哈佛大学退休,同时退休的还有他的好友物理学家卡尔森。一天,两人打赌。詹姆斯说:“我一定会让你不久就养上一只鸟的。”卡尔森不以为然:“我不信!因为我从来就没有想过要养一只鸟。”没过几天,恰逢卡尔森生日,詹姆斯送上了礼物——一只精致的鸟笼。卡尔森笑了:“我只当它是一件漂亮的工艺品。你就别费劲了。”从此以后,只要客人来访,看见书桌旁那只空荡荡的鸟笼,他们几乎都会无一例外地问:“教授,你养的鸟什么时候死了?”卡尔森只好一次次地向客人解释:“我从来就没有养过鸟。”然而,这种回答每每换来的却是客人困惑而有些不信任的目光。无奈之下,卡尔森教授只好买了一只鸟,詹姆斯的“鸟笼效应”奏效了。实际上,在我们的身边,包括我们自己,很多时候不是先在自己的心里挂上一只笼子,然后再不由自主地朝其中填满一些什么东西吗?

“鸟笼效应”是一个很有意思的规律,它说的是:如果一个人买了一个空的鸟笼放在自己家的客厅里,过了一段时间,他一般会丢掉这个鸟笼或者买一只鸟回来养。原因是这样的:即使这个主人长期对着空鸟笼并不别扭,每次来访的客人都会很惊讶地问他这个空鸟笼是怎么回事情,或者把怪异的目光投向空鸟笼,每次如此。终于他不愿意忍受每次都要进行解释的麻烦,丢掉鸟笼或者买只鸟回来相配。经济学家解释说,这是因为买一只鸟比解释为什么有一只空鸟笼要简便得多。即使没有人来问,或者不需要加以解释,“鸟笼效应”也会造成人的一种心理上的压力,使其主动去买来一只鸟与笼子相配套。

同样“鸟笼效应”也被称为“空花瓶效应”,我听说过一个故事,一个女孩子的男朋友送了她一束花,她很高兴,特意让妈妈从从家里带来一只水晶花瓶,结果为了不让这个花瓶空着,她的男朋友就必须隔几天就送花给她。当然这是此效应的一种甜蜜的体现。

俗话说:邻居处的好,相当捡个宝.

张长和刘青是邻居,张长由于工作的缘故要迁居,家里的东西大部分都要卖出去,在清理完所有的家具后,只剩下了一个雅致的书桌了,这书桌价格昂贵,如果作为次品卖出也收回不了多少钱,于是,张长决定把它送给邻居刘青作为纪念礼物,刘青也欣悦的接受了并对此表示了感谢.

书桌搬回自家书屋后,刘青发现书屋那破旧的木藤椅与书桌比起来真是大煞风景,于是刘青决定买一个好的皮质的转椅来搭配书桌,于是乎花了250元买了一个合适的转椅,心里觉得舒服了许多\ldots{}...

一天,朋友来刘青家做客,刘青为展示自家的新书房,请朋友进来看看,朋友对书桌和转椅赞不绝口,朋友说:''不错,不错,只是能把书橱换一下就更好了.''刘青看了看,觉得书橱有些破旧了,确实也应该换一下了,于是乎又花钱换了书橱\ldots{}...

这天,又有一些朋友光顾了刘青家,照样来到了书房,还是同样夸赞了一番,但总是最后带有瑕疵:''你的书房什么都好,就是光线暗了些,要是能把墙打开,建一个落地窗就更加明亮了.''听后,刘青觉得也是,于是乎\ldots{}...

为了一个书房,把刘青折腾的不轻,而这一切的一切又都是由于那张书桌,何必呢?!
\section{牛鞭效应,}
\label{sec-75}


营销过程中的需求变异放大现象被通俗地称为“牛鞭效应”。 (指供应链上的信息流从最终客户向原始供应商端传递时候,由于无法有效地实现信息的共享,
使得信息扭曲而逐渐放大,导致了需求信息出现越来越大的波动。)

“牛鞭效应”是市场营销中普遍存在的高风险现象,是销售商与供应商在需求预测修正、订货批量决策、价格波动、短缺博弈、库存责任失衡和应付环境变异
等方面博弈的结果,增大了供应商的生产、供应、库存管理和市场营销的不稳定性。企业可以从6个方面规避或化解需求放大变异的影响:即订货分级管理;
加强入库管理,合理分担库存责任;缩短提前期,实行外包服务;规避短缺情况下的博弈行为;参考历史资料,适当减量修正,分批发送;提前回款期限。

“牛鞭效应”是市场营销活动中普遍存在的高风险现象,它直接加重了供应商的供应和库存风险,甚至扰乱生产商的计划安排与营销管理秩序,导致生产、供
应、营销的混乱,解决“牛鞭效应”难题是企业正常的营销管理和良好的顾客服务的必要前提。
\section{拍球效应}
\label{sec-76}


爱好篮球的人都知道,拍篮球时,用的力越大,篮球就跳的越高。这就是“拍球效应”。拍球效应的寓意就是:承受的压力越大,人的潜能发挥程度越高,反之,人的压力较轻,潜能发挥程度就较小。

有一位经验丰富的老船长,当他的货轮卸货后在浩瀚的大海上返航时,突然遭遇到了可怕的风暴。水手们惊慌失措,老船长果断地命令水手们立刻打开货舱,往里面灌水。“船长是不是疯了,往船舱里灌水只会增加船的压力,使船下沉,这不是自寻死路吗?”一个年轻的水手嘟囔。

看着船长严厉的脸色,水手们还是照做了。随着货舱里的水位越升越高,随着船一寸一寸地下沉,依旧猛烈的狂风巨浪对船的威胁却一点一点地减少,货轮渐渐平稳了。

船长望着松了一口气的水手们说:“百万吨的巨轮很少有被打翻的,被打翻的常常是根基轻的小船。船在负重的时候,是最安全的;空船时,则是最危险的。当然这种负重是要根据船的承载能力界定的,适当的压力可以抵挡暴风骤雨的侵袭,但如果是船不能承受之重,它就会如你们担心的那样,消失在海面。”老船长就是运用了压力效应,才使得人船俱存。那些得过且过,没有一点压力,像风暴中没有载货的船,往往一场人生的狂风巨浪便会把他们打翻。而那些负荷过重的人,却不是被风浪击倒,而是自己沉寂于忙碌的生活。

压力伴随着人的一生,谁都不可能避免。它就像呼吸一样永远存在,只有呼吸停止了,压力才消失。有压力才有动力,人要是活在一个没有压力的环境下,就容易颓废,就很难有进步,如同水没有落差就不会流一样。工农大众感受更多的是身体的疲劳和生存的压力,知识分子感受更多的是精神的创伤和发展的压力。很多研究发现,适度的压力有利于我们保持良好的状态,更加有助于挖掘我们的潜能,从而提高个人与社会的整体效率。心理学家对心理压力、工作难度和作业成绩三者关系有这样的解释:在简单易为的工作情境下,较高的心理压力将产生较佳的成绩;在复杂困难的工作情境下,较低的心理压力将产生较高的成绩。比如运动员每到参加比赛,一定要将自己调整到感到适度的压力,让自己兴奋,进入最佳的竞技状态,如果他不紧张、没压力感,则不利于出成绩。再如考试时,适度的压力能调动我们的大脑,让我们兴奋,考出好的成绩。所以,适度的压力对于促进社会发展、挖掘内在潜力资源,是有正面意义的。
\section{旁观者效应}
\label{sec-77}


在紧急事件中由于有他人在场而产生的对救助行为的抑制作用。旁观者人数越多,抑制程度越高。产生原因主要有:由于众人在场,社会责任被分散;个人不能确定该怎么做,想看看在场其他人怎么做,而其他人也有类似想法,等等。

危机现场中人数愈多,救助行为出现的可能反而愈少。此现象为旁观者效应。

利他行为受到许多环境因素的影响,其中一个影响因素便是“旁观者效应”。旁观者效应指的是,个体对于紧急事态的反应,在单个人时与同其他人在一起时是不同的。由于他人在场个体会抑制利他行为。

1964年3月,在纽约的克尤公园发生了一起震惊全美的谋杀案。一位年轻的酒吧女经理,在凌晨3点回家的途中,被一不相识的男性杀人狂杀死。这名男子作案时间长达半个小时,当时,住在公园附近公寓里的住户中有38人看到或听到女经理被刺的情形和反复的呼救声,但没有一个人下来保护她,也没有一个人及时打电话给警察。事后,美国大小媒体同声谴责纽约人的异化与冷漠。
\section{配套效应,狄德罗效应}
\label{sec-78}


狄德罗效应是一种常见的“愈得愈不足效应”,在没有得到某种东西时,心里很平稳,而一旦得到了,却不满足。

18世纪法国有个哲学家叫丹尼斯.狄德罗。有一天,朋友送他一件质地精良、做工考究的睡袍,狄德罗非常喜欢。可他穿着华贵的睡袍在书房走来走去时,总觉得家具不是破旧不堪,就是风格不对,地毯的针脚也粗得吓人。于是,为了与睡袍配套,旧的东西先后更新,书房终于跟上了睡袍的档次,可他却觉得很不舒服,因为“自己居然被一件睡袍胁迫了”,就把这种感觉写成一篇文章叫《与旧睡袍别离之后的烦恼》。200年后,美国哈佛大学经济学家朱丽叶.施罗尔在《过度消费的美国人》一书中,提出了一个新概念——“狄德罗效应”,或“配套效应”,专指人们在拥有了一件新的物品后,不断配置与其相适应的物品,以达到心理上平衡的现象。

先说生活中的“狄德罗效应”吧,人们分到或买到一套新住宅,为了配套,总是要大肆装修一番,铺上大理石或木地板后,自然要以黑白木封墙再安装像样的灯池;四壁豪华后自然还要配红木等硬木家具;出入这样的住宅,显然不能再破衣烂衫,必定要“拿得出手”的衣服与鞋袜;就此“狄德罗”下去,有的也就觉得男主人或女主人不够配套,遂走上了离妻换夫的路子。

狄德罗效应给人们一种启示:对于那些非必需的东西尽量不要。因为如果你接受了一件,那么外界的和心理的压力会使你不断地接受更多非必需的东西。
\section{破窗效应}
\label{sec-79}


政治学家威尔逊和犯罪学家凯琳提出了一个“破窗效应”理论,认为:如果有人打坏了一幢建筑物的窗户玻璃,而这扇窗户又得不到及时的维修,别人就可能受到某些暗示性的纵容去打烂更多的窗户。久而久之,这些破窗户就给人造成一种无序的感觉。结果在这种公众麻木不仁的氛围中,犯罪就会滋生、繁荣。

纽约市交通警察局长布拉顿受到了“破窗理论”的启发。纽约的地铁被认为是“可以为所欲为、无法无天的场所”,针对纽约地铁犯罪率的飙升,布拉顿采取的措施是号召所有的交警认真推进有关“生活质量”的法律,他以“破窗理论”为师,虽然地铁站的重大刑案不断增加,他却全力打击逃票。结果发现,每七名逃票者中,就有一名是通缉犯;每二十名逃票者中,就有一名携带凶器。结果,从抓逃票开始,地铁站的犯罪率竟然下降,治安大幅好转。他的做法显示出,小奸小恶正是暴力犯罪的温床。因为针对这些看似微小、却有象征意义的违章行为大力整顿,却大大减少了刑事犯罪。

在日本,有一种称做“红牌作战”的质量管理活动。日本的企业将有油污、不清洁的设备贴上具有警示意义的“红牌”,将藏污纳垢的办公室和车间死角也贴“红牌”,以促其迅速改观,从而使工作场所清洁整齐,营造出一个舒爽有序的工作氛围。在这样一种积极暗示下,久而久之,人人都遵守规则,认真工作。实践证明,这种工作现象的整洁对于保障企业的产品质量起到了非常重要的作用。

从“破窗效应”中,我们可以得到这样一个道理:任何一种不良现象的存在,都在传递着一种信息,这种信息会导致不良现象的无限扩展,同时必须高度警觉那些看起来是偶然的、个别的、轻微的“过错”,如果对这种行为不闻不问、熟视无睹、反应迟钝或纠正不力,就会纵容更多的人“去打烂更多的窗户玻璃”,就极有可能演变成“千里之堤,溃于蚁穴”的恶果。
\section{瀑布心理效应}
\label{sec-80}


某人一句随便说出的话,却弄得别人十分“不得意”,有点“一石激起千层浪”的意味。这种现象在心理学上,被称之为“瀑布心理效应”,即信息发出者的心理比较平静,但传出的信息被接受后却引起了不平静的心理,从而导致态度行为的变化等,这种心理效应现象,正象大自然中的瀑布一样,上面平平静静,下面却溅花腾雾。

中国有句古话叫做“说者无心,听者有意”,你明明只是无心地说了一句话,却“有意”地伤害到了别人。轻则引起对方的反感,重则给自己引来灾祸。因此,当你在和陌生人打交道时,就需要谨言慎行,注意自己说话的分寸。
\section{齐加尼克效应}
\label{sec-81}


法国心理学家齐加尼克曾经做过一个实验:将一批学生分成两组,让他们同时完成20项工作。结果一组顺利完成了任务,而另一组却未完成。试验表明,虽然受训者在接受任务时均呈现出一种紧张状态,但顺利完成任务者,其紧张情绪逐渐消失,而未完成任务者,紧张情绪却持续存在,且呈加剧倾向。后一种现象被称为''齐加尼克''效应。

随着当代科学技术的飞速发展和知识信息量的增加,作为“白领”阶层的脑力劳动者,其工作节奏日趋紧张,心理负荷亦日益加重。特别是脑力劳动是以大脑的积极思维为主的活动,一般不受时间和空间的限制,是持续而不间断的活动,所以紧张也往往是持续存在的。

有些压力是良性的,它让我们振作。但更多的来自于我们感到自己无力控制的事物的压力,则往往导致齐加尼克效应,使我们更疲劳。
\section{青蛙效应}
\label{sec-82}


“青蛙效应”源自十九世纪末,美国康奈尔大学曾进行过一次著名的“青蛙试验”:他们将一只青蛙放在煮沸的大锅里,青蛙触电般地立即窜了出去。后来,人们又把它放在一个装满凉水的大锅里,任其自由游动。然后用小火慢慢加热,青蛙虽然可以感觉到外界温度的变化,却因惰性而没有立即往外跳,直到到后来热度难忍而失去逃生能力而被煮熟。科学家经过分析认为,这只青蛙第一次之所以能“逃离险境”,是因为它受到了沸水的剧烈刺激,于是便使出全部的力量跳了出来,第二次由于没有明显感觉到刺激,因此,这只青蛙便失去了警惕,没有了危机意识,它觉得这一温度正适合,然而当它感觉到危机时,已经没有能力从水里逃出来了。

“青蛙效应”告诉人们,企业竞争环境的改变大多是渐热式的,如果管理者与员工对环境之变化没有疼痛的感觉,最后就会像这只青蛙一样,被煮熟、淘汰了仍不知道。一个企业不要满足于眼前的既得利益,不要沉湎于过去的胜利和美好愿望之中,而忘掉危机的逐渐形成和看不到失败一步步地逼近,最后像青蛙一般在安逸中死去。而一个人或企业应居安思危,适时宣扬危机,适度加压,使处危境而不知危境的人猛醒,使放慢脚步的人加快脚步,不断超越自己,超越过去。
\section{情绪效应}
\label{sec-83}


所谓情绪效应是指一个人的情绪状态可以影响到对某一个人今后的评价。尤其是在第一印象形成过程中,主体的情绪状态更具有十分重要的作用,第一次接触时主体的喜怒哀乐对于对方关系的建立或是对于对方的评价,可以产生不可思议的差异。与此同时,交往双方可以产生“情绪传染”的心理效果。主体情绪不正常,也可以引起对方不良态度的反映,就影响良好人际关系的建立。因此,管理者在对被管理者做思想政治工作时,一定要注意到被管理者的情绪,双方在平等和睦的气氛中交淡,这样才能收到良好的管理效果。

一天早晨,有一位智者看到死神向一座城市走去,于是上前问道:“你要去做什么?”

死神回答说:“我要到前方那个城市里去带走100个人。”

那个智者说:“这太可怕了!”

死神说:“但这就是我的工作,我必须这么做。”

这个智者告别死神,并抢在它前面跑到那座城市里,提醒所遇到的每一个人:请大家小心,死神即将来带走100个人。

第二天早上,他在城外又遇到到了死神,带着不满的口气问道:“昨天你告诉我你要从这儿带走100个人,可是为什么有1000个人死了?”

死神看了看智者,平静地回答说:“我从来不超量工作,而且也确实准备按昨天告诉你的那样做了,只带走100个人。可是恐惧和焦虑带走了其他那些人。”

恐惧和焦虑可以起到和死神一样的作用,这就是情绪效应。实际上,在我们的生活中,这样的效应每天都在发生,只不过我们已经习以为常。
\section{权威效应}
\label{sec-84}


权威效应,又称为权威暗示效应,是指一个人要是地位高,有威信,受人敬重,那他所说的话及所做的事就容易引起别人重视,并让他们相信其正确性,即“人微言轻、人贵言重”。

“权威效应”的普遍存在,首先是由于人们有“安全心理”,即人们总认为权威人物往往是正确的楷模,服从他们会使自己具备安全感,增加不会出错的“保险系数”;其次是由于人们有“赞许心理”,即人们总认为权威人物的要求往往和社会规范相一致,按照权威人物的要求去做,会得到各方面的赞许和奖励。

在企业中,领导也可利用“权威效应”去引导和改变下属的工作态度以及行为,这往往比命令的效果更好。因此,一个优秀的领导肯定是企业的权威,或者为企业培养了一个权威,然后利用权威暗示效应进行领导。当然,要树立权威就必须要先对权威有一个全面深层的理解,这样才能正确地树立权威,才能让权威保持得更加长久。
\section{热炉法则}
\label{sec-85}


每个单位都有规章制度,单位中的任何人触犯规章制度都要受到惩处。“热炉”法则形象地阐述了惩处原则:

1、热炉火红,不用手去摸也知道炉子是热的,是会灼伤人的——警告性原则。领导者要经常对下属进行规章制度教育,以警告或劝戒不要触犯规章制度,否则会受到惩处。

2、每当你碰到热炉,肯定会被火灼伤—一致性原则。“说”和“做”是一致的,说到就会做到。也就是说只要触犯单位的规章制度,就一定会受到惩处。

3、当你碰到热炉时,立即就会被灼伤——即时性原则。惩处必须在错误行为发生后立即进行,决不能拖泥带水,决不能有时间差,以便达到使犯错人及时改正错误行为的目的.

4、不管是谁碰到热炉,都会被灼伤——公平性原则。不论是管理者还是下属,只要触犯单位的规章制度,都要受到惩处。在单位规章制度面前人人平等。
\section{锐化效应}
\label{sec-86}


在社会知觉中,波斯托曼作了有趣的实验,事先对人们所重视的价值作了调查,接着把与这种价值有关的单词在银幕上用瞬时显示器进行提示。当测定各人的认知阈限时发现,以前认为价值越大的单词,认知阈限就越低。也就是说,人的价值观对其知觉是有促进作用的。像这种由主体方面的内在条件而促进知觉的作用,就叫做知觉的锐化效应。

如何使远景和价值观发挥作用
\subsection{建立基础}
\label{sec-86-1}


首先,组织必须界定所有利益人并确立优先顺序。通常最主要的利益人有客户、员工和股东。由于他们之间常常起利益冲突,所以如果未能确定利益人的优先顺序,组织内部就会充斥纷争。调查发现,第一利益人是股东(占50\%),其次是客户(占33\%)。而由于市场竞争变得日益激烈,而资金的获取变得相对容易,这种关系正朝着有利于客户的方向变化。一个比较有用的公式是:忠实的客户+工作积极的员工=满意的利益人。
\subsection{远大的远景}
\label{sec-86-2}


远景描绘的是组织的长期目标,也就是未来十年或二十年后的状况。它应该能够向员工提出挑战,发动和激励员工努力工作,确定个人的发展方向。因为组织的成功依赖于了解和满足客户的未来需求,所以远景必须以客户为中心。被调查的对象中,只有34\%确立了远大的远景,非赢利性组织比赢利性组织做得好,百分比达到63\%。
\subsection{强有力的价值观}
\label{sec-86-3}


价值观必须是切中要害的、可衡量的。如果做到这一点,价值观将极大地提升公司的竞争力。否则,只会浪费大量的时间。

所谓''切中要害'',是指价值观必须与未来成功的关键要素联系起来。如果未来成功的关键要素为上乘的质量、出色的服务和创新,那么整个组织的价值观必须能够反映他们。在被调查的组织之中,只有大约一半左右实现了这种联系。常见的情形是,价值观软弱无力,与未来成功的关键要素毫不相干。

当价值观被融入到实际工作中去时,他们变得实实在在,可以衡量。员工将清楚地看到在组织的价值观下,他们应该怎样行动。
\subsection{沟通}
\label{sec-86-4}


对于客户与员工来说,行动远比宣言重要。宣言仅仅说出了衡量行动的标准。客户和员工会经常对照价值观评判管理层的行为。人们在读了宣言后,会环顾四周,打量一番。正如一位副总裁所说的:''仍然有些人,包括我们首席执行官在内,无法接受将一种价值观作为自己的行为准绳……,要沟通价值观,最重要的就是行为,而这恰恰是我们的问题所在。''
\subsection{生根发芽}
\label{sec-86-5}


远景和价值观必须植入组织的系统框架才能产生最佳行为,并且不依赖任何个人而经久不衰。新进员工也许还不具备必要的技能,但他们将被组织的远景和价值观吸引。培训将体现远景和价值观,绩效管理基于工作成绩--业绩,和取得工作成绩的方式--价值观。报酬(工资、奖金、晋升)将激励他们追求组织的远景和价值观。
\subsection{品牌效应}
\label{sec-86-6}


组织品牌向所有群体传达组织的远景和价值观。它就像一面旗帜,将员工凝聚在一起,向客户传递公司的实质。它涵盖组织发出的信息的所有要素--人员、行为、产品、服务或业绩。从这个意义上讲,品牌的涵义比单纯的广告要广,而且不仅仅针对客户。
\subsection{考评}
\label{sec-86-7}


应该严格评估远景和价值观的执行情况:远景化为目标和战略,价值观化为可以测量的行为。对于企业和大型非赢利性组织来说,客户满意度调查和员工满意度调查是非常重要的手段。其它指标包括员工流动情况、安全状况、业绩成果,以及高层直接观察。考评工作通常由不同部门在不同时间使用不同工具进行。为使考评工作更为有效,需要作跨部门整合,定期在董事会上讨论研究,并将考评结果作为认可、嘉奖和进一步发展的依据。调查表明,目前很少有组织能将这件事做得很好。
\subsection{社会惰化效应}
\label{sec-86-8}


所谓社会惰化,是指个人与群体其他成员一起完成某种事情时,或个人活动时有他人在场,往往个人所付出的努力比单独时偏少,不如单干时出力多,个人的活动积极性与效率下降的现象,也称之为社会惰化作用,另也叫社会干扰、社会致弱、社会逍遥、社会懈怠。

富有战斗力的团队精神,也是任何集体都梦寐以求的。企业拥有了团队精神就是拥有了核心竞争力,就具备了在现代市场竞争中无往而不胜的战略优势,就可以弥补诸如资金、技术等方面的不足。这是因为团队可以产生正的效应,也就是说团队的产出比成员单个工作的产出之和大,因为团队精神可以刺激个人的努力,因此2+ 2可以等于5。但事实上,团队产生的效应常常是负的。为什么会有这种结果呢?原因在于社会惰化效应。

也许是因为团队成员都认为其他人没有公平付出。假想你认为当你在辛苦工作时,别人却在偷懒,那么你肯定也会减少工作量来重建公平感;另一种原因是责任的分散。所谓法不责众,因为团队的成绩不会归功于个人,个人的投入和团队产出之间的关系不明朗。这样有的个体可能成为“搭便车者”,依附团队的努力。换句话说,如果个体认为自己的贡献无法被衡量,效率就会下降。
\section{狮羊效应}
\label{sec-87}


“狮.羊效应源于拿破仑的一句家家喻户晓的名言:一只狮子带领的九十九只绵羊可以打败一只绵羊带领的九十九只狮子。这句名言说明了主帅的重要性。
\section{视网膜效应}
\label{sec-88}


简单地说,这种效应的意思就是当我们自己拥有一件东西或一项特征时,我们就会比平常人更会注意到别人是否跟我们一样具备这种特征。

记得4年前我刚回国时,首先想到要买的就是一部车。经过一段时间的评估后,我决定买一部墨绿色的中型轿车。当时我的印象是一般人的车都买白色或黑色,所以认为自己的选择很独特,而且又很有品位。正在为自己能买到一部与众不同的车而沾沾自喜时,我突然发现不论是在高速公路上、小巷子里,甚至于我住的大楼停车场中,都看到许多与我同型,而且是墨绿色的轿车。

我开始觉得很奇怪,为什么大家突然间都开始买墨绿色的车?所以我就把我的观察与同事们分享。有一位女同事当时正好怀孕,听我讲完后就说:“我倒是没有看到很多墨绿色的车。可是最近我发现,无论在哪里都会看到孕妇。我记得上个星期天在逛百货公司时,短短两小时就看到6个孕妇,台湾的人口出生率最近是不是有提高呢?”我与其他同事异口同声地都说没发现孕妇有增加的现象,她看到那么多大概是很凑巧。

卡耐基先生很久以前就提出一个论点,那就是每个人的特质中大约有80%是长处或优点,而20%左右是我们的缺点。当一个人只知道自己的缺点是什么,而不知发掘优点时,“视网膜效应”就会促使这个人发现他身边也有许多人拥有类似的缺点,进而使他人际关系无法改善,生活也不会快乐。

所以卡耐基先生在80年前创办卡耐基训练时,就一直强调一个人要人缘好、要受人欢迎,一定要培养欣赏自己与肯定自己的能力。因为在“视网膜效应”的运作下,一个看到自己优点的人,才有能力看到他人可取之处。而能用积极的态度看待他人,往往是搞好人际关系的必备条件。
\section{手表效应}
\label{sec-89}


“手表效应”。大家都有这种体会:一个人如果只有一只手表,他知道现在几点了;如果有两只手表,他往往不知道现在几点了,也就是说,他无法知道哪一只手表更为精确,于是他也就无法确定精确的时间。这就是“手表效应”的原义。

同时拥有两只表时却无法确定时间,反而会让看表的人失去对准确时间的信心。你要做的就是选择其中较信赖的一只,尽力校准它,并以此作为你的标准,听从它的指引行事。记住尼采的话:''兄弟,如果你是幸运的,你只需有一种道德而不要贪多,这样,你过桥更容易些。''

不能同时采用两种不同的管理方法,不能同时设置两个不同的目标,否则将使这个企业无所适从。同样的,一个人不能同时由两个以上的人来指挥,否则这个人会无所适从。但是,大象配置了两个看似同一级别的领导人,正是违反了这样的原则,导致公司发生了种种的矛盾和冲突。

如果每个人都''选择你所爱,爱你所选择'',无论成败都可以心安理得。然而,困扰很多人的是:他们被''两只表''弄得无所,心身交瘁,不知自己该信仰哪一个,还有人在环境、他人的压力下,违心选择了自己并不喜欢的道路,为此而郁郁终生,即使取得了受人瞩目的成就,也体会不到成功的快乐。
\section{睡眠效应}
\label{sec-90}


传播结束一段时间后,高可信性信源带来的正效果在下降,而低可信性信源带来的负效果却朝向正效果转化。

广告中睡眠效应的现象

(1)重金打造的名人广告,起初确实创造了轰动效应,极大地吸引了人们的眼球。但随着越来越多的广告主采用此策略,受众开始出现了见怪不惊的“审美疲劳”,不再相信“听我的没错”式的简单告知。况且,信息超载环境下的受众早已变得忙碌而又健忘,名人广告见多了,难免张冠李戴起来。这样,就使得广告不得不靠不断地重复来强化效果。

(2)有时一些广告发布后业内人士批评其平庸的创意,受众反映其乏味甚至反感,指责牵强的情节,拙劣的表现,蹩脚的名人登场等,总之,是不好看的广告。但奇妙的是,也许当广告主还在为广告反响不佳懊恼时却惊喜地发现,广告产品的销量却在上升。挨骂的广告居然也能卖产品?广告主无心之下栽了一根“刺”,却意外开出了花。这就要归功于睡眠效应了。

睡眠效应的原因

(1)广告偏好不等于产品品牌偏好。尽管已经证实,广告偏好与其推介的产品品牌偏好之间存在着正相关,但却不是惟一的相关。广告偏好只是人们看广告时产生的评价和情感反应,它明显影响着产品品牌偏好,因喜欢广告而对产品产生好感的情形普遍存在。消费者看广告时可能仅仅在评价广告的好看与否,而在对产品购买决策时进行的却是利弊的权衡。能引起厌烦的广告至少说明被注意到了,遭到异口同声指责的广告有很高的品牌回忆度,有利于进入人们购买决策时的选择域。当产品带来的实际利益高于广告引起的情感好恶时,睡眠效应可能发生了。

(2)时间对信息源的疏远和其他信息的注入。受众对广告的好恶评价更多地具有即时性,随广告展示而产生,但对产品的购买意向和行为却往往是迟效性的。从看广告到决策购买期间,一方面是时间的间隔会使作为信息源的广告被逐渐疏远,当初否定性的情感逐渐淡化;另一方面,这期间会不断有新信息的注入和参与。

(3)因购买经验和品牌忠诚而忽视对广告的消极反应。按照心的归因,消费者评价广告时会利用自己的经验进行主观的解释。假如对某品牌持固有的好感和重复购买的忠诚,看到它的不太令人满意的广告,就可能解释为“谁给他们设计的广告,水平太差了!
\section{顺序效应}
\label{sec-91}


顺序效应(Sequence Effects)是指刺激呈现的顺序影响人们判断的现象。

a.人们回忆经历的时候只是回忆一些片断,而非回忆全部的细节。一般来说,以下几个因素对人们的回忆有很大影响:苦乐顺序的发展趋势,最高和最低点,结尾。

b.人们一般喜欢多次连续进步的体验。比如赌钱,连续赢两次5块钱就比一次赢10块钱更容易让人快乐;相比虎头蛇尾,人们更喜欢鸡头豹尾,即使也许前者总成绩更好。

c.在对两个刺激进行比较时,与刺激的客观顺序实际上并无关系,但当人们出现把通常最先出现的刺激或后面出现的刺激评价过大的倾向称之为顺序效应。如面试考官在对多名考生依次进行评定时,往往会受面试顺序的影响,而不能客观评定考生的情况。例如,一个考官在面试了三个很不理想的考生之后,第四位考生即使很一般,考官也会对他有比前三位好得多的印象。反之,如果一位考官连续面试了三位很理想的考生,即使第四个考生水平一般,考民也会认为他比实际的水平还要差。
\section{Stroop效应}
\label{sec-92}


stroop效应是指字义对命名的干扰效应。一般认为,念字和命名是两个不同的认知过程。Stroop于1935年做了一个实验,他利用的刺激字与书写它们所用的颜色相矛盾,结果发现,说字的颜色时会受到字义的干扰,但在用一年级小学生做实验时却没有发现这个现象。
\section{投射效应}
\label{sec-93}


所谓投射效应是指以己度人,认为自己具有某种特性,他人也一定会有与自己相同的特性,把自己的感情、意志、特性投射到他人身上并强加于人的一种认知障碍。即在人际认知过程中,人们常常假设他人与自己具有相同的特性、爱好或倾向等,常常认为别人理所当然地知道自己心中的想法。比如,一个心地善良的人会以为别人都是善良的;一个经常算计别人的人就会觉得别人也在算计他等等。以小人之心度君子之腹”就是一种典型的投射效应。当别人的行为与我们不同时,我们习惯用自己的标准去衡量别人的行为,认为别人的行为违反常规;喜欢嫉妒的人常常将别人行为的动机归纳为嫉妒,如果别人对他稍不恭敬,他便觉得别人在嫉妒自己。

具体讲,投射效应有以下三种表现:

第一是相同投射。在与陌生人交往时,因为互相不了解,相同投射效应特容易发生,通常在不知不觉中就已然从自我出发做出判断。自己感到热,以为别人也闷热难耐以致客人来了就大放冷气空调;自己家喝酒,招待客人就推杯换盏猛劝酒。有的老师在讲课时,对于某些概念不加说明,以为这是十分简单的基本常识,学生们应该了解和熟悉,但是,在老师看来很简单的东西,在学生看来则不一定简单。这种投射作用发生的主要机制在于忽视自己与对方的差别,在意识中没有把自我和对象区别开来,而是混为一谈,认为他人也跟自己一样,从而合二为一,对对方进行了自己同化。

第二是愿望投射。即把自己的主观愿望加于对方的投射现象。认知主体以为对象正如自己所希望的那样。比如一个自我感觉良好的学生,希望并相信导师对他的论文给以好评,结果他就会把一般性的评语都理解成赞赏的评价。

第三是情感投射。一般来说,人们对自己喜欢的人越看越觉得有很多优点;对自己不喜欢的人,则越看越讨厌,越来越觉得他有很多缺点,令人难以容忍。因而人们总是过度地赞扬和吹捧自己喜爱者,而严厉地指责甚至肆意诽谤自己所厌恶者。这种现象在爱情生活中表现得十分明显。
\section{武器效应}
\label{sec-94}


伯克威茨先让实验助手故意制造挫折情境,激怒实验参加者,然后,实验安排一个机会,让他们可以对激怒自己的实验者实施电击。电击时有两种情境:一种是可以看到桌子上放着一只左轮手枪,一种是只看到一只羽毛球拍。实验结果与研究者的假设相符,即被激怒的人们看到手枪时,比看到羽毛球拍实施了更多的电击。是手枪增强了人们侵犯的行为。后来,人们将武器增强侵犯行为的现象称为“武器效应”。

“武器效应”揭示的是线索对于行为出现的重要作用。同样,在管理中要懂得利用“武器效应”,在需要员工加倍努力完成任务时,让员工看到实在的“奖励”,员工会更有干劲;同样也要让员工看到犯错误的“惩罚”,这样员工会加倍小心。
\section{心理摆效应}
\label{sec-95}


人的感情在受外界刺激的影响下,具有多度性和两极性的特点。每一种情感具有不同的等级,还有着与之相对立的情感状态,如爱与恨、欢乐与忧愁等。“心理摆效应”就是指在特定背景的心理活动过程中,感情的等级越高,呈现的“心理斜坡”就越大,因此也就很容易向相反的情绪状态进行转化,即如果此刻你感到兴奋无比,那相反的心理状态极有可能在另一时刻不可避免地出现。克服这种“心理摆效应”的方法:

1、要消除一些思想上的偏差。人生不能总是高潮,生活也不可能永远是诗。人生有也有散,生活有乐也有苦。有些人由于希望永远生活在激情、浪漫、刺激等理想的境界之中,因而对缺乏上述因素的平凡生活状态总是心存排斥之意,他们的心境自然也就会因生活场景的变化而大起大落。

2、人们应该学会体验各种生活状态的不同乐趣。既能在激荡人心的活动中体验着激情的热烈奔放,又能在平淡如水的日常生活中享受悠然自得的生活情趣。唯有此,自己才能在生活场景中发生较大转换时,避免心理上产生巨大的失落感和消极的情绪。

3、要加强理智对情绪的调控作用。人在让自己快乐兴奋的生活时空中,应该保持适度的冷静和清醒。而当自己转入情绪的低谷时,要尽量避免不停地对比和回顾自己情绪高潮时的“激动画面”,隔绝有关刺激源,把注意力转入到一些能平和自己心境或振奋自己精神的事情和活动当中去。
\section{植物心理学和巴克斯特效应BACKSTER:}
\label{sec-96}


我出生在新泽西州的LAFAYETTE市。我的学业受二战影响而中断,当时我在德科萨斯农机学院读书,是第一个应征入伍的海军军官。在服役前,我对催眠现象极感兴趣,做过许多研究。我对使用催眠术进行情报和反情报方面有一些建议,因为我刚入伍,所以没有引起人们的注意。

二战结束退伍8个月后,我有机会在美国陆军反情报处学习。经过短期培训后,我留在马里兰州的总部担任讲师,教授情报调查课程。那时,我的催眠术在情报和反情报方面的应用开始引起人们的重视。我开始有自己的办公室,关起窗子进行专门的研究,但进展十分缓慢。有一次,我曾为安全起见,给司令官的秘书使用了催眠术,取得了绝密文件。那个秘书醒来后也没有察觉。为保密起见,当天晚上我将文件加以安全保管。第二天,我将文件交给司令官,说:一个是通知特警逮捕我,一个是认真地听我解释。后来他们仔细地听取了我的解释。当时正值中央情报局刚刚成立,听到这件事后,中央情报局让我提前退伍,雇用我从事使用测谎仪进行背景调查方面的工作。

测谎仪象是一种用于情报工作的特殊仪器,它也好象很自然地联系到了我的催眠﹑夜游这些曾研究过的领域。我试想将它们用于我的工作。但实际上,对在中央情报局从事的工作,我感到乏味。因为当时中央情报局刚刚成立,许多人都希望在FBI找到一份象样的工作。我们对XXXXX的人进行测谎检测。将通过检测的人,列入合格名单内。再进行工种调查,以分派合适的工作。我辞掉了政府工作,开始专门从事测谎仪的研究。我在DC设立了自己的实验室,在BOLTMORE扩建了另一个办公室,第三个在纽约。实验室最终在纽约固定下来,并于1959年和我的同行,测谎仪行业的竞争对手一起成立了学校。当时的全美第一所学校教授使用测谎仪。学校成立后,需要使用大量的测谎仪。而测谎仪的1/3部份是肤电反应器,用图线来反应人的情绪变化等。

这是一张标准的测试图,使用模拟的单针心电记录器,画出的有关血压﹑脉搏强度变化。上面的记录是呼吸曲线。我感兴趣的是电子曲线,即肤电反应器画出的曲线。我在给植物浇水,是一盆牛舌兰花。这盆花有一株长长的叶杆,叶子也是长长的,当时它可能生长了1年到1年半的时间,是我的秘书从楼下一个花店买来的,那家店要关门不做生意了,当时我们还买了一颗橡胶树。在浇水时我很好奇,我想知道在根部的水份将花多长时间,穿过长长的叶杆最终到达叶子的顶尖。我想:好阿,正好这些测谎设备可以用来测量它的电阻变化,还能测量出它的肤电感应。把它的叶子连上电极看看会怎样?因为当水份到达叶尖时,夹在电极中间的叶子的导电性能会增强。

在浇水后,我想我将看到画出的曲线会呈现向上的趋势,因为当水份到达后,电阻会变小。出乎意料,曲线的趋势却呈现着不断向下,我把指针移到了上端,曲线连续向下滑。在这里的这段曲线的形状引起了我的注意,如果是肤电反应,我们会解释这段曲线代表着情绪波动,这段曲线代表着情绪恢复,随后整体呈现向下的趋势。这是整个图的局部,这张是一幅标准的呈现向下的肤电图。我意识到这段局部的曲线形状,显示了和人相同的情绪反应,我当时真是吃了一惊。
\section{自己人效应}
\label{sec-97}


有一种效应叫自己人效应,就是说要使对方接受你的观点、态度,你就不惜同对方保持同体观的关系,也就是说,要把对方与自己视为一体。管理心理学中有句名言:''如果你想要人们相信你是对的,并按照你的意见行事,那就首先需要人们喜欢你,否则,你的尝试就会失败。''

冯玉祥将军在他的“丘八诗”中号召士兵:“重层压迫均推倒,要使平等现五洲。”他热爱体贴士兵,关心他们的生活,曾亲自为伤兵尝汤药,擦身搓背,甚至和士兵一样吃粗茶淡饭。所以,士兵们都感到冯将军没有架子,与自己处于平等地位,因而都尊重和听他的话,有什么想不通的事都愿意找他说。

说服别人按照你的建议去做,只是向人们提出好建议是远远不够的,可以强化和发挥“自己人效应”,让人们喜欢你。避免好的建议遭到拒绝。

首先,应强调双方一致的地方,使对方认为你是“自己人”,从而使你提出的建议易于被接受。

其次,努力使双方处于平等的地位。你要想取得对方的信赖,先得和对方缩短心理距离,与之处于平等地位,这样就能提高你的人际影响力。

再次,要有良好的个性品质。人的良好个性品质是增强人际影响力的重要因素。心理学研究证明:具备开朗、坦率、大度、正直、实在等良好个性品质的人,人际影响力就强;反之,有傲慢、以自我为中心、言行不一、欺下媚上、嫉贤妒能、斤斤计较等不良个性品质的人,是最不受欢迎的人,也就没有人际影响力可言。所以,我们每个人要加强良好个性品质修养,以增强自己的人际影响力。
\section{紫格尼克效应}
\label{sec-98}


你不妨试一下:一笔画个圆圈,在交接处有意留出一小段空白。回头再瞧一下这个圆吧,此刻你脑子里必定会闪现出要填补这段空白弧形的意念。因为你总有一种出于未完成感的心态,竭力寻求终结途径,以获得心理上的满足。

有一位叫布鲁玛.紫格尼克的心理学家,她给128个孩子布置了一系列作业,她让孩子们完成一部份作业,另一部份则令其中途停顿。一小时后测试结果。110个孩子对中途停顿的作业记忆犹新。紫格尼克的结论是:人们对业已完成的工作较为健忘,因为“完成欲”已经得到满足,而对未完成的工作则在脑海里萦绕不已。这就是所谓的‘紫格尼克效应”。

“紫格尼克效应”的心理机制是什么呢?被誉为现代社会心理学之父的德国心理学家勒温认为,人类有一种自然倾向去完成一个行为单位,如去解答一个谜语,学习一本书等,这就叫“心理张力”。研究还指出,任何人都企图满足自己的需要,完成动作。其中既有先天的需要(饥、渴等),也有半需要(迫切的趋向)。在勒温看来,个人能动性的源泉是多元的,形形色色的。被唤起但未得到满足的心理需要产生一个张力系统,决定着个人行为的倾向、心理的基调和特点。如果中断了满足需要的过程或解决某项任务的进程而产生了张力系统,就可以使一个人采取达到目标的行动。勒温认为,没有完成的任务使得没有解决的张力系统永远存在,当任务完成之后,与之并存的张力系统也将随之消失。由此可见,一个人的“心理张力”系统,是产生 “紫格尼克效应”的心理机制。

\end{document}