% Created 2017-07-04 二 08:50
\documentclass{ctexart}
\usepackage[utf8]{inputenc}
\usepackage{abstract}
\usepackage{geometry}
\geometry{left=2.5cm,right=2.5cm,top=2.5cm,bottom=2.5cm}
\usepackage{flafter}
\setlength{\headheight}{15pt}
\usepackage[title]{appendix}
\usepackage{fixltx2e}
\usepackage{graphicx}
\usepackage{longtable}
\usepackage{float}
\usepackage{wrapfig}
\usepackage{soul}
\usepackage{textcomp}
\usepackage{marvosym}
\usepackage{wasysym}
\usepackage{latexsym}
\usepackage{amssymb}
\usepackage{bpchem}
\usepackage{mhchem}
\usepackage{listings}
\usepackage{xcolor}
\usepackage{multirow,array,multicol,indentfirst}
\usepackage{SIunits,fancyhdr}
\usepackage{tikz,pifont,footnote}
\usepackage[colorlinks,linkcolor=blue,anchorcolor=blue,citecolor=green]{hyperref}
\usepackage{enumerate,comment}
\usepackage{lastpage}
\usepackage{layout}
\newtheorem{thm}{{定理}}
\newtheorem{proposition}{{命题}}
\newtheorem{lemma}{{引理}}
\newtheorem{corollary}{{推论}}
\newtheorem{definition}{{定义}}
\newtheorem{rules}{{规则}}
\newtheorem{suggest}{{建议}}
\newtheorem{example}{{例}}
\CTEXsetup[format={\raggedright}]{section}
\CTEXsetup[format={\raggedright}]{subsection}
\CTEXsetup[format={\raggedright}]{subsubsection}
\pagestyle{fancy}
\author{六盘水豆荚信息技术服务有限公司}
\date{}
\title{财务管理制度}
\hypersetup{
  pdfkeywords={},
  pdfsubject={},
  pdfcreator={Emacs 25.2.1 (Org mode 8.2.10)}}
\begin{document}

\maketitle
\tableofcontents


\section{总则}
\label{sec-1}
\subsection{为了加强公司的财务工作,完善公司的财务管理,强化内部控制,规范财务管理流程,根据国家现行的有关财经法规、财务制度、政策结合本公司的实际情况,制定本制度。}
\label{sec-1-1}
\subsection{财务管理的目标是追求企业价值最大化。}
\label{sec-1-2}
\subsection{财务管理的职能是对企业经济活动进行记录、反映、预算、控制、决策。}
\label{sec-1-3}
\subsection{财务管理的内容包括:}
\label{sec-1-4}
\subsubsection{资金管理,统一监控、调剂单位内部货币资金流动,提高资金使用率,掌握资金动态,为管理决策提供参谋。}
\label{sec-1-4-1}
\subsubsection{收支管理,按照财务管理制度规定的流程、标准和范围等,对涉及财务收支的经济业务实施规范管理。加强内部控制,推动预算管理。}
\label{sec-1-4-2}
\subsubsection{财产管理,对企业库存现金、存货、固定资产等的保存、管理、使用、处置进行规范,明确各环节责任,保证企业财产安全完整,并发挥应有效用。}
\label{sec-1-4-3}
\subsubsection{成本与费用管理,针对企业经营特点,以先进管理思想为指导,规范控制成本费用开支,为完善内部管理提供依据。}
\label{sec-1-4-4}
\subsection{财务部门构架与职责}
\label{sec-1-5}
\subsection{财务总监的岗位职责}
\label{sec-1-6}
\section{审批权限}
\label{sec-2}
\subsection{公司的各项费用支出、采购种类物品的款项,由经办人部门经理(主管)审批、财务部部长核对、财务总监审批后,再由总经理批准后方可报销。}
\label{sec-2-1}
\subsection{所有付款申请表连同真实完整合法的原始单据,上交财务部后(除供应商贷款外),报销审批支付时限为五个工作日。}
\label{sec-2-2}
\subsection{固定资产、无形资产购置的审批权限:}
\label{sec-2-3}
\subsubsection{固定资产分类}
\label{sec-2-3-1}
\begin{enumerate}
\item 房屋、建筑物:营业用房、非营业用房、简易房、建筑物。
\label{sec-2-3-1-1}
\item 机械设备:生产设备、检测设备为、维修设备、其他机器设备。
\label{sec-2-3-1-2}
\item 交通运输工具。
\label{sec-2-3-1-3}
\item 办公设备:办公家具、办公设备。
\label{sec-2-3-1-4}
\item 电子设备:照相机、电子计算机系统设备、复印机及打印机设备等。
\label{sec-2-3-1-5}
\end{enumerate}
\subsubsection{无形资产:包括电脑软件、专利权、非专利技术、土地使用权、特许权、商誉等。}
\label{sec-2-3-2}
\subsubsection{审批权限:经部门负责人审核后,财务总监审批后,再由总经理批准。}
\label{sec-2-3-3}
\subsection{费用报销审批程序:报销人员填写报销单 $\to$ 部门负责人审核真实、完整、无误 $\to$ 财务负责人审核单据、金额无误 $\to$ 财务总监、总经理批准 $\to$ 出纳付款。总经理报销的费用需要报总裁(或总裁授权人)批准。}
\label{sec-2-4}
\subsection{没有按完整的审批程序执行,出纳应拒绝付款。如果审核(批)人员不在公司,通过邮件或微信审批的,应打印邮件或截图附后;口头或电话确认的,事后应及时补签。}
\label{sec-2-5}
\subsection{财务部门负责人单据审核包括:}
\label{sec-2-6}
\subsubsection{报销单填写是否规范,内容是否填写完整、小写报销金额前是否有写¥、大写金额前是否有封 $\phi$}
\label{sec-2-6-1}
\section{收支管理}
\label{sec-3}

\section{财产管理}
\label{sec-4}

\section{销售与收款业务管理}
\label{sec-5}

\section{购货与付款业务管理}
\label{sec-6}

\section{成本与费用管理}
\label{sec-7}

\section{原始凭证与财务档案管理}
\label{sec-8}

\section{财务责任追究制度}
\label{sec-9}

\section{附则}
\label{sec-10}
% Emacs 25.2.1 (Org mode 8.2.10)
\end{document}
%%% Local Variables:
%%% mode: latex
%%% TeX-master: t
%%% End:
